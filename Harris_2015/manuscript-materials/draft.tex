\documentclass[12pt,]{article}
\usepackage[compact]{titlesec}
\usepackage{lmodern}
\usepackage{amssymb,amsmath}
\usepackage{ifxetex,ifluatex}
\usepackage{fixltx2e} % provides \textsubscript
\ifnum 0\ifxetex 1\fi\ifluatex 1\fi=0 % if pdftex
  \usepackage[T1]{fontenc}
  \usepackage[utf8]{inputenc}
\else % if luatex or xelatex
  \ifxetex
    \usepackage{mathspec}
    \usepackage{xltxtra,xunicode}
  \else
    \usepackage{fontspec}
  \fi
  \defaultfontfeatures{Mapping=tex-text,Scale=MatchLowercase}
  \newcommand{\euro}{€}
\fi
% use upquote if available, for straight quotes in verbatim environments
\IfFileExists{upquote.sty}{\usepackage{upquote}}{}
% use microtype if available
\IfFileExists{microtype.sty}{%
\usepackage{microtype}
\UseMicrotypeSet[protrusion]{basicmath} % disable protrusion for tt fonts
}{}
\usepackage{setspace}
\usepackage{lineno}
\linenumbers
\usepackage{setspace}
\usepackage{geometry}
\geometry{verbose,letterpaper,tmargin=2.54cm, bmargin=2.54cm,lmargin=2.54cm,rmargin=2.54cm}
\ifxetex
  \usepackage[setpagesize=false, % page size defined by xetex
              unicode=false, % unicode breaks when used with xetex
              xetex]{hyperref}
\else
  \usepackage[unicode=true]{hyperref}
\fi
\hypersetup{breaklinks=true,
            bookmarks=true,
            pdfauthor={},
            pdftitle={},
            colorlinks=true,
            citecolor=blue,
            urlcolor=blue,
            linkcolor=magenta,
            pdfborder={0 0 0}}
\urlstyle{same}  % don't use monospace font for urls
\setlength{\parindent}{0pt}
\setlength{\parskip}{4pt}
\setlength{\emergencystretch}{3em}  % prevent overfull lines
\setcounter{secnumdepth}{0}

\date{}

\begin{document}


\begin{spacing}{1.9}
\begin{flushleft}
\emph{Running head:} Species interactions in Markov networks

\textbf{Title: Estimating species interactions from observational data
with Markov networks}

\textbf{Author:} David J. Harris: Population Biology; 1 Shields Avenue,
Davis CA, 95616

\textbf{Abstract:} Estimating species interactions from observational
data is one of the most controversial tasks in community ecology. One
difficulty is that a single pairwise interaction can ripple through an
ecological network and produce surprising indirect consequences. For
example, two competing species would ordinarily correlate negatively in
space, but this effect can be reversed in the presence of a third
species that is capable of outcompeting both of them when it is present.
Here, I apply models from statistical physics, called Markov networks or
Markov random fields, that can predict the direct and indirect
consequences of any possible species interaction matrix. Interactions in
these models can be estimated from observational data via maximum
likelihood. Using simulated landscapes with known pairwise interaction
strengths, I evaluated Markov networks and several existing approaches.
The Markov networks consistently outperformed other methods, correctly
isolating direct interactions between species pairs even when indirect
interactions or abiotic environmental effects largely overpowered them.
Two computationally efficient approximations, based on linear and
generalized linear models, also performed well. Indirect effects
reliably caused a common null modeling approach to produce incorrect
inferences, however.

\textbf{Key words:} Ecological interactions; Occurrence data; Species
associations; Markov network; Markov random field; Ising model;
Biogeography; Presence--absence matrix; Null model

\setlength{\parindent}{1cm} \setlength{\parskip}{6pt}

\subsubsection{Introduction}\label{introduction}

If nontrophic species interactions, such as competition, are important
drivers of community assembly, then ecologists might expect to see their
influence in our data sets (MacArthur 1958, Diamond 1975). Despite
decades of work and several major controversies, however (Lewin 1983,
Strong et al. 1984, Gotelli and Entsminger 2003, Connor et al. 2013),
existing methods for detecting competition's effects on community
structure are unreliable (Gotelli and Ulrich 2009). In particular,
species' effects on one another can become lost in the complex web of
direct and indirect interactions in real assemblages. For example, the
competitive interaction between the two shrub species in Figure 1A can
become obscured by their shared tendency to occur in unshaded areas
(Figure 1B). If this sort of effect is common, then the vast majority of
ecologists' methods, which rely on test statistics describing the total
number of co-occurrences or overall correlation between putatively
interacting species (Diamond 1975, Strong et al. 1984, Gotelli and
Ulrich 2009, Veech 2013, Pollock et al. 2014) will not generally provide
much evidence regarding species' direct effects on one another.

While competition doesn't reliably leave the expected pattern in
community structure (a low co-occurrence rate at the landscape level),
it nevertheless does leave a signal in the data (Figure 1C).
Specifically, \emph{among shaded sites}, there will be a deficit of
co-occurrences between the competing shrub species, and \emph{among
unshaded sites}, there will also be such a deficit. More generally, we
can obtain much better estimates of the association between two species
from their conditional relationships (i.e.~by controlling for other
species in the network) than we could get from their overall
co-occurrence rates. This kind of precision is difficult to obtain from
null models, which begin with the assumptions that all the pairwise
interactions are zero and thus don't need to be controlled for.
Nevertheless, null models have dominated this field for more than three
decades (Diamond 1975, Strong et al. 1984, Gotelli and Ulrich 2009).

In this paper, following recent work by Azaele et al. (2010) and Fort
(2013), I show that Markov networks (undirected graphical models also
known as Markov random fields; Murphy 2012) can provide a framework for
understanding the landscape-level consequences of pairwise species
interactions, and for detecting them from observed presence-absence
matrices. Markov networks have been used in many scientific fields for
decades in similar contexts, from physics (where nearby particles
interact magnetically with one another; Cipra 1987) to spatial
statistics (where adjacent grid cells interact with one another; Harris
1974, Gelfand et al. 2005). While community ecologists explored some
related approaches in the 1980's (Whittam and Siegel-Causey 1981),
computational limitations had previously imposed severe approximations
that produced unintelligible results (e.g. ``probabilities'' greater
than one; Gilpin and Diamond 1982). Now that it is computationally
feasible to fit these models exactly, this kind of approach is worth a
second look.

The rest of the paper proceeds as follows. First, I discuss how Markov
networks operate and how they can be used to simulate landscape-level
data or to predict the direct and indirect consequences of possible
interaction matrices. Then, using simulated data sets where the ``true''
ecological structure is known, I compare this approach with several
existing methods for detecting species interactions. Finally, I discuss
opportunities for extending the approach presented here to larger
problems in community ecology.

\subsubsection{Methods}\label{methods}

\paragraph{Conditional relationships and Markov
networks.}\label{conditional-relationships-and-markov-networks.}

Ecologists are often interested in learning about direct interactions
between species, controlling for the indirect influence of other
species. In statistical terms, this implies that ecologists want to
estimate \emph{conditional} (``all-else-equal'') relationships, rather
than \emph{marginal} (``overall'') relationships. The example with the
shrubs and trees in Figure 1 shows how the two measures can have
opposite signs, and suggests that conditional relationships are more
relevant for drawing inferences about species interactions
(e.g.~competition).

Markov networks define species interactions in conditional terms, and
use these conditional relationships to determine whether a given
combination of co-occurring species is consistent with their
interactions. Given a set of binary outcomes for each species (i.e.~0
for absence and 1 for presence), the Markov network defines the relative
probability of observing a given presence-absence vector, \(\vec{y}\),
as

\centering
\(p(\vec{y}; \alpha, \beta) \propto exp(\sum_{i}\alpha_i y_i + \sum_{i\neq j}\beta_{ij}y_i y_j).\)
\raggedright

Here, \(\alpha_{i}\) is an intercept term determining the amount that
the presence of species \(i\) contributes to the log-probability of
\(\vec{y}\); it directly controls the prevalence of species \(i\).
Similarly, \(\beta_{ij}\) is the amount that the co-occurrence of
species \(i\) and species \(j\) contributes to the log-probability; it
controls the probability that the two species will be found together
(Figure 2A, Figure 2B). Because the relative probability of a
presence-absence vector increases when positively-associated species
co-occur and decreases when negatively-associated species co-occur, the
model tends to produce assemblages that have many pairs of
positively-associated species and relatively few pairs of
negatively-associated species (exactly as an ecologist might expect).

A major benefit of Markov networks is the fact that the conditional
relationships between species can be read directly off the matrix of
\(\beta\) coefficients (Murphy 2012). For example, if the coefficient
linking two mutualist species is \(+2\), then---all else equal---the
odds of observing either species increase by a factor of \(e^2\) when
its partner is present (Murphy 2012). In this way, the \(\beta\)
coefficients behave very similarly to the coefficients in a logistic
regression model (Lee and Hastie 2012).

Of course, if all else is \emph{not} equal (e.g.~Figure 1, where the
presence of one competitor is associated with release from another
competitor), then species' marginal association rates can differ from
this expectation. For this reason, it is important to consider how
coefficients' effects propagate through the network, as discussed below.

Estimating the marginal relationships predicted by a Markov network is
more difficult than estimating conditional relationships, because doing
so requires absolute probability estimates. Turning the relative
probability given by Equation 1 into an absolute probability entails
scaling by a normalizing constant, \(Z\), so that that the probabilities
of all possible assemblages will sum to one (bottom of Figure 2B).
Calculating \(Z\) exactly, as is done in these analyses, quickly becomes
infeasible as the number of species increases: with \(2^N\) possible
assemblages of \(N\) species, the number of bookkeeping operations
required spirals exponentially into the billions and beyond. Numerous
techniques are available for keeping the calculations tractable,
e.g.~via analytic approximations (Lee and Hastie 2012) or Monte Carlo
sampling (Salakhutdinov 2008), but they are beyond the scope of this
paper. Alternatively, as noted below, some common linear and generalized
linear methods can also be used as computationally efficient
approximations to the full network.

\paragraph{\texorpdfstring{Estimating \(\alpha\) and \(\beta\)
coefficients from presence-absence
data.}{Estimating \textbackslash{}alpha and \textbackslash{}beta coefficients from presence-absence data.}}\label{estimating-alpha-and-beta-coefficients-from-presence-absence-data.}

In the previous two sections, the values of \(\alpha\) and \(\beta\)
were known. In practice, however, ecologists will often need to estimate
these parameters from co-occurrence data (i.e.~from a matrix of ones and
zeros indicating which species are present at which sites). When the
number of species is reasonably small, one can compute exact maximum
likelihood estimates for all of the \(\alpha\) and \(\beta\)
coefficients given a presence-absence matrix by optimizing
\(p(\vec{y}; \alpha, \beta)\). Fully-observed Markov networks like the
ones considered here have unimodal likelihood surfaces (Murphy 2012),
ensuring that this procedure will always converge on the global maximum.
This maximum represents the unique combination of \(\alpha\) and
\(\beta\) coefficients that would be expected to produce exactly the
observed co-occurrence frequencies (i.e.~it matches the sufficient
statistics of the model distribution to the sufficient statistics of the
data). For the analyses in this paper, I used the \emph{rosalia} package
(Harris 2015a) for the R programming language (R Core Team 2015) to
define the objective function and gradient in R code and then find the
optimum.

\paragraph{Simulated landscapes.}\label{simulated-landscapes.}

In order to compare different methods, I simulated two sets of
landscapes using known parameters.

The first set of simulated landscapes included the three competing
species shown in Figure 1. For each of 1000 replicates, I generated a
landscape with 100 sites by sampling exactly from a probability
distribution defined by the interaction coefficients in that figure
(Appendix A). Each of the methods described below was then evaluated on
its ability to correctly infer that the two shrub species competed with
one another, despite their frequent co-occurrence.

I also simulated a second set of landscapes using a stochastic community
model based on generalized Lotka-Volterra dynamics, as described in the
Appendix. In these simulations, each species pair was randomly assigned
to either compete for a portion of the available carrying capacity
(negative interaction) or to act as mutualists (positive interaction).
In these simulations, mutualism operated by mitigating the effects of
intraspecific competition on each partner's death rate. For these
analyses, I simulated landscapes with up to 20 species and 25, 200, or
1600 sites (50 replicates per landscape size; see Appendix).

\paragraph{Recovering species interactions from simulated
data.}\label{recovering-species-interactions-from-simulated-data.}

I used Markov networks and several other techniques for determining the
sign and strength of the associations between pairs of species (Appendix
B).

For the Markov networks, I used the rosalia package (Harris 2015a), as
described above. Given the large number of parameters associated with
some of the networks to be estimated, I regularized the likelihood using
a logistic prior distribution with a scale of 2 on the \(\alpha\) and
\(\beta\) terms. This convex prior distribution has a similar shape to
the Laplace distribution used in LASSO regularization (especially in the
tails), but does not force the estimates to be exactly zero.

I also evaluated five alternative methods from the existing literature
to calibrate the performance of the Markov network as a method for
estimating species interactions. The first two alternative interaction
measures were the sample correlations and the partial correlations
between each pair of species' data vectors on the landscape (Albrecht
and Gotelli 2001, Faisal et al. 2010). Because partial correlations are
undefined for landscapes with perfectly-correlated species pairs, I used
a regularized estimate based on James-Stein shrinkage, as implemented in
the corpcor package's \texttt{pcor.shrink} function with the default
settings ({\textbf{???}}). In the context of non-Gaussian data, the
partial correlation can be thought of as a computationally efficient
approximation to the full Markov network model (Loh and Wainwright
2013).

The third alternative, generalized linear models (GLMs), can also be
thought of as a computationally efficient approximation to the Markov
network (Lee and Hastie 2012). Following Faisal et al. (2010), I fit
regularized logistic regression models for each species, using the other
species on the landscape as predictors. {[}{[}Observational
presence-absence data does not have enough degrees of freedom to
estimate this many parameters, so I symmetrized the pairwise
correlations by averaging the coefficient for predicting species \(i\)'s
status from species \(j\) and vice versa.{]}{]}

The fourth alternative method, described in Gotelli and Ulrich (2009),
involved simulating possible landscapes from a null model that retains
the row and column sums of the original matrix (Strong et al. 1984).
Using the default options in the Pairs software described in Gotelli and
Ulrich (2009), I simulated the null distribution of scaled C-scores (a
test statistic describing the number of \emph{non}-co-occurrences
between two species). The software then calculated a \(Z\) statistic for
each species pair using this null distribution. After multiplying this
statistic by \(-1\) so that positive values corresponded to facilitation
and negative values corresponded to competition, I used it as another
estimate of species interactions.

The last two methods used the latent correlation matrix estimated by the
BayesComm package ({\textbf{???}}) to evaluate the claim from recent
papers that the correlation coefficients estimated by ``joint species
distribution models'' provide an accurate assessment of species'
pairwise interactions (Pollock et al. 2014, see also Harris 2015b). I
reported both the average correlation between each pair of species'
latent variables across 1000 Monte Carlo samples, as well as the average
\emph{partial} correlation.

\paragraph{Evaluating model
performance.}\label{evaluating-model-performance.}

For the simulated landscapes based on Figure 1, method evaluation was
fairly qualitative: any method whose test statistic for the two shrubs
indicated a negative relationship passed; other methods failed. For
methods that provide confidence intervals or other inferential
statistics (Appendix), I also reported how often they rejected the null
hypothesis of no interaction.

Because the different methods mostly describe species interactions on
different scales (e.g.~correlations versus \(Z\) scores versus
regression coefficients), I used linear regression through the origin to
rescale the different estimates produced by each method so that they had
a consistent interpretation. After rescaling each method's estimates, I
calculated squared errors between the scaled interaction estimates and
``true'' interaction values for each of the {[}{[}N{]}{]} species pairs
across all the simulated data sets. These squared errors determined the
proportion of variance explained for different combinations of method
and landscape size (compared with a baseline model that assumed all
interaction strengths to be zero).

\subsubsection{Results}\label{results}

\paragraph{Three species.}\label{three-species.}

As shown in Figure 1, the marginal relationship between the two shrub
species was positive---despite their competition for space at a
mechanistic level---due to indirect effects of the dominant tree
species. As a result, the covariance method falsely reported positive
associations for 94\% of the simulated landscapes, and the
randomization-based null model falsely reported such associations 100\%
of the time. The methods for evaluating conditional relationships
(Markov networks, GLMs, and both types of partial covariance estimates),
however, successfully controlled for the indirect pathway via the tree
species and each correctly identified the direct negative interaction
between the shrubs 94\% of the time.

\paragraph{Twenty species.}\label{twenty-species.}

The accuracy of the four evaluated methods varied substantially,
depending on the parameters that produced the simulated communities
(Figure 3). In general, however, there was a consistent ordering\ldots{}

The three methods that estimated marginal relationships among species
and did not explicitly control for indirect species interactions (the
null model, the joint species distribution model's correlation
coefficients, and the sample correlation coefficients) differed from one
another in their \(R^2\) values, but their estimates were surprisingly
similar to one another. In particular, the correlation coefficients were
extremely similar to the estimates from the other two methods
(Spearman's rho = X for the joint species distribution model's
correlation matrix and Y for the null model).

\subsubsection{Discussion}\label{discussion}

The results presented above show that Markov networks can reliably
recover species' pairwise interactions from observational data, even for
cases where a common null modeling technique reliably fails.
Specifically, Markov networks were successful even when direct
interactions were largely overwhelmed by indirect effects (Figure 1).
For cases where fitting a Markov network is computationally infeasible,
these results also indicate that partial covariances and generalized
linear models (the two methods that estimated conditional relationships
rather than marginal ones) can both provide useful approximations. The
partial correlations' success on simulated data may not carry over to
real data sets, however; Loh and Wainwright (2013) show that the linear
approximations can be less reliable in cases where the true interaction
matrix contains more structure (e.g.~guilds or trophic levels).
Similarly, the approximation involved in using separate generalized
linear models for each species can occasionally lead to catastrophic
overfitting with small-to-moderate sample sizes (Lee and Hastie 2012).
For these reasons, it will usually be best to fit a Markov network
rather than one of the alternative methods when one's computational
resources allow it.

It's important to note that none of these methods can identify the exact
nature of the pairwise interactions (e.g.~which species in a
positively-associated pair is facilitating the other), particularly when
real pairs of species can reciprocally influence one another in multiple
ways simultaneously (Bruno et al. 2003); with compositional data, there
is only enough information to provide a single number describing each
species pair.\\On the other hand, if ecologists are familiar enough with
the natural history of their study systems that they can augment the
likelihood function with a prior distribution or a suitable penalty,
then this information could reduce the effective degrees of freedom to
estimate and real-world results might be even better than those shown in
Figure 3.

Despite these limitations, Markov networks have enormous potential to
improve ecological understanding. For example, Markov networks provide a
simple answer to the question of how competition should affect a
species' overall prevalence, which was a major flashpoint for the null
model debates in the 1980's (Roughgarden 1983, Strong et al. 1984).
Equation 1 can be used to calculate the expected prevalence of a species
in the absence of biotic influences (\(\frac{1}{1 + e^{-\alpha}}\); Lee
and Hastie 2012). Competition's effect on prevalence in a Markov network
can then be calculated by subtracting this value and the observed
prevalence (cf Figure 2D).

Markov networks---particularly the Ising model for binary
networks---have been studied in statistical physics for nearly a century
(Cipra 1987), and the models' properties, capabilities, and limits are
well-understood in a huge range of applications. Modeling species
interactions using the same framework would thus allow ecologists to tap
into an enormous set of existing discoveries and techniques for dealing
with indirect effects, stability, and alternative stable states.
Numerous other extensions are possible: Markov networks can be fit with
a mix of discrete and continuous variables, for example (Lee and Hastie
2012). There are even methods (Whittam and Siegel-Causey 1981,
Tjelmeland and Besag 1998) that would allow the coefficient linking two
species in an interaction matrix to vary as a function of the abiotic
environment or of third-party species that tip the balance between
facilitation and exploitation (Bruno et al. 2003).

Finally, the results presented here have important implications for
ecologists' continued use of null models to draw inferences about
species interactions. Null and neutral models can be very useful for
clarifying our thinking about the numerical consequences of species'
richness and abundance patterns (Harris et al. 2011, Xiao et al. 2015),
but deviations from a null model must be interpreted with care
(Roughgarden 1983). In complex networks of ecological interactions (and
even in small networks with three species), it may simply not be
possible to implicate individual species pairs or specific ecological
processes like competition by rejecting a general-purpose null (Gotelli
and Ulrich 2009). Moreover, the null model's estimates were extremely
similar to what could be obtained from a simple correlation matrix,
which raises questions about how much additional value the null model
adds. Estimating pairwise coefficients directly seems like a much more
promising approach: to the extent that the models' relative performance
on real data sets is similar to the range of results shown in Figure 3,
scientists in this field could {[}{[}easily double their explanatory
power by switching from null models to partial correlations or
generalized linear models, or triple it by switching to a Markov
network{]}{]}. Regardless of the specific methods ecologists ultimately
choose, the most important consideration is clearly that we need to
control for indirect effects and estimate conditional relationships
between species, rather than marginal ones.

\paragraph{Acknowledgements:}\label{acknowledgements}

This research was funded by a Graduate Research Fellowship from the US
National Science Foundation and benefited greatly from discussions with
A. Sih, M. L. Baskett, R. McElreath, R. J. Hijmans, A. C. Perry, and C.
S. Tysor. Additionally, A. K. Barner, E. Baldridge, E. P. White, D. Li,
D. L. Miller, N. Golding, N. J. Gotelli, and C. F. Dormann provided
useful feedback on earlier drafts of this work.

\setlength{\parindent}{0cm} \setlength{\parskip}{4pt}

\textbf{References:}

\textbf{Figure captions}

\textbf{Figure 1.} \textbf{A.} A small network of three competing
species. The tree (top) tends not to co-occur with either of the two
shrub species, as indicated by the strongly negative coefficient linking
them. The two shrub species also compete with one another, as indicated
by their negative coefficient (circled), but this effect is
substantially weaker. \textbf{B.} In spite of the competitive
interactions between the two shrub species, their shared tendency to
occur in locations without trees makes their occurrence vectors
positively correlated (circled). \textbf{C.} Controlling for the tree
species' presence with a conditional method such as a partial covariance
or a Markov network allows us to correctly identify the negative
interaction between these two species (circled).

\textbf{Figure 2.} \textbf{A.} A small Markov network with two species.
The depicted abiotic environment favors the occurrence of both species
(\(\alpha >0\)), particularly species 2 (\(\alpha_2 > \alpha_1\)). The
negative \(\beta\) coefficient linking these two species implies that
they co-occur less than expected under independence. \textbf{B.}
Relative probabilities of all four possible presence-absence
combinations for Species 1 and Species 2. The exponent includes
\(\alpha_1\) whenever Species 1 is present (\(y_1 = 1\)), but not when
it is absent (\(y_1 = 0\)). Similarly, the exponent includes
\(\alpha_2\) only when species \(2\) is present (\(y_2 = 1\)), and
\(\beta\) only when both are present (\(y_1y_2 = 1\)). The normalizing
constant \(Z\), ensures that the four relative probabilities sum to 1.
In this case, \(Z\) is about 18.5. \textbf{C.} Using the probabilities,
we can find the expected frequencies of all possible co-occurrence
patterns between the two species of interest. \textbf{D.} If \(\beta\)
equaled zero (e.g.~if the species no longer competed for the same
resources), then the reduction in competition would allow each species
to increase its occurrence rate and the deficit of co-occurrences would
be eliminated.

\textbf{Figure 3.} Proportion of variance in interaction coefficients
explained by each method across varying numbers of sampled locations.

Albrecht, M., and N. J. Gotelli. 2001. Spatial and temporal niche
partitioning in grassland ants. Oecologia 126:134--141.

Azaele, S., R. Muneepeerakul, A. Rinaldo, and I. Rodriguez-Iturbe. 2010.
Inferring plant ecosystem organization from species occurrences. Journal
of theoretical biology 262:323--329.

Bruno, J. F., J. J. Stachowicz, and M. D. Bertness. 2003. Inclusion of
facilitation into ecological theory. Trends in Ecology \& Evolution
18:119--125.

Cipra, B. A. 1987. An introduction to the Ising model. American
Mathematical Monthly 94:937--959.

Connor, E. F., M. D. Collins, and D. Simberloff. 2013. The checkered
history of checkerboard distributions. Ecology 94:2403--2414.

Diamond, J. M. 1975. The island dilemma: Lessons of modern biogeographic
studies for the design of natural reserves. Biological conservation
7:129--146.

Faisal, A., F. Dondelinger, D. Husmeier, and C. M. Beale. 2010.
Inferring species interaction networks from species abundance data: A
comparative evaluation of various statistical and machine learning
methods. Ecological Informatics 5:451--464.

Fort, H. 2013. Statistical Mechanics Ideas and Techniques Applied to
Selected Problems in Ecology. Entropy 15:5237--5276.

Gelfand, A. E., A. M. Schmidt, S. Wu, J. A. Silander, A. Latimer, and A.
G. Rebelo. 2005. Modelling species diversity through species level
hierarchical modelling. Journal of the Royal Statistical Society: Series
C (Applied Statistics) 54:1--20.

Gilpin, M. E., and J. M. Diamond. 1982. Factors contributing to
non-randomness in species Co-occurrences on Islands. Oecologia
52:75--84.

Gotelli, N. J., and G. L. Entsminger. 2003. Swap algorithms in null
model analysis. Ecology:532--535.

Gotelli, N. J., and W. Ulrich. 2009. The empirical Bayes approach as a
tool to identify non-random species associations. Oecologia
162:463--477.

Harris, D. J. 2015a. Rosalia: Exact inference for small binary Markov
networks. R package version 0.1.0. Zenodo.
http://dx.doi.org/10.5281/zenodo.17808.

Harris, D. J. 2015b. Generating realistic assemblages with a Joint
Species Distribution Model. Methods in Ecology and Evolution.

Harris, D. J., K. G. Smith, and P. J. Hanly. 2011. Occupancy is
nine-tenths of the law: Occupancy rates determine the homogenizing and
differentiating effects of exotic species. The American naturalist
177:535.

Harris, T. E. 1974. Contact Interactions on a Lattice. The Annals of
Probability 2:969--988.

Lee, J. D., and T. J. Hastie. 2012, May. Learning Mixed Graphical
Models.

Lewin, R. 1983. Santa Rosalia Was a Goat. Science 221:636--639.

Loh, P.-L., and M. J. Wainwright. 2013. Structure estimation for
discrete graphical models: Generalized covariance matrices and their
inverses. The Annals of Statistics 41:3022--3049.

MacArthur, R. H. 1958. Population ecology of some warblers of
northeastern coniferous forests. Ecology 39:599--619.

Murphy, K. P. 2012. Machine Learning: A Probabilistic Perspective. The
MIT Press.

Pollock, L. J., R. Tingley, W. K. Morris, N. Golding, R. B. O'Hara, K.
M. Parris, P. A. Vesk, and M. A. McCarthy. 2014. Understanding
co-occurrence by modelling species simultaneously with a Joint Species
Distribution Model (JSDM). Methods in Ecology and Evolution:n/a--n/a.

R Core Team. 2015. R: A Language and Environment for Statistical
Computing. R Foundation for Statistical Computing, Vienna, Austria.

Roughgarden, J. 1983. Competition and Theory in Community Ecology. The
American Naturalist 122:583--601.

Salakhutdinov, R. 2008. Learning and evaluating Boltzmann machines.
Technical Report UTML TR 2008-002, Department of Computer Science,
University of Toronto, Dept. of Computer Science, University of Toronto.

Strong, D. R., D. Simberloff, L. G. Abele, and A. B. Thistle. 1984.
Ecological communities: Conceptual issues and the evidence. Princeton
University Press.

Tjelmeland, H., and J. Besag. 1998. Markov Random Fields with
Higher-order Interactions. Scandinavian Journal of Statistics
25:415--433.

Veech, J. A. 2013. A probabilistic model for analysing species
co-occurrence. Global Ecology and Biogeography 22:252--260.

Whittam, T. S., and D. Siegel-Causey. 1981. Species Interactions and
Community Structure in Alaskan Seabird Colonies. Ecology 62:1515--1524.

Xiao, X., D. J. McGlinn, and E. P. White. 2015. A strong test of the
Maximum Entropy Theory of Ecology. The American Naturalist 185:E70--E80.
\end{flushleft}
\end{spacing}

\end{document}
