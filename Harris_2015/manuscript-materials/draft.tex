\documentclass[12pt,]{article}
\usepackage{lmodern}
\usepackage{amssymb,amsmath}
\usepackage{ifxetex,ifluatex}
\usepackage{fixltx2e} % provides \textsubscript
\ifnum 0\ifxetex 1\fi\ifluatex 1\fi=0 % if pdftex
  \usepackage[T1]{fontenc}
  \usepackage[utf8]{inputenc}
\else % if luatex or xelatex
  \ifxetex
    \usepackage{mathspec}
    \usepackage{xltxtra,xunicode}
  \else
    \usepackage{fontspec}
  \fi
  \defaultfontfeatures{Mapping=tex-text,Scale=MatchLowercase}
  \newcommand{\euro}{€}
\fi
% use upquote if available, for straight quotes in verbatim environments
\IfFileExists{upquote.sty}{\usepackage{upquote}}{}
% use microtype if available
\IfFileExists{microtype.sty}{%
\usepackage{microtype}
\UseMicrotypeSet[protrusion]{basicmath} % disable protrusion for tt fonts
}{}
\usepackage{setspace}
\usepackage{lineno}
\linenumbers
\usepackage{setspace}
\usepackage{geometry}
\geometry{verbose,letterpaper,tmargin=2.54cm, bmargin=2.54cm,lmargin=2.54cm,rmargin=2.54cm}
\ifxetex
  \usepackage[setpagesize=false, % page size defined by xetex
              unicode=false, % unicode breaks when used with xetex
              xetex]{hyperref}
\else
  \usepackage[unicode=true]{hyperref}
\fi
\hypersetup{breaklinks=true,
            bookmarks=true,
            pdfauthor={},
            pdftitle={},
            colorlinks=true,
            citecolor=blue,
            urlcolor=blue,
            linkcolor=magenta,
            pdfborder={0 0 0}}
\urlstyle{same}  % don't use monospace font for urls
\setlength{\parindent}{0pt}
\setlength{\parskip}{5pt}
\setlength{\emergencystretch}{3em}  % prevent overfull lines
\setcounter{secnumdepth}{0}

\date{}

\begin{document}


\begin{spacing}{1.9}
\begin{flushleft}
\emph{Running head:} Species interactions in Markov networks

\textbf{Title: Estimating species interactions from observational data
with Markov networks}

\textbf{Author:} David J. Harris: Population Biology; 1 Shields Avenue,
Davis CA, 95616

\textbf{\emph{Abstract}}

Inferring species interactions from observational data is one of the
most controversial tasks in community ecology. One difficulty is that a
single pairwise interaction can ripple through an ecological network and
produce surprising indirect consequences. For example, two competing
species would ordinarily correlate negatively in space, but this effect
can be reversed in the presence of a third species that is capable of
outcompeting both of them when it is present. Here, I apply models from
statistical physics, called Markov networks or Markov random fields,
that can predict the direct and indirect consequences of any possible
species interaction matrix. Interactions in these models can be
estimated from observational data via maximum likelihood. Using
simulated landscapes with known pairwise interaction strengths, I
evaluated Markov networks and several existing approaches. The Markov
networks consistently outperformed other methods, correctly isolating
direct interactions between species pairs even when indirect
interactions or abiotic environmental effects largely overpowered them.
A linear approximation, based on partial covariances, also performed
well as long as the number of sampled locations exceeded the number of
species in the data. Indirect effects reliably caused a common null
modeling approach to produce incorrect inferences, however.

\textbf{Key words:} Ecological interactions; Occurrence data; Species
associations; Markov network; Markov random field; Ising model;
Biogeography; Presence--absence matrix; Null model

\textbf{\emph{Introduction}}

If nontrophic species interactions, such as competition, are important
drivers of community assembly, then ecologists might expect to see their
influence in our data sets (MacArthur 1958, Diamond 1975). Despite
decades of work and several major controversies, however (Lewin 1983,
Strong et al. 1984, Gotelli and Entsminger 2003, Connor et al. 2013),
existing methods for detecting competition's effects on community
structure are unreliable (Gotelli and Ulrich 2009), and thus important
ecological processes remain poorly understood. More generally, it can be
difficult to reason about the complex web of direct and indirect species
interactions in real assemblages, especially when these interactions
occur against a background of other ecological processes such as
dispersal and environmental filtering (Connor et al. 2013). For this
reason, it isn't always clear what kinds of patterns would even
constitute evidence of competition, as opposed to some other biological
process or random sampling error (Lewin 1983, Roughgarden 1983).

Most existing methods in this field compare the frequency with which two
putative competitors are observed to co-occur, versus the frequency that
would be expected if \emph{all} species on the landscape were
independent (Strong et al. 1984, Gotelli and Ulrich 2009). Examining a
species pair against such a ``null'' background, however, rules out the
possibility that the overall association between two species could be
driven by an outside force. For example, even though the two shrub
species in Figure 1 compete with one another for resources at a
mechanistic level, they end up clustering together on the landscape
because they both grow best in areas that are not overshadowed by trees.
If this sort of effect is common, then significant deviations from
independence will not generally provide convincing evidence of species'
direct effects on one another.

While the competition between the two shrubs in the previous example
does not leave the commonly-expected pattern in community structure
(negative association at the landscape level), it nevertheless does
leave a signal in the data (Figure 1C). Specifically, \emph{among shaded
sites}, there will be a deficit of co-occurrences, and \emph{among
unshaded sites}, there will also be such a deficit.

In this paper, I introduce Markov networks (undirected graphical models
also known as Markov random fields; Murphy 2012) as a framework for
understanding the landscape-level consequences of pairwise species
interactions, and for detecting them with observational data. Markov
networks, which generalize partial correlations to non-Gaussian data
(Lee and Hastie 2012, Loh and Wainwright 2013), have been used in many
scientific fields for decades to model associations between various
kinds of ``particles''. For example, a well-studied network named the
Ising model has played an important role in our understanding of physics
(where nearby particles tend to align magnetically with one another;
Cipra 1987). In spatial contexts, these models have been used to
describe interactions between adjacent grid cells (Harris 1974, Gelfand
et al. 2005). In neurobiology, they have helped researchers determine
which neurons are connected to one another by modeling the structure in
their firing patterns (Schneidman et al. 2006). Following recent work by
Azaele et al. (2010) and Fort (2013), I suggest that ecologists could
similarly treat species as the interacting particles in the same
modeling framework. Doing so would allow ecologists to simulate and
study the landscape-level consequences of arbitrary species interaction
matrices, even when our observations are not Gaussian. While ecologists
explored some related approaches in the 1980's (Whittam and
Siegel-Causey 1981), computational limitations had previously imposed
severe approximations that produced unintelligible results (e.g.
``probabilities'' greater than one; Gilpin and Diamond 1982). Now that
it is computationally feasible to fit these models exactly, the approach
has become worth a second look.

The rest of the paper proceeds as follows. First, I discuss how Markov
networks work and how they can be used to simulate landscape-level data
or to predict the direct and indirect consequences of possible
interaction matrices. Then, using simulated data sets where the ``true''
ecological structure is known, I compare this approach with several
existing methods for detecting species interactions. Finally, I discuss
opportunities for extending the approach presented here to larger
problems in community ecology.

\textbf{\emph{Methods}}

\textbf{\emph{Conditional relationships and Markov networks.}}
Ecologists are often interested in inferring direct interactions between
species, controlling for the indirect influence of other species. In
statistical terms, this implies that ecologists want to estimate
\emph{conditional} (``all-else-equal'') relationships, rather than
\emph{marginal} (``overall'') relationships. The most familiar
conditional relationship is the partial correlation, which indicates the
portion of the sample correlation between two species that remains after
controlling for other variables in the data set (Albrecht and Gotelli
2001). The example with the shrubs and trees in Figure 1 shows how the
two correlation measures can have opposite signs, and suggests that the
partial correlation is more relevant for drawing inferences about
species interactions (e.g.~competition). Markov networks extend this
approach to non-Gaussian data, much as generalized linear models do for
linear regression (Lee and Hastie 2012).

Markov networks give a probability value for every possible combination
of presences and absences in communities. For example, given a network
with binary outcomes (i.e.~0 for absence and 1 for presence), the
relative probability of observing a given presence-absence vector,
\(\vec{y}\), is given by

\[p(\vec{y}; \alpha, \beta) \propto exp(\sum_{i}\alpha_i y_i + \sum_{i\neq j}\beta_{ij}y_i y_j).\]

Here, \(\alpha_{i}\) is an intercept term determining the amount that
the presence of species \(i\) contributes to the log-probability of
\(\vec{y}\); it directly controls the prevalence of species \(i\).
Similarly, \(\beta_{ij}\) is the amount that the co-occurrence of
species \(i\) and species \(j\) contributes to the log-probability; it
controls the probability that the two species will be found together
(Figure 2A, Figure 2B). \(\beta\) thus acts as an analog of the partial
covariance, but for non-Gaussian networks. Because the relative
probability of a presence-absence vector increases when
positively-associated species co-occur and decreases when
negatively-associated species co-occur, the model tends to produce
assemblages that have many pairs of positively-associated species and
relatively few pairs of negatively-associated species (exactly as an
ecologist might expect).

A major benefit of Markov networks is the fact that the conditional
relationships between species can be read directly off the matrix of
\(\beta\) coefficients (Murphy 2012). For example, if the coefficient
linking two mutualist species is \(+2\), then---all else equal---the
odds of observing either species increase by a factor of \(e^2\) when
its partner is present (Murphy 2012). Of course, if all else is
\emph{not} equal (e.g.~Figure 1, where the presence of one competitor is
associated with release from another competitor), then species' marginal
association rates can differ from this expectation. For this reason, it
is important to consider how coefficients' effects propagate through the
network, as discussed below.

Estimating the marginal relationships predicted by a Markov network is
more difficult than estimating conditional relationships, because doing
so requires absolute probability estimates. Turning the relative
probability given by Equation 1 into an absolute probability entails
scaling by a \emph{partition function}, \(Z(\alpha, \beta)\), which
ensures that the probabilities of all possible assemblages that could be
produced by the model sum to one (bottom of Figure 2B). Calculating
\(Z(\alpha, \beta)\) exactly, as is done in this paper, quickly becomes
infeasible as the number of species increases: with \(2^N\) possible
assemblages of \(N\) species, the number of bookkeeping operations
required for exact inference spirals exponentially into the billions and
beyond. Numerous techniques are available for working with Markov
networks that keep the computations tractable, e.g.~via analytic
approximations (Lee and Hastie 2012) or Monte Carlo sampling
(Salakhutdinov 2008), but they are beyond the scope of this paper.

\textbf{\emph{Simulations.}} In order to compare different methods for
drawing inferences from observational data, I simulated two sets of
landscapes using known parameters.

The first set of simulated landscapes included the three competing
species shown in Figure 1. For each of 1000 replicates, I generated a
landscape with 100 sites by sampling exactly from a probability
distribution defined by the interaction coefficients in that figure
(Appendix A). Each of the methods described below (a Markov network, two
correlation-based methods and a null model) was then evaluated on its
ability to correctly infer that the two shrub species competed with one
another, despite their frequent co-occurrence.

I also simulated a second set of landscapes with five, ten, or twenty
potentially-interacting species on landscapes composed of 20, 100, 500,
or 2500 observed communities (24 replicate simulations for each
combination; Appendix B). These simulated data sets span the range from
small, single-observer data sets to large collaborative efforts such as
the North American Breeding Bird Survey. As described in Appendix B, I
randomly drew the ``true'' coefficient values for each replicate so that
most species pairs interacted negligibly, a few pairs interacted very
strongly, and competition was three times more common than facilitation.
I then used Gibbs sampling to randomly generate replicate landscapes
with varying numbers of species and sites (Appendix B). For half of the
simulated landscapes, I treated each species' \(\alpha\) coefficient as
a constant, as described above. For the other half, I treated the
\(\alpha\) coefficients as linear functions of two abiotic environmental
factors that varied from location to location across the landscape
(Appendix B). The latter set of simulated landscapes provide an
important test of the methods' ability to distinguish co-occurrence
patterns that were generated from pairwise biotic interactions from
those that were generated by external forces like abiotic environmental
filtering. This task was made especially difficult because---as with
most analyses of presence-absence data for co-occurrence patterns---the
inference procedure did not have access to any information about the
environmental or spatial variables that helped shape the landscape (cf
Connor et al. 2013, Blois et al. 2014).

\textbf{\emph{Inferring \(\alpha\) and \(\beta\) coefficients from
presence-absence data.}} The previous sections involved known values of
\(\alpha\) and \(\beta\). In practice, ecologists will often need to
estimate these parameters from data instead. When the number of species
is reasonably small, one can compute exact maximum likelihood estimates
for all of the \(\alpha\) and \(\beta\) coefficients by optimizing
\(p(\vec{y}; \alpha, \beta)\). Fully-observed Markov networks like the
ones considered here have unimodal likelihood surfaces (Murphy 2012),
ensuring that this procedure will always converge on the global maximum.
This maximum is the unique combination of \(\alpha\) and \(\beta\)
coefficients that would be expected to produce exactly the observed
co-occurrence frequencies. For the analyses in this paper, I used the
\emph{rosalia} package (Harris 2015a) for the R programming language (R
Core Team 2015) to define the objective function and gradient as R code.
The rosalia package then uses the \texttt{BFGS} method in R's
\texttt{optim} function to find the best values for \(\alpha\) and
\(\beta\).

For analyses with 5 or more species, I made a small modification to the
maximum likelihood procedure described above. Given the large number of
parameters associated with some of the networks to be estimated, I
regularized the likelihood using a logistic prior distribution (Gelman
et al. 2008) with a scale of 1 on the \(\alpha\) and \(\beta\) terms.

\textbf{\emph{Other inference techniques for comparison.}} After fitting
Markov networks to the simulated landscapes described above, I used
several other techniques for inferring the sign and strength of marginal
associations between pairs of species (Appendix B).

The first two interaction measures were the sample covariances and the
partial covariances between each pair of species' data vectors on the
landscape (Albrecht and Gotelli 2001). Because partial covariances are
undefined for landscapes with perfectly-correlated species pairs, I used
a regularized estimate based on ridge regression {[}Wieringen and
Peeters (2014); i.e.~linear regression with a Gaussian prior{]}. For
these analyses, I set the ridge parameter to 0.2 divided by the number
of sites on the landscape.

The third method, described in Gotelli and Ulrich (2009), involved
simulating possible landscapes from a null model that retains the row
and column sums of the original matrix (Strong et al. 1984). Using the
default options in the Pairs software described in Gotelli and Ulrich
(2009), I simulated the null distribution of scaled C-scores (a test
statistic describing the number of \emph{non}-co-occurrences between two
species). The software then calculated a \(Z\) statistic for each
species pair using this null distribution. After multiplying this
statistic by \(-1\) so that positive values corresponded to facilitation
and negative values corresponded to competition, I used it as another
estimate of species interactions.

\textbf{\emph{Method evaluation.}} I evaluated each method qualitatively
on the simulated landscapes based on Figure 1: any method that reliably
determined that the two shrub species were negatively associated passed;
other methods failed.

For the larger landscapes, I rescaled the four methods' estimates using
linear regression through the origin so that they all had a consistent
interpretation. In each case, I regressed the ``true'' \(\beta\)
coefficient for each species pair against the model's estimate,
re-weighting the pairs so that each landscape contributed equally to the
rescaled estimate. For each method, I calculated the squared deviations
between each species pair's rescaled estimate and the ``true''
interactions used to generate the assemblages. Finally, I averaged these
squared errors for each combination of species richness, landscape size,
statistical method, and presence/absence of environmental filtering
across all 12 replicates; the mean squared errors associated with these
subsets of the data determined the proportion of variance explained by
each method under different conditions.

\textbf{\emph{Results}}

\textbf{\emph{Three species.}} As shown in Figure 1, the marginal
relationship between the two shrub species was positive---despite their
competition for space at a mechanistic level--- due to indirect effects
of the dominant tree species. As a result, the covariance method falsely
reported positive associations on 94\% of simulated landscapes, and the
randomization-based null model falsely reported such associations 100\%
of the time. The two methods for evaluating conditional relationships
(Markov networks and partial covariances), however, successfully
controlled for the indirect pathway via the tree species and each
correctly identified the direct negative interaction between the shrubs
94\% of the time.

\textbf{\emph{Larger landscapes.}} The accuracy of the four evaluated
methods varied substantially, depending on the parameters that produced
the simulated communities (Figure 3). In general, however, there was a
consistent ordering: overall, the Markov network explained 54\% of the
variance in the ``true'' parameters, followed by partial covariances
(33\%), and sample covariances (22\%).

The null model scores initially explained only 12\% of the variation.
After manually reducing the value of one especially strong outlier
(\(Z=1004\), implying \(p<10^{-1000000}\)), this increased to 17\%
(Appendix B). Figure 3 reflects the adjusted version of the results.

\textbf{\emph{Discussion}}

The results presented above show that Markov networks, unlike a common
null modeling approach, can recover species' pairwise interactions from
observational data. The Markov networks were successful even when direct
interactions are largely overwhelmed by indirect effects (Figure 1) or
environmental effects (lower panels of Figure 3). For cases where
fitting a Markov network is computationally infeasible, these results
also indicate that partial covariances---which can be computed
straightforwardly by linear regression---can often provide a
surprisingly useful approximation. The success of the partial
correlation's success on simulated data may not carry over to real data
sets, however; Loh and Wainwright (2013) suggest that the linear
approximations may be less reliable in cases where the true interaction
matrix contains more structure (e.g.~guilds or trophic levels). On the
other hand, if ecologists are familiar enough with the natural history
of their study systems to describe this kind of structure as a prior
distribution on the parameters or as a penalty on the likelihood, then
real-world results might be even better than those shown in Figure 3.

Ecologists will also need natural history to pin down the exact nature
of the interactions identified by a network model (e.g.~which species in
a positively-associated pair is facilitating the other), particularly
when real pairs of species can reciprocally influence one another in
multiple ways simultaneously (Bruno et al. 2003); the \(\beta\)
coefficients in Markov networks have to reduce this complexity to a
single number. In short, partial correlations and Markov networks both
help prevent us from mistaking marginal associations for conditional
ones, but they can't tell us the underlying biological mechanisms at
work.

Despite these limitations, Markov networks have enormous potential to
improve ecological inferences. For example, Markov networks provide a
simple answer to the question of how competition should affect a
species' overall prevalence, which was a major flash point for the null
model debates in the 1980's (Roughgarden 1983, Strong et al. 1984).
Equation 1 can be used to calculate the expected prevalence of a species
in the absence of biotic influences (\(\frac{1}{1 + e^{-\alpha}}\); Lee
and Hastie 2012). Competition's effect on prevalence in a Markov network
can then be calculated by subtracting this value and the observed
prevalence (cf Figure 2D).

Markov networks---particularly the Ising model for binary
networks---have been studied in statistical physics for nearly a century
(Cipra 1987), and the models' properties, capabilities, and limits are
well-understood in a huge range of applications, from spatial statistics
(Gelfand et al. 2005) to neuroscience (Schneidman et al. 2006) to models
of human behavior (Lee et al. 2013). Modeling species interactions using
the same framework would thus allow ecologists to tap into an enormous
set of existing discoveries and techniques for dealing with indirect
effects, stability, and alternative stable states (i.e.~phase
transitions; Cipra (1987)).

This modeling approach is also highly extensible, even when it is
inconvenient to compute the likelihood exactly. For example, the mistnet
software package for joint species distribution modeling (Harris 2015b)
can fit \emph{approximate} Markov networks to large species assemblages
(\textgreater{}300 species) while simultaneously modeling each species'
nonlinear response to the abiotic environment. Combining multiple
ecological processes into a common model could help ecologists to
disentangle different factors that can confound simpler co-occurrence
analyses (cf Connor et al. 2013). Numerous other extensions are
possible: Markov networks can be fit with a mix of discrete and
continuous variables, for example (Lee and Hastie 2012). There are even
methods (Whittam and Siegel-Causey 1981, Tjelmeland and Besag 1998) that
would allow the coefficient linking two species in an interaction matrix
to vary as a function of the abiotic environment or of third-party
species that tip the balance between facilitation and exploitation
(Bruno et al. 2003).

Finally, the results presented here have important implications for
ecologists' continued use of null models to draw inferences about
species interactions. The small simulated data sets described by Figure
1 show that test statistics based on marginal co-occurrence (such as
C-scores) will not always have a straightforward relationship with the
underlying ecological processes. More generally, deviations from the
null model generally provided less information about direct species
interactions than correlation coefficients did. Scientists currently
using null modeling approaches in their research may be able to capture
twice as much of the variance in species' true interaction strengths
using partial covariances from linear regression instead, or three times
as much using a Markov network.

Null and neutral models can be very useful for clarifying our thinking
about the numerical consequences of species' richness and abundance
patterns (Harris et al. 2011, Xiao et al. 2015), but deviations from a
null model must be interpreted with care (Roughgarden 1983). In complex
networks of ecological interactions---and even in small networks with
three species---it may simply not be possible to implicate individual
species pairs or specific ecological processes like competition by
rejecting a general- purpose null. Direct estimates of species'
conditional associations may be the only way to make these inferences
reliably.

\textbf{\emph{Acknowledgements:}} This research was funded by a Graduate
Research Fellowship from the US National Science Foundation and
benefited greatly from discussions with A. Sih, M. L. Baskett, R.
McElreath, R. J. Hijmans, A. C. Perry, and C. S. Tysor. Additionally, A.
K. Barner, E. Baldridge, E. P. White, D. Li, D. L. Miller, N. Golding,
and N. J. Gotelli provided useful feedback on an earlier draft of this
work.

\textbf{\emph{References:}}


Albrecht, M., and N. J. Gotelli. 2001. Spatial and temporal niche
partitioning in grassland ants. Oecologia 126:134--141.

Azaele, S., R. Muneepeerakul, A. Rinaldo, and I. Rodriguez-Iturbe. 2010.
Inferring plant ecosystem organization from species occurrences. Journal
of theoretical biology 262:323--329.

Blois, J. L., N. J. Gotelli, A. K. Behrensmeyer, J. T. Faith, S. K.
Lyons, J. W. Williams, K. L. Amatangelo, A. Bercovici, A. Du, J. T.
Eronen, and others. 2014. A framework for evaluating the influence of
climate, dispersal limitation, and biotic interactions using fossil
pollen associations across the late Quaternary. Ecography 37:1095--1108.

Bruno, J. F., J. J. Stachowicz, and M. D. Bertness. 2003. Inclusion of
facilitation into ecological theory. Trends in Ecology \& Evolution
18:119--125.

Cipra, B. A. 1987. An introduction to the Ising model. American
Mathematical Monthly 94:937--959.

Connor, E. F., M. D. Collins, and D. Simberloff. 2013. The checkered
history of checkerboard distributions. Ecology 94:2403--2414.

Diamond, J. M. 1975. The island dilemma: Lessons of modern biogeographic
studies for the design of natural reserves. Biological conservation
7:129--146.

Fort, H. 2013. Statistical Mechanics Ideas and Techniques Applied to
Selected Problems in Ecology. Entropy 15:5237--5276.

Gelfand, A. E., A. M. Schmidt, S. Wu, J. A. Silander, A. Latimer, and A.
G. Rebelo. 2005. Modelling species diversity through species level
hierarchical modelling. Journal of the Royal Statistical Society: Series
C (Applied Statistics) 54:1--20.

Gelman, A., A. Jakulin, M. G. Pittau, and Y.-S. Su. 2008. A Weakly
Informative Default Prior Distribution for Logistic and Other Regression
Models. The Annals of Applied Statistics 2:1360--1383.

Gilpin, M. E., and J. M. Diamond. 1982. Factors contributing to
non-randomness in species Co-occurrences on Islands. Oecologia
52:75--84.

Gotelli, N. J., and G. L. Entsminger. 2003. Swap algorithms in null
model analysis. Ecology:532--535.

Gotelli, N. J., and W. Ulrich. 2009. The empirical Bayes approach as a
tool to identify non-random species associations. Oecologia
162:463--477.

Harris, D. J. 2015a. Rosalia: Exact inference for small binary Markov
networks. R package version 0.1.0. Zenodo.
http://dx.doi.org/10.5281/zenodo.17808.

Harris, D. J. 2015b. Generating realistic assemblages with a Joint
Species Distribution Model. Methods in Ecology and Evolution.

Harris, D. J., K. G. Smith, and P. J. Hanly. 2011. Occupancy is
nine-tenths of the law: Occupancy rates determine the homogenizing and
differentiating effects of exotic species. The American naturalist
177:535.

Harris, T. E. 1974. Contact Interactions on a Lattice. The Annals of
Probability 2:969--988.

Lee, E. D., C. P. Broedersz, and W. Bialek. 2013. Statistical mechanics
of the US Supreme Court. arXiv:1306.5004 {[}cond-mat, physics:physics,
q-bio{]}.

Lee, J. D., and T. J. Hastie. 2012, May. Learning Mixed Graphical
Models.

Lewin, R. 1983. Santa Rosalia Was a Goat. Science 221:636--639.

Loh, P.-L., and M. J. Wainwright. 2013. Structure estimation for
discrete graphical models: Generalized covariance matrices and their
inverses. The Annals of Statistics 41:3022--3049.

MacArthur, R. H. 1958. Population ecology of some warblers of
northeastern coniferous forests. Ecology 39:599--619.

Murphy, K. P. 2012. Machine Learning: A Probabilistic Perspective. The
MIT Press.

R Core Team. 2015. R: A Language and Environment for Statistical
Computing. R Foundation for Statistical Computing, Vienna, Austria.

Roughgarden, J. 1983. Competition and Theory in Community Ecology. The
American Naturalist 122:583--601.

Salakhutdinov, R. 2008. Learning and evaluating Boltzmann machines.
Technical Report UTML TR 2008-002, Department of Computer Science,
University of Toronto, Dept. of Computer Science, University of Toronto.

Schneidman, E., M. J. Berry, R. Segev, and W. Bialek. 2006. Weak
pairwise correlations imply strongly correlated network states in a
neural population. Nature 440:1007--1012.

Strong, D. R., D. Simberloff, L. G. Abele, and A. B. Thistle. 1984.
Ecological communities: Conceptual issues and the evidence. Princeton
University Press.

Tjelmeland, H., and J. Besag. 1998. Markov Random Fields with
Higher-order Interactions. Scandinavian Journal of Statistics
25:415--433.

Whittam, T. S., and D. Siegel-Causey. 1981. Species Interactions and
Community Structure in Alaskan Seabird Colonies. Ecology 62:1515--1524.

Wieringen, W. N. van, and C. F. Peeters. 2014. Ridge Estimation of
Inverse Covariance Matrices from High-Dimensional Data. arXiv preprint
arXiv:1403.0904.

Xiao, X., D. J. McGlinn, and E. P. White. 2015. A strong test of the
Maximum Entropy Theory of Ecology. The American Naturalist 185:E70--E80.

\textbf{\emph{Figure captions}}

\textbf{\emph{Figure 1.}} \textbf{A.} A small network of three competing
species. The tree (top) tends not to co-occur with either of the two
shrub species, as indicated by the strongly negative coefficient linking
them. The two shrub species also compete with one another, as indicated
by their negative coefficient (circled), but this effect is
substantially weaker. \textbf{B.} In spite of the competitive
interactions between the two shrub species, their shared tendency to
occur in locations without trees makes their occurrence vectors
positively correlated (circled). \textbf{C.} Controlling for the tree
species' presence with a conditional method such as a partial covariance
or a Markov network allows us to correctly identify the negative
interaction between these two species (circled).

\textbf{\emph{Figure 2.}} \textbf{A.} A small Markov network with two
species. The depicted abiotic environment favors the occurrence of both
species (\(\alpha >0\)), particularly species 2
(\(\alpha_2 > \alpha_1\)). The negative \(\beta\) coefficient linking
these two species implies that they co-occur less than expected under
independence. \textbf{B.} Relative probabilities of all four possible
presence-absence combinations for Species 1 and Species 2. The exponent
includes \(\alpha_1\) whenever Species 1 is present (\(y_1 = 1\)), but
not when it is absent (\(y_1 = 0\)). Similarly, the exponent includes
\(\alpha_2\) only when species \(2\) is present (\(y_2 = 1\)), and
\(\beta\) only when both are present (\(y_1y_2 = 1\)). The normalizing
constant \(Z\), ensures that the four relative probabilities sum to 1.
In this case, \(Z\) is about 18.5. \textbf{C.} Using the probabilities,
we can find the expected frequencies of all possible co-occurrence
patterns between the two species of interest. \textbf{D.} If \(\beta\)
equaled zero (e.g.~if the species no longer competed for the same
resources), then the reduction in competition would allow each species
to increase its occurrence rate and the deficit of co-occurrences would
be eliminated.

\textbf{\emph{Figure 3.}} Proportion of variance in interaction
coefficients explained by each method with 5, 10, or 20 species arrayed
across varying numbers of sampled locations when environmental filtering
was absent (top row) or present (bottom row). A negative \(R^2\) values
implies that the squared error associated with the corresponding subset
of the predictions was larger than the error one would get from assuming
that all coefficients equalled zero.

\end{flushleft}
\end{spacing}

\end{document}
