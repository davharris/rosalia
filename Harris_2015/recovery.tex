\documentclass[11pt,]{article}
\usepackage{lmodern}
\usepackage{amssymb,amsmath}
\usepackage{ifxetex,ifluatex}
\usepackage{fixltx2e} % provides \textsubscript
\ifnum 0\ifxetex 1\fi\ifluatex 1\fi=0 % if pdftex
  \usepackage[T1]{fontenc}
  \usepackage[utf8]{inputenc}
\else % if luatex or xelatex
  \ifxetex
    \usepackage{mathspec}
    \usepackage{xltxtra,xunicode}
  \else
    \usepackage{fontspec}
  \fi
  \defaultfontfeatures{Mapping=tex-text,Scale=MatchLowercase}
  \newcommand{\euro}{€}
\fi
% use upquote if available, for straight quotes in verbatim environments
\IfFileExists{upquote.sty}{\usepackage{upquote}}{}
% use microtype if available
\IfFileExists{microtype.sty}{%
\usepackage{microtype}
\UseMicrotypeSet[protrusion]{basicmath} % disable protrusion for tt fonts
}{}
\usepackage[margin=1in]{geometry}
\ifxetex
  \usepackage[setpagesize=false, % page size defined by xetex
              unicode=false, % unicode breaks when used with xetex
              xetex]{hyperref}
\else
  \usepackage[unicode=true]{hyperref}
\fi
\hypersetup{breaklinks=true,
            bookmarks=true,
            pdfauthor={David J. Harris},
            pdftitle={Appendix 3: Estimating species interactions},
            colorlinks=true,
            citecolor=blue,
            urlcolor=blue,
            linkcolor=magenta,
            pdfborder={0 0 0}}
\urlstyle{same}  % don't use monospace font for urls
\usepackage{color}
\usepackage{fancyvrb}
\newcommand{\VerbBar}{|}
\newcommand{\VERB}{\Verb[commandchars=\\\{\}]}
\DefineVerbatimEnvironment{Highlighting}{Verbatim}{commandchars=\\\{\}}
% Add ',fontsize=\small' for more characters per line
\usepackage{framed}
\definecolor{shadecolor}{RGB}{248,248,248}
\newenvironment{Shaded}{\begin{snugshade}}{\end{snugshade}}
\newcommand{\KeywordTok}[1]{\textcolor[rgb]{0.13,0.29,0.53}{\textbf{{#1}}}}
\newcommand{\DataTypeTok}[1]{\textcolor[rgb]{0.13,0.29,0.53}{{#1}}}
\newcommand{\DecValTok}[1]{\textcolor[rgb]{0.00,0.00,0.81}{{#1}}}
\newcommand{\BaseNTok}[1]{\textcolor[rgb]{0.00,0.00,0.81}{{#1}}}
\newcommand{\FloatTok}[1]{\textcolor[rgb]{0.00,0.00,0.81}{{#1}}}
\newcommand{\ConstantTok}[1]{\textcolor[rgb]{0.00,0.00,0.00}{{#1}}}
\newcommand{\CharTok}[1]{\textcolor[rgb]{0.31,0.60,0.02}{{#1}}}
\newcommand{\SpecialCharTok}[1]{\textcolor[rgb]{0.00,0.00,0.00}{{#1}}}
\newcommand{\StringTok}[1]{\textcolor[rgb]{0.31,0.60,0.02}{{#1}}}
\newcommand{\VerbatimStringTok}[1]{\textcolor[rgb]{0.31,0.60,0.02}{{#1}}}
\newcommand{\SpecialStringTok}[1]{\textcolor[rgb]{0.31,0.60,0.02}{{#1}}}
\newcommand{\ImportTok}[1]{{#1}}
\newcommand{\CommentTok}[1]{\textcolor[rgb]{0.56,0.35,0.01}{\textit{{#1}}}}
\newcommand{\DocumentationTok}[1]{\textcolor[rgb]{0.56,0.35,0.01}{\textbf{\textit{{#1}}}}}
\newcommand{\AnnotationTok}[1]{\textcolor[rgb]{0.56,0.35,0.01}{\textbf{\textit{{#1}}}}}
\newcommand{\CommentVarTok}[1]{\textcolor[rgb]{0.56,0.35,0.01}{\textbf{\textit{{#1}}}}}
\newcommand{\OtherTok}[1]{\textcolor[rgb]{0.56,0.35,0.01}{{#1}}}
\newcommand{\FunctionTok}[1]{\textcolor[rgb]{0.00,0.00,0.00}{{#1}}}
\newcommand{\VariableTok}[1]{\textcolor[rgb]{0.00,0.00,0.00}{{#1}}}
\newcommand{\ControlFlowTok}[1]{\textcolor[rgb]{0.13,0.29,0.53}{\textbf{{#1}}}}
\newcommand{\OperatorTok}[1]{\textcolor[rgb]{0.81,0.36,0.00}{\textbf{{#1}}}}
\newcommand{\BuiltInTok}[1]{{#1}}
\newcommand{\ExtensionTok}[1]{{#1}}
\newcommand{\PreprocessorTok}[1]{\textcolor[rgb]{0.56,0.35,0.01}{\textit{{#1}}}}
\newcommand{\AttributeTok}[1]{\textcolor[rgb]{0.77,0.63,0.00}{{#1}}}
\newcommand{\RegionMarkerTok}[1]{{#1}}
\newcommand{\InformationTok}[1]{\textcolor[rgb]{0.56,0.35,0.01}{\textbf{\textit{{#1}}}}}
\newcommand{\WarningTok}[1]{\textcolor[rgb]{0.56,0.35,0.01}{\textbf{\textit{{#1}}}}}
\newcommand{\AlertTok}[1]{\textcolor[rgb]{0.94,0.16,0.16}{{#1}}}
\newcommand{\ErrorTok}[1]{\textcolor[rgb]{0.64,0.00,0.00}{\textbf{{#1}}}}
\newcommand{\NormalTok}[1]{{#1}}
\setlength{\parindent}{0pt}
\setlength{\parskip}{6pt plus 2pt minus 1pt}
\setlength{\emergencystretch}{3em}  % prevent overfull lines
\providecommand{\tightlist}{%
  \setlength{\itemsep}{0pt}\setlength{\parskip}{0pt}}
\setcounter{secnumdepth}{0}

%%% Use protect on footnotes to avoid problems with footnotes in titles
\let\rmarkdownfootnote\footnote%
\def\footnote{\protect\rmarkdownfootnote}

%%% Change title format to be more compact
\usepackage{titling}

% Create subtitle command for use in maketitle
\newcommand{\subtitle}[1]{
  \posttitle{
    \begin{center}\large#1\end{center}
    }
}

\setlength{\droptitle}{-2em}
  \title{Appendix 3: Estimating species interactions}
  \pretitle{\vspace{\droptitle}\centering\huge}
  \posttitle{\par}
\subtitle{Inferring species interactions from co-occurrence data with Markov
networks}
  \author{David J. Harris}
  \preauthor{\centering\large\emph}
  \postauthor{\par}
  \date{}
  \predate{}\postdate{}


% Redefines (sub)paragraphs to behave more like sections
\ifx\paragraph\undefined\else
\let\oldparagraph\paragraph
\renewcommand{\paragraph}[1]{\oldparagraph{#1}\mbox{}}
\fi
\ifx\subparagraph\undefined\else
\let\oldsubparagraph\subparagraph
\renewcommand{\subparagraph}[1]{\oldsubparagraph{#1}\mbox{}}
\fi

\begin{document}
\maketitle

This document describes how the different models were fit to the
simulated data from Appendix 2 and how each model's performance was
evaluated.\footnote{The PDF version of this document has been manually
  altered to omit 150 lines of output from the \texttt{corpcor} package
  of the form
  ``\texttt{\#\#\ Estimating\ optimal\ shrinkage\ intensity\ lambda\ (correlation\ matrix):\ 0.0326}''}

Note that the \texttt{pairs} program was run separately (outside of R)
with the following options:

\begin{itemize}
\tightlist
\item
  Batch mode
\item
  Sequential swap (``s'')
\item
  Printing all pairs (``y'')
\item
  C-score co-occurrence measure (``c'')
\item
  Default confidence limits (0.05)
\item
  Default iterations (100)
\item
  Maximum of 20 species
\end{itemize}

Initialization:

\begin{Shaded}
\begin{Highlighting}[]
\NormalTok{mc.cores =}\StringTok{ }\DecValTok{8}

\KeywordTok{library}\NormalTok{(dplyr)        }\CommentTok{# For manipulating data structures}
\KeywordTok{library}\NormalTok{(corpcor)      }\CommentTok{# For regularized partial covariances}
\KeywordTok{library}\NormalTok{(rosalia)      }\CommentTok{# For Markov networks}
\KeywordTok{library}\NormalTok{(arm)          }\CommentTok{# For regularized logistic regression}
\KeywordTok{library}\NormalTok{(BayesComm)    }\CommentTok{# For joint species distribution modeling}
\KeywordTok{library}\NormalTok{(parallel)     }\CommentTok{# for mclapply}
\KeywordTok{set.seed}\NormalTok{(}\DecValTok{1}\NormalTok{)}
\end{Highlighting}
\end{Shaded}

A function to import the data file and run each method on it:

\begin{Shaded}
\begin{Highlighting}[]
\NormalTok{fit_all =}\StringTok{ }\NormalTok{function(identifier)\{}
  \NormalTok{######## Import ########}
  
  \NormalTok{data_filename =}\StringTok{ }\KeywordTok{paste0}\NormalTok{(}\StringTok{"fakedata/matrices/"}\NormalTok{, identifier, }\StringTok{".csv"}\NormalTok{)}
  \NormalTok{truth_filename =}\StringTok{ }\KeywordTok{paste0}\NormalTok{(}\StringTok{"fakedata/truths/"}\NormalTok{, identifier, }\StringTok{".txt"}\NormalTok{)}
  
  \CommentTok{# first column is row numbers; drop it}
  \NormalTok{raw_obs =}\StringTok{ }\KeywordTok{as.matrix}\NormalTok{(}\KeywordTok{read.csv}\NormalTok{(data_filename)[ , -}\DecValTok{1}\NormalTok{])}
  
  \CommentTok{# Identify species that are never present (or never absent) so they }
  \CommentTok{# can be dropped}
  \NormalTok{species_is_variable =}\StringTok{ }\KeywordTok{diag}\NormalTok{(}\KeywordTok{var}\NormalTok{(raw_obs)) >}\StringTok{ }\DecValTok{0}
  \NormalTok{pair_is_variable =}\StringTok{ }\KeywordTok{tcrossprod}\NormalTok{(species_is_variable) >}\StringTok{ }\DecValTok{0}
  
  \NormalTok{x =}\StringTok{ }\NormalTok{raw_obs[ , species_is_variable]}
  \NormalTok{truth =}\StringTok{ }\KeywordTok{unlist}\NormalTok{(}\KeywordTok{read.table}\NormalTok{(truth_filename))[pair_is_variable[}\KeywordTok{upper.tri}\NormalTok{(pair_is_variable)]]}
  
  \NormalTok{splitname =}\StringTok{ }\KeywordTok{strsplit}\NormalTok{(identifier, }\StringTok{"/|-|}\CharTok{\textbackslash{}\textbackslash{}}\StringTok{."}\NormalTok{)[[}\DecValTok{1}\NormalTok{]]}
  \NormalTok{n_sites =}\StringTok{ }\KeywordTok{as.integer}\NormalTok{(splitname[[}\DecValTok{1}\NormalTok{]])}
  \NormalTok{rep_name =}\StringTok{ }\NormalTok{splitname[[}\DecValTok{2}\NormalTok{]]}
  
  \CommentTok{# Species IDs}
  \NormalTok{sp1 =}\StringTok{ }\KeywordTok{combn}\NormalTok{(}\KeywordTok{colnames}\NormalTok{(x), }\DecValTok{2}\NormalTok{)[}\DecValTok{1}\NormalTok{, ]}
  \NormalTok{sp2 =}\StringTok{ }\KeywordTok{combn}\NormalTok{(}\KeywordTok{colnames}\NormalTok{(x), }\DecValTok{2}\NormalTok{)[}\DecValTok{2}\NormalTok{, ]}
  
  \NormalTok{######## Partial correlations ########}
  \NormalTok{p_corr =}\StringTok{ }\KeywordTok{pcor.shrink}\NormalTok{(x)}
  
  \NormalTok{######## Correlations ########}
  \NormalTok{corr =}\StringTok{ }\KeywordTok{cor}\NormalTok{(x)}
  \NormalTok{######## GLM ########}
  \NormalTok{coef_matrix =}\StringTok{ }\KeywordTok{matrix}\NormalTok{(}\DecValTok{0}\NormalTok{, }\KeywordTok{ncol}\NormalTok{(x), }\KeywordTok{ncol}\NormalTok{(x))}
  \NormalTok{for(i in }\DecValTok{1}\NormalTok{:}\KeywordTok{ncol}\NormalTok{(x))\{}
    \NormalTok{if(}\KeywordTok{var}\NormalTok{(x[,i]) >}\StringTok{ }\DecValTok{0}\NormalTok{)\{}
      \NormalTok{coefs =}\StringTok{ }\KeywordTok{coef}\NormalTok{(}\KeywordTok{bayesglm}\NormalTok{(x[,i] ~}\StringTok{ }\NormalTok{x[ , -i], }\DataTypeTok{family =} \NormalTok{binomial))[-}\DecValTok{1}\NormalTok{]}
      \NormalTok{coef_matrix[i, -i] =}\StringTok{ }\NormalTok{coefs}
    \NormalTok{\}}
  \NormalTok{\}}
  \NormalTok{coef_matrix =}\StringTok{ }\NormalTok{(coef_matrix +}\StringTok{ }\KeywordTok{t}\NormalTok{(coef_matrix)) /}\StringTok{ }\DecValTok{2}
  
  
  \NormalTok{######## Markov network ########}
  \NormalTok{rosie =}\StringTok{ }\KeywordTok{rosalia}\NormalTok{(x, }\DataTypeTok{maxit =} \DecValTok{200}\NormalTok{, }\DataTypeTok{trace =} \DecValTok{0}\NormalTok{, }\DataTypeTok{prior =} \KeywordTok{make_logistic_prior}\NormalTok{(}\DataTypeTok{scale =} \DecValTok{2}\NormalTok{), }\DataTypeTok{hessian =} \OtherTok{TRUE}\NormalTok{)}
  
  \NormalTok{rosie_point =}\StringTok{ }\NormalTok{rosie$beta[}\KeywordTok{upper.tri}\NormalTok{(rosie$beta)]}
  \NormalTok{rosie_se =}\StringTok{ }\KeywordTok{sqrt}\NormalTok{(}\KeywordTok{diag}\NormalTok{(}\KeywordTok{solve}\NormalTok{(rosie$opt$hessian)))[-(}\DecValTok{1}\NormalTok{:}\KeywordTok{sum}\NormalTok{(species_is_variable))]}

  
  
  \NormalTok{######## BayesComm and partial BayesComm ########}
  \NormalTok{bc =}\StringTok{ }\KeywordTok{BC}\NormalTok{(}\DataTypeTok{Y =} \NormalTok{x, }\DataTypeTok{model =} \StringTok{"community"}\NormalTok{, }\DataTypeTok{its =} \DecValTok{1000}\NormalTok{)}
  
  \NormalTok{bc_pcors =}\StringTok{ }\KeywordTok{sapply}\NormalTok{(}
    \DecValTok{1}\NormalTok{:}\KeywordTok{nrow}\NormalTok{(bc$trace$R), }
    \NormalTok{function(i)\{}
      \NormalTok{Sigma =}\StringTok{ }\KeywordTok{matrix}\NormalTok{(}\DecValTok{0}\NormalTok{, }\DataTypeTok{nrow =} \KeywordTok{ncol}\NormalTok{(x), }\DataTypeTok{ncol =} \KeywordTok{ncol}\NormalTok{(x))}
      \NormalTok{Sigma[}\KeywordTok{upper.tri}\NormalTok{(Sigma)] <-}\StringTok{ }\NormalTok{bc$trace$R[i, ]  }\CommentTok{# Fill in upper triangle}
      \NormalTok{Sigma <-}\StringTok{ }\NormalTok{Sigma +}\StringTok{ }\KeywordTok{t}\NormalTok{(Sigma)                   }\CommentTok{# Fill in lower triangle}
      \KeywordTok{diag}\NormalTok{(Sigma) <-}\StringTok{ }\DecValTok{1}  \CommentTok{# Diagonal equals 1 in multivariate probit model}
      \NormalTok{pcor =}\StringTok{ }\KeywordTok{cor2pcor}\NormalTok{(Sigma)}
      \NormalTok{pcor[}\KeywordTok{upper.tri}\NormalTok{(pcor)]}
    \NormalTok{\}}
  \NormalTok{)}
  
  
  
  
  
  \KeywordTok{bind_rows}\NormalTok{(}
    \KeywordTok{data_frame}\NormalTok{(}
      \DataTypeTok{rep_name =} \NormalTok{rep_name,}
      \DataTypeTok{n_sites =} \NormalTok{n_sites,}
      \DataTypeTok{sp1 =} \NormalTok{sp1,}
      \DataTypeTok{sp2 =} \NormalTok{sp2,}
      \DataTypeTok{method =} \StringTok{"Markov network"}\NormalTok{,}
      \DataTypeTok{truth =} \NormalTok{truth,}
      \DataTypeTok{estimate =} \NormalTok{rosie_point, }
      \DataTypeTok{lower =} \KeywordTok{qnorm}\NormalTok{(.}\DecValTok{025}\NormalTok{, rosie_point, rosie_se), }
      \DataTypeTok{upper =} \KeywordTok{qnorm}\NormalTok{(.}\DecValTok{975}\NormalTok{, rosie_point, rosie_se)}
    \NormalTok{),}
    \KeywordTok{data_frame}\NormalTok{(}
      \DataTypeTok{rep_name =} \NormalTok{rep_name,}
      \DataTypeTok{n_sites =} \NormalTok{n_sites,}
      \DataTypeTok{sp1 =} \NormalTok{sp1,}
      \DataTypeTok{sp2 =} \NormalTok{sp2,}
      \DataTypeTok{method =} \StringTok{"GLM"}\NormalTok{,}
      \DataTypeTok{truth =} \NormalTok{truth,}
      \DataTypeTok{estimate =} \NormalTok{coef_matrix[}\KeywordTok{upper.tri}\NormalTok{(coef_matrix)], }
      \DataTypeTok{lower =} \OtherTok{NA}\NormalTok{, }
      \DataTypeTok{upper =} \OtherTok{NA}
    \NormalTok{),}
    \KeywordTok{data_frame}\NormalTok{(}
      \DataTypeTok{rep_name =} \NormalTok{rep_name,}
      \DataTypeTok{n_sites =} \NormalTok{n_sites,}
      \DataTypeTok{sp1 =} \NormalTok{sp1,}
      \DataTypeTok{sp2 =} \NormalTok{sp2,}
      \DataTypeTok{method =} \StringTok{"correlation"}\NormalTok{,}
      \DataTypeTok{truth =} \NormalTok{truth,}
      \DataTypeTok{estimate =} \NormalTok{corr[}\KeywordTok{upper.tri}\NormalTok{(corr)], }
      \DataTypeTok{lower =} \OtherTok{NA}\NormalTok{, }
      \DataTypeTok{upper =} \OtherTok{NA}
    \NormalTok{),}
    \KeywordTok{data_frame}\NormalTok{(}
      \DataTypeTok{rep_name =} \NormalTok{rep_name,}
      \DataTypeTok{n_sites =} \NormalTok{n_sites,}
      \DataTypeTok{sp1 =} \NormalTok{sp1,}
      \DataTypeTok{sp2 =} \NormalTok{sp2,}
      \DataTypeTok{method =} \StringTok{"partial correlation"}\NormalTok{,}
      \DataTypeTok{truth =} \NormalTok{truth,}
      \DataTypeTok{estimate =} \NormalTok{p_corr[}\KeywordTok{upper.tri}\NormalTok{(p_corr)], }
      \DataTypeTok{lower =} \OtherTok{NA}\NormalTok{, }
      \DataTypeTok{upper =} \OtherTok{NA}
    \NormalTok{),}
    \KeywordTok{data_frame}\NormalTok{(}
      \DataTypeTok{rep_name =} \NormalTok{rep_name,}
      \DataTypeTok{n_sites =} \NormalTok{n_sites,}
      \DataTypeTok{sp1 =} \NormalTok{sp1,}
      \DataTypeTok{sp2 =} \NormalTok{sp2,}
      \DataTypeTok{method =} \StringTok{"partial BayesComm"}\NormalTok{,}
      \DataTypeTok{truth =} \NormalTok{truth,}
      \DataTypeTok{estimate =} \KeywordTok{rowMeans}\NormalTok{(bc_pcors),}
      \DataTypeTok{lower =} \KeywordTok{apply}\NormalTok{(bc_pcors, }\DecValTok{1}\NormalTok{, quantile, .}\DecValTok{025}\NormalTok{),}
      \DataTypeTok{upper =} \KeywordTok{apply}\NormalTok{(bc_pcors, }\DecValTok{1}\NormalTok{, quantile, .}\DecValTok{975}\NormalTok{)}
    \NormalTok{),}
    \KeywordTok{data_frame}\NormalTok{(}
      \DataTypeTok{rep_name =} \NormalTok{rep_name,}
      \DataTypeTok{n_sites =} \NormalTok{n_sites,}
      \DataTypeTok{sp1 =} \NormalTok{sp1,}
      \DataTypeTok{sp2 =} \NormalTok{sp2,}
      \DataTypeTok{method =} \StringTok{"BayesComm"}\NormalTok{,}
      \DataTypeTok{truth =} \NormalTok{truth,}
      \DataTypeTok{estimate =} \KeywordTok{colMeans}\NormalTok{(bc$trace$R),}
      \DataTypeTok{lower =} \KeywordTok{apply}\NormalTok{(bc$trace$R, }\DecValTok{2}\NormalTok{, quantile, .}\DecValTok{025}\NormalTok{),}
      \DataTypeTok{upper =} \KeywordTok{apply}\NormalTok{(bc$trace$R, }\DecValTok{2}\NormalTok{, quantile, .}\DecValTok{975}\NormalTok{)}
    \NormalTok{)}
  \NormalTok{)}
\NormalTok{\}}
\end{Highlighting}
\end{Shaded}

Run the above function on all the files:

\begin{Shaded}
\begin{Highlighting}[]
\CommentTok{# Find all the csv files in the fakedata/matrices folder,}
\CommentTok{# then drop .csv}
\NormalTok{identifiers =}\StringTok{ }\KeywordTok{dir}\NormalTok{(}\StringTok{"fakedata/matrices"}\NormalTok{, }\DataTypeTok{pattern =} \StringTok{"}\CharTok{\textbackslash{}\textbackslash{}}\StringTok{.csv$"}\NormalTok{) %>%}
\StringTok{  }\KeywordTok{gsub}\NormalTok{(}\StringTok{"}\CharTok{\textbackslash{}\textbackslash{}}\StringTok{.csv$"}\NormalTok{, }\StringTok{""}\NormalTok{, .)}

\KeywordTok{mclapply}\NormalTok{(identifiers, fit_all, }\DataTypeTok{mc.cores =} \NormalTok{mc.cores, }\DataTypeTok{mc.preschedule =} \OtherTok{FALSE}\NormalTok{, }\DataTypeTok{mc.silent =} \OtherTok{TRUE}\NormalTok{) %>%}\StringTok{ }
\StringTok{  }\KeywordTok{bind_rows}\NormalTok{() %>%}\StringTok{ }
\StringTok{  }\KeywordTok{write.csv}\NormalTok{(}\DataTypeTok{file =} \StringTok{"estimates.csv"}\NormalTok{)}
\end{Highlighting}
\end{Shaded}

\end{document}
