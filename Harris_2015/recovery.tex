\documentclass[11pt,]{article}
\usepackage{lmodern}
\usepackage{amssymb,amsmath}
\usepackage{ifxetex,ifluatex}
\usepackage{fixltx2e} % provides \textsubscript
\ifnum 0\ifxetex 1\fi\ifluatex 1\fi=0 % if pdftex
  \usepackage[T1]{fontenc}
  \usepackage[utf8]{inputenc}
\else % if luatex or xelatex
  \ifxetex
    \usepackage{mathspec}
    \usepackage{xltxtra,xunicode}
  \else
    \usepackage{fontspec}
  \fi
  \defaultfontfeatures{Mapping=tex-text,Scale=MatchLowercase}
  \newcommand{\euro}{€}
\fi
% use upquote if available, for straight quotes in verbatim environments
\IfFileExists{upquote.sty}{\usepackage{upquote}}{}
% use microtype if available
\IfFileExists{microtype.sty}{%
\usepackage{microtype}
\UseMicrotypeSet[protrusion]{basicmath} % disable protrusion for tt fonts
}{}
\usepackage[margin=1in]{geometry}
\ifxetex
  \usepackage[setpagesize=false, % page size defined by xetex
              unicode=false, % unicode breaks when used with xetex
              xetex]{hyperref}
\else
  \usepackage[unicode=true]{hyperref}
\fi
\hypersetup{breaklinks=true,
            bookmarks=true,
            pdfauthor={David J. Harris},
            pdftitle={Appendix 3: Estimating species interactions},
            colorlinks=true,
            citecolor=blue,
            urlcolor=blue,
            linkcolor=magenta,
            pdfborder={0 0 0}}
\urlstyle{same}  % don't use monospace font for urls
\usepackage{color}
\usepackage{fancyvrb}
\newcommand{\VerbBar}{|}
\newcommand{\VERB}{\Verb[commandchars=\\\{\}]}
\DefineVerbatimEnvironment{Highlighting}{Verbatim}{commandchars=\\\{\}}
% Add ',fontsize=\small' for more characters per line
\usepackage{framed}
\definecolor{shadecolor}{RGB}{248,248,248}
\newenvironment{Shaded}{\begin{snugshade}}{\end{snugshade}}
\newcommand{\KeywordTok}[1]{\textcolor[rgb]{0.13,0.29,0.53}{\textbf{{#1}}}}
\newcommand{\DataTypeTok}[1]{\textcolor[rgb]{0.13,0.29,0.53}{{#1}}}
\newcommand{\DecValTok}[1]{\textcolor[rgb]{0.00,0.00,0.81}{{#1}}}
\newcommand{\BaseNTok}[1]{\textcolor[rgb]{0.00,0.00,0.81}{{#1}}}
\newcommand{\FloatTok}[1]{\textcolor[rgb]{0.00,0.00,0.81}{{#1}}}
\newcommand{\ConstantTok}[1]{\textcolor[rgb]{0.00,0.00,0.00}{{#1}}}
\newcommand{\CharTok}[1]{\textcolor[rgb]{0.31,0.60,0.02}{{#1}}}
\newcommand{\SpecialCharTok}[1]{\textcolor[rgb]{0.00,0.00,0.00}{{#1}}}
\newcommand{\StringTok}[1]{\textcolor[rgb]{0.31,0.60,0.02}{{#1}}}
\newcommand{\VerbatimStringTok}[1]{\textcolor[rgb]{0.31,0.60,0.02}{{#1}}}
\newcommand{\SpecialStringTok}[1]{\textcolor[rgb]{0.31,0.60,0.02}{{#1}}}
\newcommand{\ImportTok}[1]{{#1}}
\newcommand{\CommentTok}[1]{\textcolor[rgb]{0.56,0.35,0.01}{\textit{{#1}}}}
\newcommand{\DocumentationTok}[1]{\textcolor[rgb]{0.56,0.35,0.01}{\textbf{\textit{{#1}}}}}
\newcommand{\AnnotationTok}[1]{\textcolor[rgb]{0.56,0.35,0.01}{\textbf{\textit{{#1}}}}}
\newcommand{\CommentVarTok}[1]{\textcolor[rgb]{0.56,0.35,0.01}{\textbf{\textit{{#1}}}}}
\newcommand{\OtherTok}[1]{\textcolor[rgb]{0.56,0.35,0.01}{{#1}}}
\newcommand{\FunctionTok}[1]{\textcolor[rgb]{0.00,0.00,0.00}{{#1}}}
\newcommand{\VariableTok}[1]{\textcolor[rgb]{0.00,0.00,0.00}{{#1}}}
\newcommand{\ControlFlowTok}[1]{\textcolor[rgb]{0.13,0.29,0.53}{\textbf{{#1}}}}
\newcommand{\OperatorTok}[1]{\textcolor[rgb]{0.81,0.36,0.00}{\textbf{{#1}}}}
\newcommand{\BuiltInTok}[1]{{#1}}
\newcommand{\ExtensionTok}[1]{{#1}}
\newcommand{\PreprocessorTok}[1]{\textcolor[rgb]{0.56,0.35,0.01}{\textit{{#1}}}}
\newcommand{\AttributeTok}[1]{\textcolor[rgb]{0.77,0.63,0.00}{{#1}}}
\newcommand{\RegionMarkerTok}[1]{{#1}}
\newcommand{\InformationTok}[1]{\textcolor[rgb]{0.56,0.35,0.01}{\textbf{\textit{{#1}}}}}
\newcommand{\WarningTok}[1]{\textcolor[rgb]{0.56,0.35,0.01}{\textbf{\textit{{#1}}}}}
\newcommand{\AlertTok}[1]{\textcolor[rgb]{0.94,0.16,0.16}{{#1}}}
\newcommand{\ErrorTok}[1]{\textcolor[rgb]{0.64,0.00,0.00}{\textbf{{#1}}}}
\newcommand{\NormalTok}[1]{{#1}}
\usepackage{graphicx,grffile}
\makeatletter
\def\maxwidth{\ifdim\Gin@nat@width>\linewidth\linewidth\else\Gin@nat@width\fi}
\def\maxheight{\ifdim\Gin@nat@height>\textheight\textheight\else\Gin@nat@height\fi}
\makeatother
% Scale images if necessary, so that they will not overflow the page
% margins by default, and it is still possible to overwrite the defaults
% using explicit options in \includegraphics[width, height, ...]{}
\setkeys{Gin}{width=\maxwidth,height=\maxheight,keepaspectratio}
\setlength{\parindent}{0pt}
\setlength{\parskip}{6pt plus 2pt minus 1pt}
\setlength{\emergencystretch}{3em}  % prevent overfull lines
\providecommand{\tightlist}{%
  \setlength{\itemsep}{0pt}\setlength{\parskip}{0pt}}
\setcounter{secnumdepth}{0}

%%% Use protect on footnotes to avoid problems with footnotes in titles
\let\rmarkdownfootnote\footnote%
\def\footnote{\protect\rmarkdownfootnote}

%%% Change title format to be more compact
\usepackage{titling}

% Create subtitle command for use in maketitle
\newcommand{\subtitle}[1]{
  \posttitle{
    \begin{center}\large#1\end{center}
    }
}

\setlength{\droptitle}{-2em}
  \title{Appendix 3: Estimating species interactions}
  \pretitle{\vspace{\droptitle}\centering\huge}
  \posttitle{\par}
\subtitle{Inferring species interactions from co-occurrence data with Markov
networks}
  \author{David J. Harris}
  \preauthor{\centering\large\emph}
  \postauthor{\par}
  \date{}
  \predate{}\postdate{}


% Redefines (sub)paragraphs to behave more like sections
\ifx\paragraph\undefined\else
\let\oldparagraph\paragraph
\renewcommand{\paragraph}[1]{\oldparagraph{#1}\mbox{}}
\fi
\ifx\subparagraph\undefined\else
\let\oldsubparagraph\subparagraph
\renewcommand{\subparagraph}[1]{\oldsubparagraph{#1}\mbox{}}
\fi

\begin{document}
\maketitle

This document describes how the different models were fit to the
simulated data from Appendix 2 and how each model's performance was
evaluated.{[}\^{}1{]}

Note that the \texttt{pairs} program was run separately (outside of R)
with the following options:

\begin{itemize}
\tightlist
\item
  Batch mode
\item
  Sequential swap (``s'')
\item
  Printing all pairs (``y'')
\item
  C-score co-occurrence measure (``c'')
\item
  Default confidence limits (0.05)
\item
  Default iterations (100)
\item
  Maximum of 20 species
\end{itemize}

Initialization:

\begin{Shaded}
\begin{Highlighting}[]
\NormalTok{mc.cores =}\StringTok{ }\DecValTok{8}

\KeywordTok{library}\NormalTok{(dplyr)        }\CommentTok{# For manipulating data structures}
\KeywordTok{library}\NormalTok{(corpcor)      }\CommentTok{# For regularized partial covariances}
\KeywordTok{library}\NormalTok{(rosalia)      }\CommentTok{# For Markov networks}
\KeywordTok{library}\NormalTok{(arm)          }\CommentTok{# For regularized logistic regression}
\KeywordTok{library}\NormalTok{(BayesComm)    }\CommentTok{# For joint species distribution modeling}
\KeywordTok{library}\NormalTok{(parallel)     }\CommentTok{# for mclapply}
\KeywordTok{set.seed}\NormalTok{(}\DecValTok{1}\NormalTok{)}
\end{Highlighting}
\end{Shaded}

A function to import the data file and run each method on it:

\begin{Shaded}
\begin{Highlighting}[]
\NormalTok{fit_all =}\StringTok{ }\NormalTok{function(identifier)\{}
  \NormalTok{######## Import ########}
  
  \NormalTok{data_filename =}\StringTok{ }\KeywordTok{paste0}\NormalTok{(}\StringTok{"fakedata/matrices/"}\NormalTok{, identifier, }\StringTok{".csv"}\NormalTok{)}
  \NormalTok{truth_filename =}\StringTok{ }\KeywordTok{paste0}\NormalTok{(}\StringTok{"fakedata/truths/"}\NormalTok{, identifier, }\StringTok{".txt"}\NormalTok{)}
  
  \CommentTok{# first column is row numbers; drop it}
  \NormalTok{raw_obs =}\StringTok{ }\KeywordTok{as.matrix}\NormalTok{(}\KeywordTok{read.csv}\NormalTok{(data_filename)[ , -}\DecValTok{1}\NormalTok{])}
  
  \CommentTok{# Identify species that are never present (or never absent) so they }
  \CommentTok{# can be dropped}
  \NormalTok{species_is_variable =}\StringTok{ }\KeywordTok{diag}\NormalTok{(}\KeywordTok{var}\NormalTok{(raw_obs)) >}\StringTok{ }\DecValTok{0}
  \NormalTok{pair_is_variable =}\StringTok{ }\KeywordTok{tcrossprod}\NormalTok{(species_is_variable) >}\StringTok{ }\DecValTok{0}
  
  \NormalTok{x =}\StringTok{ }\NormalTok{raw_obs[ , species_is_variable]}
  \NormalTok{truth =}\StringTok{ }\KeywordTok{unlist}\NormalTok{(}\KeywordTok{read.table}\NormalTok{(truth_filename))[pair_is_variable[}\KeywordTok{upper.tri}\NormalTok{(pair_is_variable)]]}
  
  \NormalTok{splitname =}\StringTok{ }\KeywordTok{strsplit}\NormalTok{(identifier, }\StringTok{"/|-|}\CharTok{\textbackslash{}\textbackslash{}}\StringTok{."}\NormalTok{)[[}\DecValTok{1}\NormalTok{]]}
  \NormalTok{n_sites =}\StringTok{ }\KeywordTok{as.integer}\NormalTok{(splitname[[}\DecValTok{1}\NormalTok{]])}
  \NormalTok{rep_name =}\StringTok{ }\NormalTok{splitname[[}\DecValTok{2}\NormalTok{]]}
  
  \CommentTok{# Species IDs}
  \NormalTok{sp1 =}\StringTok{ }\KeywordTok{combn}\NormalTok{(}\KeywordTok{colnames}\NormalTok{(x), }\DecValTok{2}\NormalTok{)[}\DecValTok{1}\NormalTok{, ]}
  \NormalTok{sp2 =}\StringTok{ }\KeywordTok{combn}\NormalTok{(}\KeywordTok{colnames}\NormalTok{(x), }\DecValTok{2}\NormalTok{)[}\DecValTok{2}\NormalTok{, ]}
  
  \NormalTok{######## Partial correlations ########}
  \NormalTok{p_corr =}\StringTok{ }\KeywordTok{pcor.shrink}\NormalTok{(x)}
  
  \NormalTok{######## Correlations ########}
  \NormalTok{corr =}\StringTok{ }\KeywordTok{cor}\NormalTok{(x)}
  \NormalTok{######## GLM ########}
  \NormalTok{coef_matrix =}\StringTok{ }\KeywordTok{matrix}\NormalTok{(}\DecValTok{0}\NormalTok{, }\KeywordTok{ncol}\NormalTok{(x), }\KeywordTok{ncol}\NormalTok{(x))}
  \NormalTok{for(i in }\DecValTok{1}\NormalTok{:}\KeywordTok{ncol}\NormalTok{(x))\{}
    \NormalTok{if(}\KeywordTok{var}\NormalTok{(x[,i]) >}\StringTok{ }\DecValTok{0}\NormalTok{)\{}
      \NormalTok{coefs =}\StringTok{ }\KeywordTok{coef}\NormalTok{(}\KeywordTok{bayesglm}\NormalTok{(x[,i] ~}\StringTok{ }\NormalTok{x[ , -i], }\DataTypeTok{family =} \NormalTok{binomial))[-}\DecValTok{1}\NormalTok{]}
      \NormalTok{coef_matrix[i, -i] =}\StringTok{ }\NormalTok{coefs}
    \NormalTok{\}}
  \NormalTok{\}}
  \NormalTok{coef_matrix =}\StringTok{ }\NormalTok{(coef_matrix +}\StringTok{ }\KeywordTok{t}\NormalTok{(coef_matrix)) /}\StringTok{ }\DecValTok{2}
  
  
  \NormalTok{######## Markov network ########}
  \NormalTok{rosie =}\StringTok{ }\KeywordTok{rosalia}\NormalTok{(x, }\DataTypeTok{maxit =} \DecValTok{200}\NormalTok{, }\DataTypeTok{trace =} \DecValTok{0}\NormalTok{, }
                  \DataTypeTok{prior =} \KeywordTok{make_logistic_prior}\NormalTok{(}\DataTypeTok{scale =} \DecValTok{2}\NormalTok{), }\DataTypeTok{hessian =} \OtherTok{TRUE}\NormalTok{)}
  
  \NormalTok{rosie_point =}\StringTok{ }\NormalTok{rosie$beta[}\KeywordTok{upper.tri}\NormalTok{(rosie$beta)]}
  \NormalTok{rosie_se =}\StringTok{ }\KeywordTok{sqrt}\NormalTok{(}\KeywordTok{diag}\NormalTok{(}\KeywordTok{solve}\NormalTok{(rosie$opt$hessian)))[-(}\DecValTok{1}\NormalTok{:}\KeywordTok{sum}\NormalTok{(species_is_variable))]}

  
  
  \NormalTok{######## BayesComm and partial BayesComm ########}
  \NormalTok{bc =}\StringTok{ }\KeywordTok{BC}\NormalTok{(}\DataTypeTok{Y =} \NormalTok{x, }\DataTypeTok{model =} \StringTok{"community"}\NormalTok{, }\DataTypeTok{its =} \DecValTok{1000}\NormalTok{)}
  
  \NormalTok{bc_pcors =}\StringTok{ }\KeywordTok{sapply}\NormalTok{(}
    \DecValTok{1}\NormalTok{:}\KeywordTok{nrow}\NormalTok{(bc$trace$R), }
    \NormalTok{function(i)\{}
      \NormalTok{Sigma =}\StringTok{ }\KeywordTok{matrix}\NormalTok{(}\DecValTok{0}\NormalTok{, }\DataTypeTok{nrow =} \KeywordTok{ncol}\NormalTok{(x), }\DataTypeTok{ncol =} \KeywordTok{ncol}\NormalTok{(x))}
      \NormalTok{Sigma[}\KeywordTok{upper.tri}\NormalTok{(Sigma)] <-}\StringTok{ }\NormalTok{bc$trace$R[i, ]  }\CommentTok{# Fill in upper triangle}
      \NormalTok{Sigma <-}\StringTok{ }\NormalTok{Sigma +}\StringTok{ }\KeywordTok{t}\NormalTok{(Sigma)                   }\CommentTok{# Fill in lower triangle}
      \KeywordTok{diag}\NormalTok{(Sigma) <-}\StringTok{ }\DecValTok{1}  \CommentTok{# Diagonal equals 1 in multivariate probit model}
      \NormalTok{pcor =}\StringTok{ }\KeywordTok{cor2pcor}\NormalTok{(Sigma)}
      \NormalTok{pcor[}\KeywordTok{upper.tri}\NormalTok{(pcor)]}
    \NormalTok{\}}
  \NormalTok{)}
  
  
  
  
  
  \KeywordTok{bind_rows}\NormalTok{(}
    \KeywordTok{data_frame}\NormalTok{(}
      \DataTypeTok{rep_name =} \NormalTok{rep_name,}
      \DataTypeTok{n_sites =} \NormalTok{n_sites,}
      \DataTypeTok{sp1 =} \NormalTok{sp1,}
      \DataTypeTok{sp2 =} \NormalTok{sp2,}
      \DataTypeTok{method =} \StringTok{"Markov network"}\NormalTok{,}
      \DataTypeTok{truth =} \NormalTok{truth,}
      \DataTypeTok{estimate =} \NormalTok{rosie_point, }
      \DataTypeTok{lower =} \KeywordTok{qnorm}\NormalTok{(.}\DecValTok{025}\NormalTok{, rosie_point, rosie_se), }
      \DataTypeTok{upper =} \KeywordTok{qnorm}\NormalTok{(.}\DecValTok{975}\NormalTok{, rosie_point, rosie_se)}
    \NormalTok{),}
    \KeywordTok{data_frame}\NormalTok{(}
      \DataTypeTok{rep_name =} \NormalTok{rep_name,}
      \DataTypeTok{n_sites =} \NormalTok{n_sites,}
      \DataTypeTok{sp1 =} \NormalTok{sp1,}
      \DataTypeTok{sp2 =} \NormalTok{sp2,}
      \DataTypeTok{method =} \StringTok{"GLM"}\NormalTok{,}
      \DataTypeTok{truth =} \NormalTok{truth,}
      \DataTypeTok{estimate =} \NormalTok{coef_matrix[}\KeywordTok{upper.tri}\NormalTok{(coef_matrix)], }
      \DataTypeTok{lower =} \OtherTok{NA}\NormalTok{, }
      \DataTypeTok{upper =} \OtherTok{NA}
    \NormalTok{),}
    \KeywordTok{data_frame}\NormalTok{(}
      \DataTypeTok{rep_name =} \NormalTok{rep_name,}
      \DataTypeTok{n_sites =} \NormalTok{n_sites,}
      \DataTypeTok{sp1 =} \NormalTok{sp1,}
      \DataTypeTok{sp2 =} \NormalTok{sp2,}
      \DataTypeTok{method =} \StringTok{"correlation"}\NormalTok{,}
      \DataTypeTok{truth =} \NormalTok{truth,}
      \DataTypeTok{estimate =} \NormalTok{corr[}\KeywordTok{upper.tri}\NormalTok{(corr)], }
      \DataTypeTok{lower =} \OtherTok{NA}\NormalTok{, }
      \DataTypeTok{upper =} \OtherTok{NA}
    \NormalTok{),}
    \KeywordTok{data_frame}\NormalTok{(}
      \DataTypeTok{rep_name =} \NormalTok{rep_name,}
      \DataTypeTok{n_sites =} \NormalTok{n_sites,}
      \DataTypeTok{sp1 =} \NormalTok{sp1,}
      \DataTypeTok{sp2 =} \NormalTok{sp2,}
      \DataTypeTok{method =} \StringTok{"partial correlation"}\NormalTok{,}
      \DataTypeTok{truth =} \NormalTok{truth,}
      \DataTypeTok{estimate =} \NormalTok{p_corr[}\KeywordTok{upper.tri}\NormalTok{(p_corr)], }
      \DataTypeTok{lower =} \OtherTok{NA}\NormalTok{, }
      \DataTypeTok{upper =} \OtherTok{NA}
    \NormalTok{),}
    \KeywordTok{data_frame}\NormalTok{(}
      \DataTypeTok{rep_name =} \NormalTok{rep_name,}
      \DataTypeTok{n_sites =} \NormalTok{n_sites,}
      \DataTypeTok{sp1 =} \NormalTok{sp1,}
      \DataTypeTok{sp2 =} \NormalTok{sp2,}
      \DataTypeTok{method =} \StringTok{"partial BayesComm"}\NormalTok{,}
      \DataTypeTok{truth =} \NormalTok{truth,}
      \DataTypeTok{estimate =} \KeywordTok{rowMeans}\NormalTok{(bc_pcors),}
      \DataTypeTok{lower =} \KeywordTok{apply}\NormalTok{(bc_pcors, }\DecValTok{1}\NormalTok{, quantile, .}\DecValTok{025}\NormalTok{),}
      \DataTypeTok{upper =} \KeywordTok{apply}\NormalTok{(bc_pcors, }\DecValTok{1}\NormalTok{, quantile, .}\DecValTok{975}\NormalTok{)}
    \NormalTok{),}
    \KeywordTok{data_frame}\NormalTok{(}
      \DataTypeTok{rep_name =} \NormalTok{rep_name,}
      \DataTypeTok{n_sites =} \NormalTok{n_sites,}
      \DataTypeTok{sp1 =} \NormalTok{sp1,}
      \DataTypeTok{sp2 =} \NormalTok{sp2,}
      \DataTypeTok{method =} \StringTok{"BayesComm"}\NormalTok{,}
      \DataTypeTok{truth =} \NormalTok{truth,}
      \DataTypeTok{estimate =} \KeywordTok{colMeans}\NormalTok{(bc$trace$R),}
      \DataTypeTok{lower =} \KeywordTok{apply}\NormalTok{(bc$trace$R, }\DecValTok{2}\NormalTok{, quantile, .}\DecValTok{025}\NormalTok{),}
      \DataTypeTok{upper =} \KeywordTok{apply}\NormalTok{(bc$trace$R, }\DecValTok{2}\NormalTok{, quantile, .}\DecValTok{975}\NormalTok{)}
    \NormalTok{)}
  \NormalTok{)}
\NormalTok{\}}
\end{Highlighting}
\end{Shaded}

Run the above function on all the files:

\begin{Shaded}
\begin{Highlighting}[]
\CommentTok{# Find all the csv files in the fakedata/matrices folder,}
\CommentTok{# then drop .csv}
\NormalTok{identifiers =}\StringTok{ }\KeywordTok{dir}\NormalTok{(}\StringTok{"fakedata/matrices"}\NormalTok{, }\DataTypeTok{pattern =} \StringTok{"}\CharTok{\textbackslash{}\textbackslash{}}\StringTok{.csv$"}\NormalTok{) %>%}
\StringTok{  }\KeywordTok{gsub}\NormalTok{(}\StringTok{"}\CharTok{\textbackslash{}\textbackslash{}}\StringTok{.csv$"}\NormalTok{, }\StringTok{""}\NormalTok{, .)}
\end{Highlighting}
\end{Shaded}

\begin{Shaded}
\begin{Highlighting}[]
\KeywordTok{mclapply}\NormalTok{(identifiers, fit_all, }\DataTypeTok{mc.cores =} \NormalTok{mc.cores, }
         \DataTypeTok{mc.preschedule =} \OtherTok{FALSE}\NormalTok{, }\DataTypeTok{mc.silent =} \OtherTok{TRUE}\NormalTok{) %>%}\StringTok{ }
\StringTok{  }\KeywordTok{bind_rows}\NormalTok{() %>%}\StringTok{ }
\StringTok{  }\KeywordTok{write.csv}\NormalTok{(}\DataTypeTok{file =} \StringTok{"estimates.csv"}\NormalTok{)}
\end{Highlighting}
\end{Shaded}

\subsection{Logistic prior used for the Markov
network}\label{logistic-prior-used-for-the-markov-network}

From the R help file for the logistic distribution:

\begin{quote}
The Logistic distribution with location = m and scale = s has\ldots{}
density
\end{quote}

\begin{quote}
\begin{quote}
f(x) = 1/s exp((x-m)/s) (1 + exp((x-m)/s))\^{}-2.
\end{quote}
\end{quote}

\begin{quote}
It is a long-tailed distribution with mean m and variance \(\pi\)\^{}2
/3 s\^{}2.
\end{quote}

As noted in the main text, I used location zero and scale 2 (plotted in
red below). The ``true'' parameter distribution for \(\beta\) is drawn
in black, for reference. These distributions have different means and
the prior is substantially wider than the distribution of true
parameters (mean absolute deviation from zero about 2.8 times larger).

\begin{Shaded}
\begin{Highlighting}[]
\KeywordTok{curve}\NormalTok{(}\KeywordTok{ifelse}\NormalTok{(x <}\StringTok{ }\DecValTok{0}\NormalTok{, .}\DecValTok{75} \NormalTok{*}\StringTok{ }\KeywordTok{dexp}\NormalTok{(}\KeywordTok{abs}\NormalTok{(x)), .}\DecValTok{25} \NormalTok{*}\StringTok{ }\KeywordTok{dexp}\NormalTok{(x)), }\DataTypeTok{from =} \NormalTok{-}\DecValTok{15}\NormalTok{, }\DataTypeTok{to =} \DecValTok{15}\NormalTok{,}
      \DataTypeTok{n =} \DecValTok{1000}\NormalTok{, }\DataTypeTok{ylab =} \StringTok{"Density"}\NormalTok{)}
\KeywordTok{curve}\NormalTok{(}\KeywordTok{dlogis}\NormalTok{(x, }\DataTypeTok{location =} \DecValTok{0}\NormalTok{, }\DataTypeTok{scale =} \DecValTok{2}\NormalTok{), }\DataTypeTok{add =} \OtherTok{TRUE}\NormalTok{, }\DataTypeTok{col =} \StringTok{"red"}\NormalTok{)}
\end{Highlighting}
\end{Shaded}

\includegraphics{recovery_files/figure-latex/unnamed-chunk-5-1.pdf}

The log of this density was added to the log-likelihood to calculate an
un-normalized log-posterior, which is optimized by the \texttt{optim}
function in the \texttt{stats} package.

This prior distribution is ``weakly informative'' in the sense of Gelman
et al. (2008. Annals of Applied Statistics, ``A weakly informative
default prior distribution for logistic and other regression models''),
as shown below: the logistic regularizer pulls the estimated values
toward zero less strongly than Gelman et al.'s weakly-informative Cauchy
prior does.

\begin{Shaded}
\begin{Highlighting}[]
\NormalTok{x =}\StringTok{ }\KeywordTok{as.matrix}\NormalTok{(}
  \KeywordTok{read.csv}\NormalTok{(}\StringTok{"fakedata/matrices/25-no_env1.csv"}\NormalTok{)[,-}\DecValTok{1}\NormalTok{]}
\NormalTok{)}

\CommentTok{# Don't estimate parameters for species that were never observed}
\NormalTok{x =}\StringTok{ }\NormalTok{x[ , }\KeywordTok{colMeans}\NormalTok{(x) !=}\StringTok{ }\DecValTok{0}\NormalTok{]}


\CommentTok{# Estimate parameters using a weakly informative logistic prior}
\NormalTok{logistic_fit =}\StringTok{ }\KeywordTok{rosalia}\NormalTok{(x, }\DataTypeTok{prior =} \KeywordTok{make_logistic_prior}\NormalTok{(}\DataTypeTok{scale =} \DecValTok{2}\NormalTok{),}
                             \DataTypeTok{trace =} \OtherTok{FALSE}\NormalTok{)}

\CommentTok{# Estimate parameters using Gelman et al.'s weakly informative Cauchy prior}
\NormalTok{cauchy_fit =}\StringTok{ }\KeywordTok{rosalia}\NormalTok{(x, }\DataTypeTok{prior =} \KeywordTok{make_cauchy_prior}\NormalTok{(}\DataTypeTok{scale =} \FloatTok{2.5}\NormalTok{),}
                           \DataTypeTok{trace =} \OtherTok{FALSE}\NormalTok{)}


\KeywordTok{plot}\NormalTok{(logistic_fit$beta, cauchy_fit$beta)}
\KeywordTok{abline}\NormalTok{(}\DecValTok{0}\NormalTok{,}\DecValTok{1}\NormalTok{, }\DataTypeTok{h =} \DecValTok{0}\NormalTok{, }\DataTypeTok{v =} \DecValTok{0}\NormalTok{, }\DataTypeTok{col =} \StringTok{"darkgray"}\NormalTok{)}
\end{Highlighting}
\end{Shaded}

\includegraphics{recovery_files/figure-latex/unnamed-chunk-6-1.pdf}

\subsection{GLM coefficient
correlations}\label{glm-coefficient-correlations}

The GLM method produces two estimates for each species pair. These
estimates tend to be very similar, so most reasonable methods for
generating a consensus estimate should produce very similar results.

\begin{Shaded}
\begin{Highlighting}[]
\NormalTok{get_glm_cor =}\StringTok{ }\NormalTok{function(identifier)\{}
  \NormalTok{data_filename =}\StringTok{ }\KeywordTok{paste0}\NormalTok{(}\StringTok{"fakedata/matrices/"}\NormalTok{, identifier, }\StringTok{".csv"}\NormalTok{)}
  
  \CommentTok{# first column is row numbers; drop it}
  \NormalTok{raw_obs =}\StringTok{ }\KeywordTok{as.matrix}\NormalTok{(}\KeywordTok{read.csv}\NormalTok{(data_filename)[ , -}\DecValTok{1}\NormalTok{])}
  
  \CommentTok{# Identify species that are never present (or never absent) so they }
  \CommentTok{# can be dropped}
  \NormalTok{species_is_variable =}\StringTok{ }\KeywordTok{diag}\NormalTok{(}\KeywordTok{var}\NormalTok{(raw_obs)) >}\StringTok{ }\DecValTok{0}
  \NormalTok{pair_is_variable =}\StringTok{ }\KeywordTok{tcrossprod}\NormalTok{(species_is_variable) >}\StringTok{ }\DecValTok{0}
  
  
  \NormalTok{x =}\StringTok{ }\NormalTok{raw_obs[ , species_is_variable]}
  
  \NormalTok{coef_matrix =}\StringTok{ }\KeywordTok{matrix}\NormalTok{(}\DecValTok{0}\NormalTok{, }\KeywordTok{ncol}\NormalTok{(x), }\KeywordTok{ncol}\NormalTok{(x))}
  
  \CommentTok{# Fill in the coefficient matrix, column-by-column}
  \NormalTok{for(i in }\DecValTok{1}\NormalTok{:}\KeywordTok{ncol}\NormalTok{(x))\{}
    \NormalTok{if(}\KeywordTok{var}\NormalTok{(x[,i]) >}\StringTok{ }\DecValTok{0}\NormalTok{)\{}
      \NormalTok{coefs =}\StringTok{ }\KeywordTok{coef}\NormalTok{(}\KeywordTok{bayesglm}\NormalTok{(x[,i] ~}\StringTok{ }\NormalTok{x[ , -i], }\DataTypeTok{family =} \NormalTok{binomial))[-}\DecValTok{1}\NormalTok{]}
      \NormalTok{coef_matrix[i, -i] =}\StringTok{ }\NormalTok{coefs}
    \NormalTok{\}}
  \NormalTok{\}}
  
  \CommentTok{# Compare the estimates from the upper triangle with the estimates from }
  \CommentTok{# the estimates in the lower triangle (which becomes the upper triangle after }
  \CommentTok{# transposing)}
  \KeywordTok{cor}\NormalTok{(coef_matrix[}\KeywordTok{upper.tri}\NormalTok{(coef_matrix)], }\KeywordTok{t}\NormalTok{(coef_matrix)[}\KeywordTok{upper.tri}\NormalTok{(coef_matrix)])}
\NormalTok{\}}

\NormalTok{glm_correlations =}\StringTok{ }\KeywordTok{unlist}\NormalTok{(}\KeywordTok{mclapply}\NormalTok{(identifiers, get_glm_cor))}

\KeywordTok{mean}\NormalTok{(glm_correlations)}
\end{Highlighting}
\end{Shaded}

\begin{verbatim}
## [1] 0.9546387
\end{verbatim}

\begin{Shaded}
\begin{Highlighting}[]
\KeywordTok{hist}\NormalTok{(glm_correlations)}
\end{Highlighting}
\end{Shaded}

\includegraphics{recovery_files/figure-latex/unnamed-chunk-7-1.pdf}

\end{document}
