\documentclass[11pt,]{article}
\usepackage{lmodern}
\usepackage{amssymb,amsmath}
\usepackage{ifxetex,ifluatex}
\usepackage{fixltx2e} % provides \textsubscript
\ifnum 0\ifxetex 1\fi\ifluatex 1\fi=0 % if pdftex
  \usepackage[T1]{fontenc}
  \usepackage[utf8]{inputenc}
\else % if luatex or xelatex
  \ifxetex
    \usepackage{mathspec}
    \usepackage{xltxtra,xunicode}
  \else
    \usepackage{fontspec}
  \fi
  \defaultfontfeatures{Mapping=tex-text,Scale=MatchLowercase}
  \newcommand{\euro}{€}
\fi
% use upquote if available, for straight quotes in verbatim environments
\IfFileExists{upquote.sty}{\usepackage{upquote}}{}
% use microtype if available
\IfFileExists{microtype.sty}{%
\usepackage{microtype}
\UseMicrotypeSet[protrusion]{basicmath} % disable protrusion for tt fonts
}{}
\usepackage[margin=1in]{geometry}
\ifxetex
  \usepackage[setpagesize=false, % page size defined by xetex
              unicode=false, % unicode breaks when used with xetex
              xetex]{hyperref}
\else
  \usepackage[unicode=true]{hyperref}
\fi
\hypersetup{breaklinks=true,
            bookmarks=true,
            pdfauthor={David J. Harris},
            pdftitle={Appendix 4: Results},
            colorlinks=true,
            citecolor=blue,
            urlcolor=blue,
            linkcolor=magenta,
            pdfborder={0 0 0}}
\urlstyle{same}  % don't use monospace font for urls
\usepackage{color}
\usepackage{fancyvrb}
\newcommand{\VerbBar}{|}
\newcommand{\VERB}{\Verb[commandchars=\\\{\}]}
\DefineVerbatimEnvironment{Highlighting}{Verbatim}{commandchars=\\\{\}}
% Add ',fontsize=\small' for more characters per line
\usepackage{framed}
\definecolor{shadecolor}{RGB}{248,248,248}
\newenvironment{Shaded}{\begin{snugshade}}{\end{snugshade}}
\newcommand{\KeywordTok}[1]{\textcolor[rgb]{0.13,0.29,0.53}{\textbf{{#1}}}}
\newcommand{\DataTypeTok}[1]{\textcolor[rgb]{0.13,0.29,0.53}{{#1}}}
\newcommand{\DecValTok}[1]{\textcolor[rgb]{0.00,0.00,0.81}{{#1}}}
\newcommand{\BaseNTok}[1]{\textcolor[rgb]{0.00,0.00,0.81}{{#1}}}
\newcommand{\FloatTok}[1]{\textcolor[rgb]{0.00,0.00,0.81}{{#1}}}
\newcommand{\ConstantTok}[1]{\textcolor[rgb]{0.00,0.00,0.00}{{#1}}}
\newcommand{\CharTok}[1]{\textcolor[rgb]{0.31,0.60,0.02}{{#1}}}
\newcommand{\SpecialCharTok}[1]{\textcolor[rgb]{0.00,0.00,0.00}{{#1}}}
\newcommand{\StringTok}[1]{\textcolor[rgb]{0.31,0.60,0.02}{{#1}}}
\newcommand{\VerbatimStringTok}[1]{\textcolor[rgb]{0.31,0.60,0.02}{{#1}}}
\newcommand{\SpecialStringTok}[1]{\textcolor[rgb]{0.31,0.60,0.02}{{#1}}}
\newcommand{\ImportTok}[1]{{#1}}
\newcommand{\CommentTok}[1]{\textcolor[rgb]{0.56,0.35,0.01}{\textit{{#1}}}}
\newcommand{\DocumentationTok}[1]{\textcolor[rgb]{0.56,0.35,0.01}{\textbf{\textit{{#1}}}}}
\newcommand{\AnnotationTok}[1]{\textcolor[rgb]{0.56,0.35,0.01}{\textbf{\textit{{#1}}}}}
\newcommand{\CommentVarTok}[1]{\textcolor[rgb]{0.56,0.35,0.01}{\textbf{\textit{{#1}}}}}
\newcommand{\OtherTok}[1]{\textcolor[rgb]{0.56,0.35,0.01}{{#1}}}
\newcommand{\FunctionTok}[1]{\textcolor[rgb]{0.00,0.00,0.00}{{#1}}}
\newcommand{\VariableTok}[1]{\textcolor[rgb]{0.00,0.00,0.00}{{#1}}}
\newcommand{\ControlFlowTok}[1]{\textcolor[rgb]{0.13,0.29,0.53}{\textbf{{#1}}}}
\newcommand{\OperatorTok}[1]{\textcolor[rgb]{0.81,0.36,0.00}{\textbf{{#1}}}}
\newcommand{\BuiltInTok}[1]{{#1}}
\newcommand{\ExtensionTok}[1]{{#1}}
\newcommand{\PreprocessorTok}[1]{\textcolor[rgb]{0.56,0.35,0.01}{\textit{{#1}}}}
\newcommand{\AttributeTok}[1]{\textcolor[rgb]{0.77,0.63,0.00}{{#1}}}
\newcommand{\RegionMarkerTok}[1]{{#1}}
\newcommand{\InformationTok}[1]{\textcolor[rgb]{0.56,0.35,0.01}{\textbf{\textit{{#1}}}}}
\newcommand{\WarningTok}[1]{\textcolor[rgb]{0.56,0.35,0.01}{\textbf{\textit{{#1}}}}}
\newcommand{\AlertTok}[1]{\textcolor[rgb]{0.94,0.16,0.16}{{#1}}}
\newcommand{\ErrorTok}[1]{\textcolor[rgb]{0.64,0.00,0.00}{\textbf{{#1}}}}
\newcommand{\NormalTok}[1]{{#1}}
\usepackage{longtable,booktabs}
\usepackage{graphicx,grffile}
\makeatletter
\def\maxwidth{\ifdim\Gin@nat@width>\linewidth\linewidth\else\Gin@nat@width\fi}
\def\maxheight{\ifdim\Gin@nat@height>\textheight\textheight\else\Gin@nat@height\fi}
\makeatother
% Scale images if necessary, so that they will not overflow the page
% margins by default, and it is still possible to overwrite the defaults
% using explicit options in \includegraphics[width, height, ...]{}
\setkeys{Gin}{width=\maxwidth,height=\maxheight,keepaspectratio}
\setlength{\parindent}{0pt}
\setlength{\parskip}{6pt plus 2pt minus 1pt}
\setlength{\emergencystretch}{3em}  % prevent overfull lines
\providecommand{\tightlist}{%
  \setlength{\itemsep}{0pt}\setlength{\parskip}{0pt}}
\setcounter{secnumdepth}{0}

%%% Use protect on footnotes to avoid problems with footnotes in titles
\let\rmarkdownfootnote\footnote%
\def\footnote{\protect\rmarkdownfootnote}

%%% Change title format to be more compact
\usepackage{titling}

% Create subtitle command for use in maketitle
\newcommand{\subtitle}[1]{
  \posttitle{
    \begin{center}\large#1\end{center}
    }
}

\setlength{\droptitle}{-2em}
  \title{Appendix 4: Results}
  \pretitle{\vspace{\droptitle}\centering\huge}
  \posttitle{\par}
\subtitle{Inferring species interactions from co-occurrence data with Markov
networks}
  \author{David J. Harris}
  \preauthor{\centering\large\emph}
  \postauthor{\par}
  \date{}
  \predate{}\postdate{}


% Redefines (sub)paragraphs to behave more like sections
\ifx\paragraph\undefined\else
\let\oldparagraph\paragraph
\renewcommand{\paragraph}[1]{\oldparagraph{#1}\mbox{}}
\fi
\ifx\subparagraph\undefined\else
\let\oldsubparagraph\subparagraph
\renewcommand{\subparagraph}[1]{\oldsubparagraph{#1}\mbox{}}
\fi

\begin{document}
\maketitle

{
\hypersetup{linkcolor=black}
\setcounter{tocdepth}{2}
\tableofcontents
}
\section{A: Import packages and data}\label{a-import-packages-and-data}

\begin{Shaded}
\begin{Highlighting}[]
\KeywordTok{library}\NormalTok{(dplyr)}
\KeywordTok{library}\NormalTok{(magrittr)}
\KeywordTok{library}\NormalTok{(mgcv)}
\KeywordTok{library}\NormalTok{(ggplot2)}
\KeywordTok{library}\NormalTok{(tidyr)}
\KeywordTok{library}\NormalTok{(knitr)}
\KeywordTok{library}\NormalTok{(lme4)}
\end{Highlighting}
\end{Shaded}

\subsection{Import the results from Appendix
3:}\label{import-the-results-from-appendix-3}

\begin{Shaded}
\begin{Highlighting}[]
\NormalTok{x =}\StringTok{ }\KeywordTok{read.csv}\NormalTok{(}\StringTok{"estimates.csv"}\NormalTok{, }\DataTypeTok{stringsAsFactors =} \OtherTok{FALSE}\NormalTok{)}
\NormalTok{x$simulation_type =}\StringTok{ }\KeywordTok{gsub}\NormalTok{(}\StringTok{"[0-9]"}\NormalTok{, }\StringTok{""}\NormalTok{, x$rep_name)}
\end{Highlighting}
\end{Shaded}

\subsection{\texorpdfstring{Import the results from the \emph{Pairs}
software:}{Import the results from the Pairs software:}}\label{import-the-results-from-the-pairs-software}

\begin{Shaded}
\begin{Highlighting}[]
\NormalTok{pairs_txt =}\StringTok{ }\KeywordTok{readLines}\NormalTok{(}\StringTok{"fakedata/matrices/Pairs.txt"}\NormalTok{)}
\KeywordTok{library}\NormalTok{(stringr)}

\CommentTok{# Find areas of the data file that correspond}
\CommentTok{# to species pairs' results}
\NormalTok{beginnings =}\StringTok{ }\KeywordTok{grep}\NormalTok{(}\StringTok{"Sp1"}\NormalTok{, pairs_txt) +}\StringTok{ }\DecValTok{1}
\NormalTok{ends =}\StringTok{ }\KeywordTok{c}\NormalTok{(}
  \KeywordTok{grep}\NormalTok{(}\StringTok{"^[^ ]"}\NormalTok{, pairs_txt)[-}\DecValTok{1}\NormalTok{],}
  \KeywordTok{length}\NormalTok{(pairs_txt) +}\StringTok{ }\DecValTok{1}
\NormalTok{) -}\StringTok{ }\DecValTok{1}

\NormalTok{partial_names =}\StringTok{ }\KeywordTok{sapply}\NormalTok{(}
  \KeywordTok{strsplit}\NormalTok{(}\KeywordTok{grep}\NormalTok{(}\StringTok{"^>"}\NormalTok{, pairs_txt, }\DataTypeTok{value =} \OtherTok{TRUE}\NormalTok{), }\StringTok{" +"}\NormalTok{), }
  \NormalTok{function(x) x[[}\DecValTok{3}\NormalTok{]]}
\NormalTok{)}
\NormalTok{filename_lines =}\StringTok{ }\KeywordTok{grep}\NormalTok{(}\StringTok{"^>"}\NormalTok{, pairs_txt)}

\CommentTok{# Sort a vector of alphanumeric strings by the numeric component}
\CommentTok{# as if they were integers.  For example, V20 should be larger }
\CommentTok{# than V12, even though V12 comes first alphabetically}
\NormalTok{alnum_sort =}\StringTok{ }\NormalTok{function(x)\{}
  \NormalTok{raw =}\StringTok{ }\KeywordTok{as.integer}\NormalTok{(}\KeywordTok{gsub}\NormalTok{(}\StringTok{"[[:alpha:]]"}\NormalTok{, }\StringTok{""}\NormalTok{, x))}
  \NormalTok{x[}\KeywordTok{order}\NormalTok{(raw)]}
\NormalTok{\}}

\NormalTok{pairs_results =}\StringTok{ }\KeywordTok{lapply}\NormalTok{(}
  \DecValTok{1}\NormalTok{:}\KeywordTok{length}\NormalTok{(filename_lines),}
  \NormalTok{function(i)\{}
    
    \NormalTok{n_sites =}\StringTok{ }\KeywordTok{as.integer}\NormalTok{(}\KeywordTok{strsplit}\NormalTok{(partial_names[[i]], }\StringTok{"-"}\NormalTok{)[[}\DecValTok{1}\NormalTok{]][[}\DecValTok{1}\NormalTok{]])}
    \NormalTok{rep_name =}\StringTok{ }\KeywordTok{strsplit}\NormalTok{(partial_names[[i]], }\StringTok{"-"}\NormalTok{)[[}\DecValTok{1}\NormalTok{]][[}\DecValTok{2}\NormalTok{]]}
    
    
    \CommentTok{# Find the line where the current data set is mentioned in}
    \CommentTok{# pairs.txt}
    \NormalTok{filename_line =}\StringTok{ }\NormalTok{filename_lines[i]}
    
    \CommentTok{# Which chunk of the data file corresponds to this file?}
    \NormalTok{chunk =}\StringTok{ }\KeywordTok{min}\NormalTok{(}\KeywordTok{which}\NormalTok{(beginnings >}\StringTok{ }\NormalTok{filename_line))}
    
    \CommentTok{# Split the chunk on whitespace.}
    \NormalTok{splitted =}\StringTok{ }\KeywordTok{strsplit}\NormalTok{(pairs_txt[beginnings[chunk]:ends[chunk]], }\StringTok{" +"}\NormalTok{)}
    
    \CommentTok{# Pull out the corresponding chunk of the "x" data frame, based on n_sites }
    \CommentTok{# and rep_name}
    \NormalTok{is_correct_sim =}\StringTok{ }\NormalTok{x$n_sites ==}\StringTok{ }\NormalTok{n_sites &}\StringTok{ }\NormalTok{x$rep_name ==}\StringTok{ }\NormalTok{rep_name}
    \NormalTok{x_subset =}\StringTok{ }\NormalTok{x[is_correct_sim &}\StringTok{ }\NormalTok{x$method ==}\StringTok{ "correlation"}\NormalTok{, ]}
    
    \CommentTok{# Pull out the species numbers and their Z-scores, then join to x_subset}
    \NormalTok{pairs_results =}\StringTok{ }\KeywordTok{lapply}\NormalTok{(}
      \NormalTok{splitted,}
      \NormalTok{function(x)\{}
        \CommentTok{# in the x data frame, species 1 is always a lower number than species 2}
        \NormalTok{spp =}\StringTok{ }\KeywordTok{alnum_sort}\NormalTok{(x[}\DecValTok{3}\NormalTok{:}\DecValTok{4}\NormalTok{])}
        \KeywordTok{data.frame}\NormalTok{(}
          \DataTypeTok{sp1 =} \NormalTok{spp[}\DecValTok{1}\NormalTok{], }
          \DataTypeTok{sp2 =} \NormalTok{spp[}\DecValTok{2}\NormalTok{], }
          \DataTypeTok{z =} \NormalTok{x[}\DecValTok{14}\NormalTok{],}
          \DataTypeTok{stringsAsFactors =} \OtherTok{FALSE}
        \NormalTok{)}
      \NormalTok{\}}
    \NormalTok{) %>%}\StringTok{ }
\StringTok{      }\NormalTok{bind_rows %>%}
\StringTok{      }\KeywordTok{mutate}\NormalTok{(}\DataTypeTok{spp =} \KeywordTok{paste}\NormalTok{(sp1, sp2, }\DataTypeTok{sep =} \StringTok{"-"}\NormalTok{))}
    
    \NormalTok{n_spp =}\StringTok{ }\DecValTok{20}
    
    \NormalTok{pairs_results$z =}\StringTok{ }\KeywordTok{as.numeric}\NormalTok{(pairs_results$z)}
    
    \CommentTok{# Re-order the pairs_results to match the other methods}
    \NormalTok{m =}\StringTok{ }\KeywordTok{matrix}\NormalTok{(}\OtherTok{NA}\NormalTok{, n_spp, n_spp)}
    \NormalTok{new_order =}\StringTok{ }\KeywordTok{match}\NormalTok{(}
      \KeywordTok{paste0}\NormalTok{(}\StringTok{"V"}\NormalTok{, }\KeywordTok{row}\NormalTok{(m)[}\KeywordTok{upper.tri}\NormalTok{(m)], }\StringTok{"-V"}\NormalTok{, }\KeywordTok{col}\NormalTok{(m)[}\KeywordTok{upper.tri}\NormalTok{(m)]),}
      \NormalTok{pairs_results$spp}
    \NormalTok{)}
    \NormalTok{ordered_pairs_results =}\StringTok{ }\NormalTok{pairs_results[}\KeywordTok{na.omit}\NormalTok{(new_order), ]}
    
    \NormalTok{ordered_pairs_results =}\StringTok{ }\NormalTok{ordered_pairs_results %>%}
\StringTok{      }\KeywordTok{filter}\NormalTok{(sp1 %in%}\StringTok{ }\KeywordTok{c}\NormalTok{(x_subset$sp1) &}\StringTok{ }\NormalTok{sp2 %in%}\StringTok{ }\NormalTok{x_subset$sp2)}
    
    \NormalTok{x_subset$estimate =}\StringTok{ }\NormalTok{ordered_pairs_results$z}
    \NormalTok{x_subset$method =}\StringTok{ "null"}
    
    \NormalTok{x_subset}
  \NormalTok{\}}
\NormalTok{) %>%}\StringTok{ }\KeywordTok{bind_rows}\NormalTok{()}

\CommentTok{# Manually adjust the Z values less than -1000 so that these outliers}
\CommentTok{# won't completely dominate the analyses below}
\NormalTok{pairs_results$estimate[pairs_results$estimate <}\StringTok{ }\NormalTok{-}\DecValTok{1000}\NormalTok{] =}\StringTok{ }\NormalTok{-}\DecValTok{50}

\NormalTok{x =}\StringTok{ }\KeywordTok{rbind}\NormalTok{(x, pairs_results)}
\end{Highlighting}
\end{Shaded}

\section{B: Compare model estimates}\label{b-compare-model-estimates}

\subsection{Calculate model performance
(R-squared):}\label{calculate-model-performance-r-squared}

For each combination of \texttt{method} and \texttt{simulation\_type},
fit a linear model. Then, for each combination of \texttt{method},
\texttt{simulation\_type}, and \texttt{n\_sites}, report the proportion
of variance explained by the corresponding linear model.

\begin{Shaded}
\begin{Highlighting}[]
\NormalTok{resids =}\StringTok{ }\NormalTok{function(data)\{}
  \KeywordTok{resid}\NormalTok{(}\KeywordTok{lm}\NormalTok{(truth ~}\StringTok{ }\NormalTok{estimate +}\StringTok{ }\DecValTok{0}\NormalTok{, }\DataTypeTok{data =} \NormalTok{data))}
\NormalTok{\}}

\NormalTok{result_summary =}\StringTok{ }\NormalTok{x %>%}\StringTok{ }
\StringTok{  }\KeywordTok{group_by}\NormalTok{(method, simulation_type) %>%}
\StringTok{  }\KeywordTok{do}\NormalTok{(}\KeywordTok{data.frame}\NormalTok{(., }\DataTypeTok{resids =} \KeywordTok{resids}\NormalTok{(.))) %>%}
\StringTok{  }\NormalTok{ungroup %>%}
\StringTok{  }\KeywordTok{group_by}\NormalTok{(method, simulation_type, n_sites) %>%}
\StringTok{  }\KeywordTok{summarise}\NormalTok{(}\DataTypeTok{r2 =} \DecValTok{1} \NormalTok{-}\StringTok{ }\KeywordTok{sum}\NormalTok{(resids^}\DecValTok{2}\NormalTok{) /}\StringTok{ }\KeywordTok{sum}\NormalTok{(truth^}\DecValTok{2}\NormalTok{))}

\CommentTok{# Set the ordering of the data frame based on R-squared}
\NormalTok{result_summary$method =}\StringTok{ }\KeywordTok{reorder}\NormalTok{(result_summary$method, -result_summary$r2)}
\NormalTok{result_summary$simulation_type =}\StringTok{ }\KeywordTok{reorder}\NormalTok{(}
  \NormalTok{result_summary$simulation_type, }
  \NormalTok{-result_summary$r2}
\NormalTok{)}

\CommentTok{# Add a column that includes method name and its mean R-squared,}
\CommentTok{# for plotting purposes below}
\NormalTok{result_summary =}\StringTok{ }\NormalTok{result_summary %>%}\StringTok{ }
\StringTok{  }\KeywordTok{group_by}\NormalTok{(method) %>%}\StringTok{ }
\StringTok{  }\KeywordTok{summarise}\NormalTok{(}\DataTypeTok{mean_r2 =} \KeywordTok{round}\NormalTok{(}\DecValTok{100} \NormalTok{*}\StringTok{ }\KeywordTok{mean}\NormalTok{(r2))) %>%}
\StringTok{  }\KeywordTok{mutate}\NormalTok{(}\DataTypeTok{method_r2 =} \KeywordTok{paste0}\NormalTok{(method, }\StringTok{" (0."}\NormalTok{, mean_r2, }\StringTok{")"}\NormalTok{)) %>%}
\StringTok{  }\KeywordTok{select}\NormalTok{(method, method_r2) %>%}
\StringTok{  }\KeywordTok{inner_join}\NormalTok{(result_summary, }\StringTok{"method"}\NormalTok{)}

\CommentTok{# Rename the simulation types for clearer graph labels}
\NormalTok{result_summary$simulation_type_long =}\StringTok{ }\NormalTok{plyr::}\KeywordTok{revalue}\NormalTok{(}
  \NormalTok{result_summary$simulation_type,}
  \KeywordTok{c}\NormalTok{(}\DataTypeTok{no_env =} \StringTok{"constant environment"}\NormalTok{,}
    \DataTypeTok{env =} \StringTok{"heterogeneous environment"}\NormalTok{,}
    \DataTypeTok{abund =} \StringTok{"abundance"}\NormalTok{)}
\NormalTok{)}

\NormalTok{result_summary$method_r2 =}\StringTok{ }\KeywordTok{reorder}\NormalTok{(result_summary$method_r2, -result_summary$r2)}
\end{Highlighting}
\end{Shaded}

\paragraph{Plot the regressions used above to calculate
R\^{}2}\label{plot-the-regressions-used-above-to-calculate-r2}

The plots below each correspond to one of the regressions performed
above to calculate R\^{}2 for a combination of simulation method and
estimation method. The null model (``Pairs'') was not included in these
plots because its Z-scores were frequently too large to fit on the same
scale as the other methods' estimates. Below, the thinner dashed lines
represent the 1:1 line, where the expected estimate matches the ``true''
value, and the thicker solid lines represent a linear regression through
the origin, as calculated above. In general, the four correlation- or
partial correlation-based methods had slopes that were steeper than the
1:1 line, because their estimates were constrained between -1 and 1. The
GLM and Markov network usually had slopes that were closer to 1.

(Warnings of the form ``Removed N rows containing missing values
(geom\_smooth)'' can be ignored: they are simply noting that the
regression lines were cut off in some panels of the figure).

\begin{Shaded}
\begin{Highlighting}[]
\NormalTok{plot_regressions =}\StringTok{ }\NormalTok{function(type, title)\{}
  \NormalTok{x[}\KeywordTok{sample.int}\NormalTok{(}\KeywordTok{nrow}\NormalTok{(x)) , ] %>%}
\StringTok{  }\KeywordTok{subset}\NormalTok{(simulation_type ==}\StringTok{ }\NormalTok{type &}\StringTok{ }\NormalTok{method !=}\StringTok{ "null"}\NormalTok{) %>%}\StringTok{ }
\StringTok{  }\KeywordTok{ggplot}\NormalTok{(}\KeywordTok{aes}\NormalTok{(}\DataTypeTok{x =} \NormalTok{estimate, }\DataTypeTok{y =} \NormalTok{truth)) +}
\StringTok{    }\KeywordTok{facet_wrap}\NormalTok{(~method) +}
\StringTok{    }\KeywordTok{ylim}\NormalTok{(}\KeywordTok{min}\NormalTok{(x$truth), }\KeywordTok{max}\NormalTok{(x$truth)) +}\StringTok{ }
\StringTok{    }\KeywordTok{geom_point}\NormalTok{(}\KeywordTok{aes}\NormalTok{(}\DataTypeTok{color =} \KeywordTok{factor}\NormalTok{(n_sites)), }\DataTypeTok{size =} \FloatTok{0.2}\NormalTok{) +}\StringTok{ }
\StringTok{    }\KeywordTok{geom_smooth}\NormalTok{(}\DataTypeTok{method =} \StringTok{"lm"}\NormalTok{, }\DataTypeTok{color =} \StringTok{"black"}\NormalTok{, }\DataTypeTok{formula =} \NormalTok{y ~}\StringTok{ }\DecValTok{0} \NormalTok{+}\StringTok{ }\NormalTok{x, }
                \DataTypeTok{fullrange =} \OtherTok{TRUE}\NormalTok{) +}\StringTok{ }
\StringTok{    }\KeywordTok{geom_abline}\NormalTok{(}\DataTypeTok{intercept =} \DecValTok{0}\NormalTok{, }\DataTypeTok{slope =} \DecValTok{1}\NormalTok{, }\DataTypeTok{linetype =} \DecValTok{2}\NormalTok{) +}
\StringTok{    }\KeywordTok{theme_bw}\NormalTok{() +}\StringTok{ }
\StringTok{    }\KeywordTok{ggtitle}\NormalTok{(title)}
\NormalTok{\}}

\KeywordTok{plot_regressions}\NormalTok{(}\StringTok{"no_env"}\NormalTok{, }\StringTok{"'Baseline' regressions"}\NormalTok{)}
\end{Highlighting}
\end{Shaded}

\begin{verbatim}
## Warning: Removed 227 rows containing missing values (geom_smooth).
\end{verbatim}

\includegraphics{show-results_files/figure-latex/unnamed-chunk-5-1.png}

\begin{Shaded}
\begin{Highlighting}[]
\KeywordTok{plot_regressions}\NormalTok{(}\StringTok{"env"}\NormalTok{, }\StringTok{"'Environment' regressions"}\NormalTok{)}
\end{Highlighting}
\end{Shaded}

\begin{verbatim}
## Warning: Removed 188 rows containing missing values (geom_smooth).
\end{verbatim}

\includegraphics{show-results_files/figure-latex/unnamed-chunk-5-2.png}

\begin{Shaded}
\begin{Highlighting}[]
\KeywordTok{plot_regressions}\NormalTok{(}\StringTok{"abund"}\NormalTok{, }\StringTok{"'Abundance' regressions"}\NormalTok{)}
\end{Highlighting}
\end{Shaded}

\begin{verbatim}
## Warning: Removed 155 rows containing missing values (geom_smooth).
\end{verbatim}

\includegraphics{show-results_files/figure-latex/unnamed-chunk-5-3.png}

\paragraph{Average the R-squared values across methods and simulation
types}\label{average-the-r-squared-values-across-methods-and-simulation-types}

\begin{Shaded}
\begin{Highlighting}[]
\NormalTok{result_summary %>%}\StringTok{ }
\StringTok{  }\KeywordTok{group_by}\NormalTok{(method, simulation_type) %>%}\StringTok{ }
\StringTok{  }\KeywordTok{summarise}\NormalTok{(}\KeywordTok{mean}\NormalTok{(r2)) %>%}\StringTok{ }
\StringTok{  }\KeywordTok{spread}\NormalTok{(simulation_type, }\StringTok{`}\DataTypeTok{mean(r2)}\StringTok{`}\NormalTok{) %>%}
\StringTok{  }\KeywordTok{kable}\NormalTok{(}\DataTypeTok{digits =} \DecValTok{3}\NormalTok{)}
\end{Highlighting}
\end{Shaded}

\begin{longtable}[c]{@{}lrrr@{}}
\toprule
method & no\_env & env & abund\tabularnewline
\midrule
\endhead
Markov network & 0.525 & 0.451 & 0.384\tabularnewline
GLM & 0.472 & 0.405 & 0.283\tabularnewline
partial correlation & 0.403 & 0.322 & 0.200\tabularnewline
partial BayesComm & 0.394 & 0.302 & 0.166\tabularnewline
correlation & 0.291 & 0.183 & 0.117\tabularnewline
null & 0.227 & 0.125 & 0.075\tabularnewline
BayesComm & 0.206 & 0.110 & 0.060\tabularnewline
\bottomrule
\end{longtable}

\subsection{Save the finer-grained R-squared results as Figure
3}\label{save-the-finer-grained-r-squared-results-as-figure-3}

\begin{Shaded}
\begin{Highlighting}[]
\NormalTok{legend_name =}\StringTok{ }\KeywordTok{expression}\NormalTok{(Method~(mean~R^}\DecValTok{2}\NormalTok{))}

\KeywordTok{pdf}\NormalTok{(}\StringTok{"manuscript-materials/figures/performance.pdf"}\NormalTok{, }\DataTypeTok{width =} \DecValTok{8}\NormalTok{, }\DataTypeTok{height =} \FloatTok{2.5}\NormalTok{)}
\KeywordTok{ggplot}\NormalTok{(result_summary, }\KeywordTok{aes}\NormalTok{(}\DataTypeTok{x =} \NormalTok{n_sites, }\DataTypeTok{y =} \NormalTok{r2, }\DataTypeTok{col =} \NormalTok{method_r2, }\DataTypeTok{shape =} \NormalTok{method_r2)) +}\StringTok{ }
\StringTok{  }\KeywordTok{facet_grid}\NormalTok{(~simulation_type_long) +}\StringTok{ }
\StringTok{  }\KeywordTok{geom_line}\NormalTok{(}\DataTypeTok{size =} \NormalTok{.}\DecValTok{5}\NormalTok{) +}\StringTok{ }
\StringTok{  }\KeywordTok{geom_point}\NormalTok{(}\DataTypeTok{size =} \FloatTok{2.5}\NormalTok{, }\DataTypeTok{fill =} \StringTok{"white"}\NormalTok{) +}\StringTok{ }
\StringTok{  }\KeywordTok{scale_shape_manual}\NormalTok{(}\DataTypeTok{values =} \KeywordTok{c}\NormalTok{(}\DecValTok{16}\NormalTok{, }\DecValTok{22}\NormalTok{, }\DecValTok{17}\NormalTok{, }\DecValTok{23}\NormalTok{, }\DecValTok{18}\NormalTok{, }\DecValTok{24}\NormalTok{, }\DecValTok{15}\NormalTok{), }\DataTypeTok{name =} \NormalTok{legend_name) +}
\StringTok{  }\KeywordTok{geom_hline}\NormalTok{(}\DataTypeTok{yintercept =} \DecValTok{0}\NormalTok{, }\DataTypeTok{size =} \DecValTok{1}\NormalTok{/}\DecValTok{2}\NormalTok{) +}
\StringTok{  }\KeywordTok{geom_vline}\NormalTok{(}\DataTypeTok{xintercept =} \DecValTok{0}\NormalTok{, }\DataTypeTok{size =} \DecValTok{1}\NormalTok{) +}\StringTok{ }
\StringTok{  }\KeywordTok{coord_cartesian}\NormalTok{(}\DataTypeTok{ylim =} \KeywordTok{c}\NormalTok{(-.}\DecValTok{01}\NormalTok{, }\FloatTok{0.76}\NormalTok{)) +}\StringTok{ }
\StringTok{  }\KeywordTok{ylab}\NormalTok{(}\KeywordTok{expression}\NormalTok{(R^}\DecValTok{2}\NormalTok{)) +}\StringTok{ }
\StringTok{  }\KeywordTok{xlab}\NormalTok{(}\StringTok{"Number of sites (log scale)"}\NormalTok{) +}\StringTok{ }
\StringTok{  }\KeywordTok{scale_x_log10}\NormalTok{(}\DataTypeTok{breaks =} \KeywordTok{unique}\NormalTok{(x$n_sites), }\DataTypeTok{limits =} \KeywordTok{range}\NormalTok{(x$n_sites)) +}
\StringTok{  }\KeywordTok{theme_bw}\NormalTok{(}\DataTypeTok{base_size =} \DecValTok{11}\NormalTok{) +}\StringTok{ }
\StringTok{  }\KeywordTok{theme}\NormalTok{(}\DataTypeTok{panel.margin =} \NormalTok{grid::}\KeywordTok{unit}\NormalTok{(}\FloatTok{1.25}\NormalTok{, }\StringTok{"lines"}\NormalTok{)) +}\StringTok{ }
\StringTok{  }\KeywordTok{theme}\NormalTok{(}\DataTypeTok{panel.border =} \KeywordTok{element_blank}\NormalTok{(), }\DataTypeTok{axis.line =} \KeywordTok{element_blank}\NormalTok{()) +}\StringTok{ }
\StringTok{  }\KeywordTok{theme}\NormalTok{(}
    \DataTypeTok{panel.grid.minor =} \KeywordTok{element_blank}\NormalTok{(), }
    \DataTypeTok{panel.grid.major.y =} \KeywordTok{element_line}\NormalTok{(}\DataTypeTok{color =} \StringTok{"lightgray"}\NormalTok{, }\DataTypeTok{size =} \DecValTok{1}\NormalTok{/}\DecValTok{4}\NormalTok{), }
    \DataTypeTok{panel.grid.major.x =} \KeywordTok{element_blank}\NormalTok{()}
  \NormalTok{) +}\StringTok{ }
\StringTok{  }\KeywordTok{theme}\NormalTok{(}\DataTypeTok{strip.background =} \KeywordTok{element_blank}\NormalTok{(), }\DataTypeTok{legend.key =} \KeywordTok{element_blank}\NormalTok{()) +}\StringTok{ }
\StringTok{  }\KeywordTok{theme}\NormalTok{(}\DataTypeTok{plot.margin =} \NormalTok{grid::}\KeywordTok{unit}\NormalTok{(}\KeywordTok{c}\NormalTok{(.}\DecValTok{01}\NormalTok{, .}\DecValTok{01}\NormalTok{, .}\DecValTok{75}\NormalTok{, .}\DecValTok{1}\NormalTok{), }\StringTok{"lines"}\NormalTok{)) +}\StringTok{ }
\StringTok{  }\KeywordTok{theme}\NormalTok{(}
    \DataTypeTok{axis.title.x =} \KeywordTok{element_text}\NormalTok{(}\DataTypeTok{vjust =} \NormalTok{-}\FloatTok{0.2}\NormalTok{, }\DataTypeTok{size =} \DecValTok{12}\NormalTok{), }
    \DataTypeTok{axis.title.y =} \KeywordTok{element_text}\NormalTok{(}\DataTypeTok{angle =} \DecValTok{0}\NormalTok{, }\DataTypeTok{hjust =} \NormalTok{-.}\DecValTok{1}\NormalTok{, }\DataTypeTok{size =} \DecValTok{12}\NormalTok{)}
  \NormalTok{) +}\StringTok{ }
\StringTok{  }\KeywordTok{scale_color_brewer}\NormalTok{(}\DataTypeTok{palette =} \StringTok{"Dark2"}\NormalTok{, }\DataTypeTok{name =} \NormalTok{legend_name)}
\KeywordTok{dev.off}\NormalTok{()}
\end{Highlighting}
\end{Shaded}

\begin{verbatim}
## pdf 
##   2
\end{verbatim}

\subsection{Estimate uncertainty in mean
R-squared:}\label{estimate-uncertainty-in-mean-r-squared}

Fit a linear mixed model describing R-squared as a function of method,
landscape size, and simulation type. Note the small standard errors
associated with the effect of estimation method.

\begin{Shaded}
\begin{Highlighting}[]
\NormalTok{landscape_estimates =}\StringTok{ }\NormalTok{x %>%}\StringTok{ }
\StringTok{  }\KeywordTok{group_by}\NormalTok{(method, simulation_type) %>%}
\StringTok{  }\KeywordTok{do}\NormalTok{(}\KeywordTok{data.frame}\NormalTok{(., }\DataTypeTok{resids =} \KeywordTok{resids}\NormalTok{(.))) %>%}
\StringTok{  }\NormalTok{ungroup %>%}
\StringTok{  }\KeywordTok{group_by}\NormalTok{(method, simulation_type, rep_name, n_sites) %>%}
\StringTok{  }\KeywordTok{summarise}\NormalTok{(}\DataTypeTok{r2 =} \DecValTok{1} \NormalTok{-}\StringTok{ }\KeywordTok{sum}\NormalTok{(resids^}\DecValTok{2}\NormalTok{) /}\StringTok{ }\KeywordTok{sum}\NormalTok{(truth^}\DecValTok{2}\NormalTok{))}


\KeywordTok{summary}\NormalTok{(}
  \KeywordTok{lmer}\NormalTok{(}
    \NormalTok{r2 ~}\StringTok{ }\NormalTok{method +}\StringTok{ }\NormalTok{n_sites +}\StringTok{ }\NormalTok{simulation_type +}\StringTok{ }\NormalTok{(}\DecValTok{1}\NormalTok{|rep_name), }
    \DataTypeTok{data =} \NormalTok{landscape_estimates}
  \NormalTok{),}
  \DataTypeTok{correlation =} \OtherTok{FALSE}
\NormalTok{)}
\end{Highlighting}
\end{Shaded}

\begin{verbatim}
## Linear mixed model fit by REML ['lmerMod']
## Formula: r2 ~ method + n_sites + simulation_type + (1 | rep_name)
##    Data: landscape_estimates
## 
## REML criterion at convergence: -4945.2
## 
## Scaled residuals: 
##     Min      1Q  Median      3Q     Max 
## -7.3997 -0.6350  0.0758  0.6883  2.7104 
## 
## Random effects:
##  Groups   Name        Variance  Std.Dev.
##  rep_name (Intercept) 0.0008304 0.02882 
##  Residual             0.0113238 0.10641 
## Number of obs: 3150, groups:  rep_name, 150
## 
## Fixed effects:
##                             Estimate Std. Error t value
## (Intercept)               -2.481e-02  7.186e-03   -3.45
## methodcorrelation          6.965e-02  7.094e-03    9.82
## methodGLM                  2.544e-01  7.094e-03   35.85
## methodMarkov network       3.217e-01  7.094e-03   45.35
## methodnull                 1.504e-02  7.094e-03    2.12
## methodpartial BayesComm    1.616e-01  7.094e-03   22.78
## methodpartial correlation  1.796e-01  7.094e-03   25.31
## n_sites                    1.050e-04  2.690e-06   39.04
## simulation_typeenv         8.853e-02  7.402e-03   11.96
## simulation_typeno_env      1.795e-01  7.402e-03   24.25
\end{verbatim}

\section{C: Inferential statistics}\label{c-inferential-statistics}

Note that, as mentioned in the main text, inferential statistics are
only calculated for the \texttt{env} and \texttt{no\_env} simulations
(and not for the \texttt{abund} simulations).

\subsection{Identify statistical significance for the Markov network and
Pairs}\label{identify-statistical-significance-for-the-markov-network-and-pairs}

For Pairs, compare the Z-scores with Gaussian quantiles for 95\%
coverage; for the Markov network, use the approximate confidence
intervals estimated in Appendix 3.

\begin{Shaded}
\begin{Highlighting}[]
\NormalTok{pairs_summary =}\StringTok{ }\NormalTok{x[x$method ==}\StringTok{ "null"} \NormalTok{&}\StringTok{ }\NormalTok{x$simulation_type !=}\StringTok{ "abund"}\NormalTok{, ]}
\NormalTok{markov_summary =}\StringTok{ }\NormalTok{x[x$method ==}\StringTok{ "Markov network"} \NormalTok{&}\StringTok{ }\NormalTok{x$simulation_type !=}\StringTok{ "abund"}\NormalTok{, ]}


\CommentTok{# Pairs's Z score is based on C-scores, which are positive when species are }
\CommentTok{# disaggregated.  So significantly negative scores imply positive interactions}
\CommentTok{# and vice versa}
\NormalTok{pairs_summary$sig_pos =}\StringTok{ }\NormalTok{pairs_summary$estimate <}\StringTok{ }\KeywordTok{qnorm}\NormalTok{(.}\DecValTok{025}\NormalTok{)}
\NormalTok{pairs_summary$sig_neg =}\StringTok{ }\NormalTok{pairs_summary$estimate >}\StringTok{ }\KeywordTok{qnorm}\NormalTok{(.}\DecValTok{975}\NormalTok{)}

\CommentTok{# Markov network is significant when lower bound is above zero or lower bound is}
\CommentTok{# below zero}
\NormalTok{markov_summary$sig_pos =}\StringTok{ }\NormalTok{markov_summary$lower >}\StringTok{ }\DecValTok{0}
\NormalTok{markov_summary$sig_neg =}\StringTok{ }\NormalTok{markov_summary$upper <}\StringTok{ }\DecValTok{0}
\end{Highlighting}
\end{Shaded}

The \texttt{error\_smoother} is a function that fits a kernel smoother
model to estimate the probability that a model estimate will match some
criterion (e.g. statistical significance) for some parameter as a
function of that parameter's ``true'' value. This function is used in
Figure 4C and in calculating Type I error rates below.

\begin{Shaded}
\begin{Highlighting}[]
\NormalTok{truth_seq =}\StringTok{ }\KeywordTok{seq}\NormalTok{(}\KeywordTok{min}\NormalTok{(markov_summary$truth), }\KeywordTok{max}\NormalTok{(markov_summary$truth), }\DataTypeTok{length =} \DecValTok{1000}\NormalTok{)}

\NormalTok{error_smoother =}\StringTok{ }\NormalTok{function(data, f, }\DataTypeTok{values =} \NormalTok{truth_seq)\{}
  \NormalTok{out =}\StringTok{ }\KeywordTok{ksmooth}\NormalTok{(}
    \DataTypeTok{x =} \NormalTok{data$truth,}
    \DataTypeTok{y =} \KeywordTok{as.numeric}\NormalTok{(}\KeywordTok{f}\NormalTok{(data)),}
    \DataTypeTok{x.points =} \NormalTok{values,}
    \DataTypeTok{kernel =} \StringTok{"normal"}\NormalTok{,}
    \DataTypeTok{bandwidth =} \FloatTok{0.25}
  \NormalTok{)}
  
  \NormalTok{out$y[}\KeywordTok{is.na}\NormalTok{(out$y)] =}\StringTok{ }\DecValTok{0}
  
  \NormalTok{out}
\NormalTok{\}}
\end{Highlighting}
\end{Shaded}

\subsection{Create Figure 4}\label{create-figure-4}

Panels A and B plot the point estimates or test statistics for different
models against one another.

Panel C shows the probability of rejecting the null hypothesis (of no
interaction between two species) in the opposite direction of the true
value (see main text).

\begin{Shaded}
\begin{Highlighting}[]
\KeywordTok{pdf}\NormalTok{(}\StringTok{"manuscript-materials/figures/error_rates.pdf"}\NormalTok{, }\DataTypeTok{height =} \FloatTok{8.5}\NormalTok{, }\DataTypeTok{width =} \FloatTok{8.5}\NormalTok{/}\DecValTok{3}\NormalTok{)}
\KeywordTok{par}\NormalTok{(}\DataTypeTok{mfrow =} \KeywordTok{c}\NormalTok{(}\DecValTok{3}\NormalTok{, }\DecValTok{1}\NormalTok{))}

\CommentTok{# Calculate p(confidently wrong) for each model as a function of}
\CommentTok{# the true values}
\NormalTok{confidently_wrong =}\StringTok{ }\NormalTok{function(data)\{}
  \NormalTok{(data$sig_pos &}\StringTok{ }\NormalTok{data$truth <}\StringTok{ }\DecValTok{0}\NormalTok{) |}\StringTok{ }\NormalTok{(data$sig_neg &}\StringTok{ }\NormalTok{data$truth >}\StringTok{ }\DecValTok{0}\NormalTok{)}
\NormalTok{\}}
\NormalTok{smoothed_markov =}\StringTok{ }\KeywordTok{error_smoother}\NormalTok{(markov_summary, confidently_wrong)}
\NormalTok{smoothed_pairs =}\StringTok{ }\KeywordTok{error_smoother}\NormalTok{(pairs_summary, confidently_wrong)}

\CommentTok{# Compare estimates}
\NormalTok{spread_estimates =}\StringTok{ }\NormalTok{x %>%}
\StringTok{  }\NormalTok{dplyr::}\KeywordTok{select}\NormalTok{(-lower, -upper, -X) %>%}
\StringTok{  }\KeywordTok{spread}\NormalTok{(method, estimate) %>%}
\StringTok{  }\KeywordTok{na.omit}\NormalTok{()}

\CommentTok{# R-squared for null versus correlation}
\KeywordTok{round}\NormalTok{(}
  \KeywordTok{summary}\NormalTok{(}\KeywordTok{lm}\NormalTok{(null ~}\StringTok{ }\KeywordTok{I}\NormalTok{(correlation*}\KeywordTok{sqrt}\NormalTok{(n_sites)), }\DataTypeTok{data =} \NormalTok{spread_estimates))$r.squared, }
  \DecValTok{2}
\NormalTok{)}
\end{Highlighting}
\end{Shaded}

\begin{verbatim}
## [1] 0.95
\end{verbatim}

\begin{Shaded}
\begin{Highlighting}[]
\CommentTok{# R-squared for glm markov network versus glm}
\KeywordTok{round}\NormalTok{(}\KeywordTok{summary}\NormalTok{(}\KeywordTok{lm}\NormalTok{(}\StringTok{`}\DataTypeTok{Markov network}\StringTok{`} \NormalTok{~}\StringTok{ }\NormalTok{GLM, }\DataTypeTok{data =} \NormalTok{spread_estimates))$r.squared, }\DecValTok{2}\NormalTok{)}
\end{Highlighting}
\end{Shaded}

\begin{verbatim}
## [1] 0.94
\end{verbatim}

\begin{Shaded}
\begin{Highlighting}[]
\KeywordTok{with}\NormalTok{(}
  \NormalTok{spread_estimates, }
  \KeywordTok{plot}\NormalTok{(}
    \StringTok{`}\DataTypeTok{Markov network}\StringTok{`}\NormalTok{, }
    \NormalTok{GLM, }
    \DataTypeTok{pch =} \StringTok{"."}\NormalTok{, }
    \DataTypeTok{col =} \StringTok{"#00000020"}\NormalTok{,}
    \DataTypeTok{ylab =} \StringTok{"Markov network estimate"}\NormalTok{,}
    \DataTypeTok{xlab =} \StringTok{"GLM estimate"}\NormalTok{,}
    \DataTypeTok{bty =} \StringTok{"l"}
  \NormalTok{)}
\NormalTok{)}
\KeywordTok{mtext}\NormalTok{(}\StringTok{"A. Markov network estimates}\CharTok{\textbackslash{}n}\StringTok{vs. GLM estimates"}\NormalTok{, }\DataTypeTok{adj =} \DecValTok{0}\NormalTok{, }\DataTypeTok{side =} \DecValTok{3}\NormalTok{, }
      \DataTypeTok{font =} \DecValTok{2}\NormalTok{, }\DataTypeTok{line =} \FloatTok{1.2}\NormalTok{, }\DataTypeTok{cex =} \NormalTok{.}\DecValTok{9}\NormalTok{) }
\KeywordTok{abline}\NormalTok{(}\KeywordTok{lm}\NormalTok{(}\StringTok{`}\DataTypeTok{Markov network}\StringTok{`} \NormalTok{~}\StringTok{ }\NormalTok{GLM, }\DataTypeTok{data =} \NormalTok{spread_estimates))}
\KeywordTok{text}\NormalTok{(}\DecValTok{0}\NormalTok{, }\DecValTok{5}\NormalTok{, }\KeywordTok{expression}\NormalTok{(R^}\DecValTok{2}\NormalTok{==}\FloatTok{0.94}\NormalTok{))}

\KeywordTok{with}\NormalTok{(}
  \NormalTok{spread_estimates, }
  \KeywordTok{plot}\NormalTok{(}
    \NormalTok{correlation *}\StringTok{ }\KeywordTok{sqrt}\NormalTok{(n_sites), }
    \NormalTok{null, }
    \DataTypeTok{pch =} \StringTok{"."}\NormalTok{, }
    \DataTypeTok{col =} \StringTok{"#00000020"}\NormalTok{,}
    \DataTypeTok{ylab =} \StringTok{"Z-score"}\NormalTok{,}
    \DataTypeTok{xlab =} \KeywordTok{expression}\NormalTok{(}\StringTok{"correlation"} \NormalTok\StringTok{ }\KeywordTok{sqrt}\NormalTok{(number~}\ErrorTok{~}\NormalTok{of~}\ErrorTok{~}\NormalTok{sites)),}
    \DataTypeTok{bty =} \StringTok{"l"}
  \NormalTok{)}
\NormalTok{)}
\KeywordTok{mtext}\NormalTok{(}\StringTok{"B. Null model estimates vs.}\CharTok{\textbackslash{}n}\StringTok{scaled correlation coefficients"}\NormalTok{, }
      \DataTypeTok{adj =} \DecValTok{0}\NormalTok{, }\DataTypeTok{side =} \DecValTok{3}\NormalTok{, }\DataTypeTok{font =} \DecValTok{2}\NormalTok{, }\DataTypeTok{line =} \FloatTok{1.2}\NormalTok{, }\DataTypeTok{cex =} \NormalTok{.}\DecValTok{9}\NormalTok{) }
\KeywordTok{abline}\NormalTok{(}\KeywordTok{lm}\NormalTok{(null ~}\StringTok{ }\KeywordTok{I}\NormalTok{(correlation*}\KeywordTok{sqrt}\NormalTok{(n_sites)), }\DataTypeTok{data =} \NormalTok{spread_estimates))}
\KeywordTok{text}\NormalTok{(}\DecValTok{10}\NormalTok{, }\DecValTok{12}\NormalTok{, }\KeywordTok{expression}\NormalTok{(R^}\DecValTok{2}\NormalTok{==}\FloatTok{0.95}\NormalTok{))}

\KeywordTok{plot}\NormalTok{(}
  \NormalTok{smoothed_pairs$x,}
  \NormalTok{smoothed_pairs$y,}
  \DataTypeTok{type =} \StringTok{"l"}\NormalTok{,}
  \DataTypeTok{xlab =} \StringTok{"}\CharTok{\textbackslash{}"}\StringTok{True}\CharTok{\textbackslash{}"}\StringTok{ interaction strength"}\NormalTok{,}
  \DataTypeTok{ylab =} \StringTok{"P(confidently predict wrong sign)"}\NormalTok{,}
  \DataTypeTok{bty =} \StringTok{"l"}\NormalTok{,}
  \DataTypeTok{yaxs =} \StringTok{"i"}\NormalTok{,}
  \DataTypeTok{col =} \DecValTok{2}\NormalTok{,}
  \DataTypeTok{ylim =} \KeywordTok{c}\NormalTok{(}\DecValTok{0}\NormalTok{, .}\DecValTok{3}\NormalTok{),}
  \DataTypeTok{lwd =} \DecValTok{2}
\NormalTok{)}
\KeywordTok{mtext}\NormalTok{(}\StringTok{"C. Error rate vs.}\CharTok{\textbackslash{}n}\StringTok{interaction strength"}\NormalTok{, }\DataTypeTok{side =} \DecValTok{3}\NormalTok{, }\DataTypeTok{adj =} \DecValTok{0}\NormalTok{, }
      \DataTypeTok{font =} \DecValTok{2}\NormalTok{, }\DataTypeTok{line =} \FloatTok{1.2}\NormalTok{, }\DataTypeTok{cex =} \NormalTok{.}\DecValTok{9}\NormalTok{) }
\KeywordTok{lines}\NormalTok{(smoothed_markov$x, smoothed_markov$y, }\DataTypeTok{lwd =} \DecValTok{2}\NormalTok{)}
\KeywordTok{legend}\NormalTok{(}\StringTok{"topleft"}\NormalTok{, }\DataTypeTok{lwd =} \DecValTok{2}\NormalTok{, }\DataTypeTok{legend =} \KeywordTok{c}\NormalTok{(}\StringTok{"Null model"}\NormalTok{, }\StringTok{"Markov network"}\NormalTok{), }
       \DataTypeTok{col =} \KeywordTok{c}\NormalTok{(}\DecValTok{2}\NormalTok{, }\DecValTok{1}\NormalTok{), }\DataTypeTok{bty =} \StringTok{"n"}\NormalTok{)}
\KeywordTok{dev.off}\NormalTok{()}
\end{Highlighting}
\end{Shaded}

\begin{verbatim}
## pdf 
##   2
\end{verbatim}

\subsection{Summarize inferential
statistics:}\label{summarize-inferential-statistics}

\paragraph{P(Pairs confidently wrong):}\label{ppairs-confidently-wrong}

\begin{Shaded}
\begin{Highlighting}[]
\KeywordTok{with}\NormalTok{(pairs_summary,  }\KeywordTok{mean}\NormalTok{((sig_neg &}\StringTok{ }\NormalTok{truth >}\StringTok{ }\DecValTok{0}\NormalTok{) |}\StringTok{ }\NormalTok{(sig_pos &}\StringTok{ }\NormalTok{truth <}\StringTok{ }\DecValTok{0}\NormalTok{)))}
\end{Highlighting}
\end{Shaded}

\begin{verbatim}
## [1] 0.1234429
\end{verbatim}

\paragraph{P(Markov network confidently
wrong)}\label{pmarkov-network-confidently-wrong}

\begin{Shaded}
\begin{Highlighting}[]
\KeywordTok{with}\NormalTok{(markov_summary, }\KeywordTok{mean}\NormalTok{((sig_neg &}\StringTok{ }\NormalTok{truth >}\StringTok{ }\DecValTok{0}\NormalTok{) |}\StringTok{ }\NormalTok{(sig_pos &}\StringTok{ }\NormalTok{truth <}\StringTok{ }\DecValTok{0}\NormalTok{)))}
\end{Highlighting}
\end{Shaded}

\begin{verbatim}
## [1] 0.01529933
\end{verbatim}

\paragraph{P(Pairs rejects null)}\label{ppairs-rejects-null}

\begin{Shaded}
\begin{Highlighting}[]
\KeywordTok{with}\NormalTok{(pairs_summary,  }\KeywordTok{mean}\NormalTok{(sig_neg |}\StringTok{ }\NormalTok{sig_pos))}
\end{Highlighting}
\end{Shaded}

\begin{verbatim}
## [1] 0.4640395
\end{verbatim}

\paragraph{P(Markov network rejects
null)}\label{pmarkov-network-rejects-null}

\begin{Shaded}
\begin{Highlighting}[]
\KeywordTok{with}\NormalTok{(markov_summary,  }\KeywordTok{mean}\NormalTok{(sig_neg |}\StringTok{ }\NormalTok{sig_pos))}
\end{Highlighting}
\end{Shaded}

\begin{verbatim}
## [1] 0.22179
\end{verbatim}

\subsection{Type I error rates:}\label{type-i-error-rates}

Type I error rates (probability of rejecting the null hypothesis given a
``true'' interaction strength of zero, according to the kernel smoother
defined by \texttt{error\_smoother} above).

\paragraph{Markov network Type I error
rates}\label{markov-network-type-i-error-rates}

\begin{Shaded}
\begin{Highlighting}[]
\NormalTok{markov_summary %>%}\StringTok{ }
\StringTok{  }\KeywordTok{filter}\NormalTok{(}\KeywordTok{abs}\NormalTok{(truth) <}\StringTok{ }\DecValTok{1}\NormalTok{) %>%}\StringTok{ }
\StringTok{  }\KeywordTok{group_by}\NormalTok{(simulation_type) %>%}\StringTok{ }
\StringTok{  }\KeywordTok{do}\NormalTok{(}\KeywordTok{data.frame}\NormalTok{(}
    \StringTok{`}\DataTypeTok{Type I error rate}\StringTok{`} \NormalTok{=}\StringTok{ }
\StringTok{      }\KeywordTok{error_smoother}\NormalTok{(., function(x)\{x$sig_pos |}\StringTok{ }\NormalTok{x$sig_neg\}, }\DataTypeTok{values =} \DecValTok{0}\NormalTok{)$y}
  \NormalTok{)) %>%}
\StringTok{  }\KeywordTok{kable}\NormalTok{(}\DataTypeTok{digits =} \DecValTok{3}\NormalTok{)}
\end{Highlighting}
\end{Shaded}

\begin{longtable}[c]{@{}lr@{}}
\toprule
simulation\_type & Type.I.error.rate\tabularnewline
\midrule
\endhead
env & 0.138\tabularnewline
no\_env & 0.024\tabularnewline
\bottomrule
\end{longtable}

\paragraph{Null model Type I error
rates}\label{null-model-type-i-error-rates}

\begin{Shaded}
\begin{Highlighting}[]
\NormalTok{pairs_summary %>%}\StringTok{ }
\StringTok{  }\KeywordTok{filter}\NormalTok{(}\KeywordTok{abs}\NormalTok{(truth) <}\StringTok{ }\DecValTok{1}\NormalTok{) %>%}\StringTok{ }
\StringTok{  }\KeywordTok{group_by}\NormalTok{(simulation_type) %>%}\StringTok{ }
\StringTok{  }\KeywordTok{do}\NormalTok{(}\KeywordTok{data.frame}\NormalTok{(}
    \StringTok{`}\DataTypeTok{Type I error rate}\StringTok{`} \NormalTok{=}\StringTok{ }
\StringTok{      }\KeywordTok{error_smoother}\NormalTok{(., function(x)\{x$sig_pos |}\StringTok{ }\NormalTok{x$sig_neg\}, }\DataTypeTok{values =} \DecValTok{0}\NormalTok{)$y}
  \NormalTok{)) %>%}
\StringTok{  }\KeywordTok{kable}\NormalTok{(}\DataTypeTok{digits =} \DecValTok{3}\NormalTok{)}
\end{Highlighting}
\end{Shaded}

\begin{longtable}[c]{@{}lr@{}}
\toprule
simulation\_type & Type.I.error.rate\tabularnewline
\midrule
\endhead
env & 0.510\tabularnewline
no\_env & 0.298\tabularnewline
\bottomrule
\end{longtable}

\subsection{\texorpdfstring{Plot Markov network confidence interval
coverage versus true \(\beta\)
value}{Plot Markov network confidence interval coverage versus true \textbackslash{}beta value}}\label{plot-markov-network-confidence-interval-coverage-versus-true-beta-value}

The top and bottom 0.5\% of the distribution have been omitted to
prevent bad behavior by the smoother (a generalized additive model) in
the tails. The horizontal red line shows the nominal 95\% coverage rate
and the black vertical line marks where the ``true'' value of \(\beta\)
was zero. The Type I error rates calculated above for the Markov network
closely match (one minus) the values of these curves at zero.

\begin{Shaded}
\begin{Highlighting}[]
\NormalTok{coverage_data =}\StringTok{ }\NormalTok{markov_summary %>%}
\StringTok{  }\KeywordTok{filter}\NormalTok{(}\KeywordTok{percent_rank}\NormalTok{(truth) >}\StringTok{ }\NormalTok{.}\DecValTok{005} \NormalTok{&}\StringTok{ }\KeywordTok{percent_rank}\NormalTok{(truth) <}\StringTok{ }\NormalTok{.}\DecValTok{995}\NormalTok{) %>%}
\StringTok{  }\KeywordTok{mutate}\NormalTok{(}\DataTypeTok{covered =} \NormalTok{truth >}\StringTok{ }\NormalTok{lower &}\StringTok{ }\NormalTok{truth <}\StringTok{ }\NormalTok{upper) %>%}
\StringTok{  }\KeywordTok{mutate}\NormalTok{(}\DataTypeTok{simulation_type =} \KeywordTok{factor}\NormalTok{(simulation_type, }\KeywordTok{c}\NormalTok{(}\StringTok{"no_env"}\NormalTok{, }\StringTok{"env"}\NormalTok{, }\StringTok{"abund"}\NormalTok{)))}

\KeywordTok{ggplot}\NormalTok{(coverage_data, }\KeywordTok{aes}\NormalTok{(}\DataTypeTok{x =} \NormalTok{truth, }\DataTypeTok{y =} \KeywordTok{as.integer}\NormalTok{(covered))) +}\StringTok{ }
\StringTok{  }\KeywordTok{facet_grid}\NormalTok{(~simulation_type) +}\StringTok{ }
\StringTok{  }\KeywordTok{geom_smooth}\NormalTok{(}\DataTypeTok{method =} \NormalTok{gam, }\DataTypeTok{formula =} \NormalTok{y ~}\StringTok{ }\KeywordTok{s}\NormalTok{(x), }\DataTypeTok{method.args =} \KeywordTok{list}\NormalTok{(}\DataTypeTok{family =} \NormalTok{binomial)) +}\StringTok{ }
\StringTok{  }\KeywordTok{theme_bw}\NormalTok{() +}\StringTok{ }
\StringTok{  }\KeywordTok{coord_cartesian}\NormalTok{(}\DataTypeTok{ylim =} \KeywordTok{c}\NormalTok{(}\DecValTok{0}\NormalTok{, }\DecValTok{1}\NormalTok{), }\DataTypeTok{expand =} \OtherTok{FALSE}\NormalTok{) +}\StringTok{ }
\StringTok{  }\KeywordTok{geom_vline}\NormalTok{(}\DataTypeTok{xintercept =} \DecValTok{0}\NormalTok{) +}\StringTok{ }
\StringTok{  }\KeywordTok{geom_hline}\NormalTok{(}\DataTypeTok{yintercept =} \FloatTok{0.95}\NormalTok{, }\DataTypeTok{color =} \StringTok{"red"}\NormalTok{) +}\StringTok{ }
\StringTok{  }\KeywordTok{ylab}\NormalTok{(}\StringTok{"Coverage"}\NormalTok{)}
\end{Highlighting}
\end{Shaded}

\includegraphics{show-results_files/figure-latex/unnamed-chunk-18-1.pdf}

\subsection{\texorpdfstring{Plot power versus true \(\beta\)
value}{Plot power versus true \textbackslash{}beta value}}\label{plot-power-versus-true-beta-value}

Calculate p(confidently detect sign of \(\beta\)) as a function of the
``true''" values. Pairs has a somewhat higher rate of rejecting the null
hypothesis in the correct direction than the Markov network, but the
benefit of this increase in power is more than offset by the large
increases in Type I errors and sign errors demonstrated above and in
Figure 4 of the main text.

\begin{Shaded}
\begin{Highlighting}[]
\NormalTok{confidently_right =}\StringTok{ }\NormalTok{function(data)\{}
  \NormalTok{(data$sig_pos &}\StringTok{ }\NormalTok{data$truth >}\StringTok{ }\DecValTok{0}\NormalTok{) |}\StringTok{ }\NormalTok{(data$sig_neg &}\StringTok{ }\NormalTok{data$truth <}\StringTok{ }\DecValTok{0}\NormalTok{)}
\NormalTok{\}}

\KeywordTok{plot}\NormalTok{(}\KeywordTok{error_smoother}\NormalTok{(markov_summary, confidently_right), }
     \DataTypeTok{type =} \StringTok{"l"}\NormalTok{, }
     \DataTypeTok{xlim =} \KeywordTok{c}\NormalTok{(-}\DecValTok{5}\NormalTok{, }\DecValTok{5}\NormalTok{),}
     \DataTypeTok{xlab =} \StringTok{"'True' coefficient value"}\NormalTok{,}
     \DataTypeTok{ylab =} \StringTok{"Power"}\NormalTok{,}
     \DataTypeTok{ylim =} \KeywordTok{c}\NormalTok{(}\DecValTok{0}\NormalTok{, }\DecValTok{1}\NormalTok{),}
     \DataTypeTok{yaxs =} \StringTok{"i"}
\NormalTok{)}
\KeywordTok{lines}\NormalTok{(}\KeywordTok{error_smoother}\NormalTok{(pairs_summary, confidently_right), }\DataTypeTok{col =} \DecValTok{2}\NormalTok{)}
\KeywordTok{legend}\NormalTok{(}\StringTok{"topleft"}\NormalTok{, }\DataTypeTok{legend =} \KeywordTok{c}\NormalTok{(}\StringTok{"Markov network"}\NormalTok{, }\StringTok{"Pairs"}\NormalTok{), }\DataTypeTok{lty =} \DecValTok{1}\NormalTok{, }\DataTypeTok{col =} \DecValTok{1}\NormalTok{:}\DecValTok{2}\NormalTok{)}
\end{Highlighting}
\end{Shaded}

\includegraphics{show-results_files/figure-latex/unnamed-chunk-19-1.pdf}

\end{document}
