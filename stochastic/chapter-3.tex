\section{Abstract}\label{abstract}

Species interactions are believed to play an important role in community
structure, but detecting their influence in the co-occurrence patterns
of observed communities has stymied ecologists for decades. While Markov
networks (undirected graphical models also known as Markov random
fields) represent a promising approach to this long-standing problem by
isolating direct interactions from indirect ones, the methods ecologists
have suggested for fitting these models are limited to small communities
with about 20 species or fewer. Additionally, the methods that have been
proposed so far do not account for environmental heterogeneity, and thus
attempt to explain all of the co-occurrence patterns in ecological data
sets with species interactions. In this paper, I introduce stochastic
approximation as an alternative method for fitting these models that
addresses both of these problems, making it feasible to use Markov
networksin cases where each species responds to multiple abiotic factors
and hundreds of competitors or mutualists. While stochastic
approximation does introduce some sampling noise during the optimization
process, it still converges to the maximum likelihood estimate for
species' interaction coefficients with probability one.

\section{Introduction}\label{introduction}

To the extent that species interactions are important for community
assembly, ecologists generally expect them to leave a signature on
species' co-occurrence patterns. Using that signature to infer the
underlying species interactions from observational data has been much
harder, however. Disagreements about how to draw these inferences led to
an acrimonious, decade-long argument among community ecologists (Lewin
1983), where essentially all of the proposed statistical approaches were
criticized for having high error rates, a poor match with the underlying
ecological questions, computational infeasibility, or all three (Connor
and Simberloff 1979; Gilpin and Diamond 1982; Strong et al. 1984;
Hastings 1987). As documented below, ecologists have, for the most part,
not solved these problems in the ensuing decades. Resolving these issues
would solve a longstanding issue in community ecology and make a
significant contribution to other fields, such as species distribution
modeling (Kissling et al. 2012).

During the time that ecologists have struggled to identify the
interactions among dozens of species, bioinformaticians have largely
solved the analogous problem of estimating interactions among thousands
of genes from their co-expression levels (e.g. Friedman, Hastie, and
Tibshirani (2008)). The discrepancy is due to the fact that ecologists
have chosen to focus, almost exclusively, on overall co-occurrence rates
(Connor and Simberloff 1979; Gotelli and Ulrich 2009; Gilpin and Diamond
1982; Veech 2013) or on closely-related values such as correlation
coefficients (Pollock et al. 2014; see Faisal et al. 2010 and Harris
2015 for two recent exceptions). As ecologists know, however, the
correlation between two species will often reflect other factors beyond
their direct pairwise interactions (e.g.~abiotic influences and indirect
biotic effects; Figure 1). When these factors are not accounted for,
this reliance on overall co-occurrence rates can reliably lead to
incorrect inferences (D. J. Harris 2015). While some methods have been
proposed for incorporating a small number of specific factors (such as
geographic or environmental dissimilarity) into ecologists' null models
at a time (Lessard et al. 2011), we still lack a good way to scale these
approaches up for cases with many biotic and abiotic factors acting
simultaneously.

\begin{figure}[htbp]
\centering
\includegraphics{figure-1.pdf}
\caption{A hypothetical network of three species, two of which depend on
an abiotic factor (annual rainfall). \textbf{A.} The ``true'' network
involves the caterpillar exploiting the two tree species as well as
competition between the two trees. The two trees also benefit from
rainfall. \textbf{B.} Although the two trees directly reduce one
another's occurrence probabilities via competition, they tend to occur
in the same areas (with sufficient water and without too many
herbivores). An analysis that ignores these other factors could
mistakenly conclude that the two trees were mutualists because of their
high co-occurrence rate (circled ``+'' sign). \textbf{C.} The methods
presented here are designed to estimate the direct interactions between
each pair of species, after accounting for other species and abiotic
factors that could affect their co-occurrence rates (note that the
circled interaction between the trees has the correct sign). Note that,
with observational data, the bi-directional effects of species
interactions (pairs of arrows between species) need to be collapsed to a
single number describing the conditional relationship between species
(one double-headed arrow).}
\end{figure}

A number of procedures have been developed for controlling for the
influence of extraneous factors and focusing on the direct relationship
between a single pair of variables, i.e.~for estimating the conditional
relationship between them. The most familiar of these is the partial
correlation. Rather than describing the overall relationship between two
variables across all conditions, partial correlations (like regression
coefficients) describe the portion of the relationship that remains
after the other variables in the data have been accounted for. D. J.
Harris (2015) found that the partial correlation can do extract accurate
information about pairwise species interactions from binary
co-occurrence data in some circumstances, but the fact that this
approach assumes multivariate Gaussian data makes it less appealing.

Recently, several papers have suggested that ecologists could use Markov
networks (undirected graphical models also known as Markov random
fields) to estimate conditional relationships from binary data by
maximum entropy or maximum likelihood (Azaele et al. (2010); D. J.
Harris (2015)). A Markov network defines a probability distribution over
possible binary species assemblages, and its coefficients (including one
coefficient describing the conditional relationship between each pair of
species in the network). This approach is optimal in the sense that it
produces a model that matches the observed properties of the data
(i.e.~the occurrence and co-occurrence rates), and that it does so with
the fewest possible parameters or constraints (i.e.~the Markov network
has the most information entropy of any possible model satisfying its
constraints; Azaele et al. 2010). As shown in D. J. Harris (2015),
Markov networks do a better job of estimating species' influences on one
another better than a number of existing methods, particularly when
sample sizes are low.

Unfortunately, Markov networks have an intractable likelihood function
whose computational difficulty more than doubles each time a new species
is added to the model. This exponential growth in the likelihood
function's complexity means that the methods that worked for Azaele et
al. (2010) and D. J. Harris (2015) with networks of 20 species would be
completely infeasible for networks with 50 species, requiring over a
billion times more computational effort
(\(2^{50}/2^{20} \approx 10^{9}\)). Extending the method to account for
abiotic variation among sites on the landscape would require repeating
these expensive computations independently for every site, increasing
the computational burden even further. If ecologists want to apply these
methods to larger problems, a different model-fitting algorithm would be
needed.

In this paper, I present a different way of optimizing the likelihood,
called ``stochastic approximation'' (Robbins and Monro 1951;
Salakhutdinov and Hinton 2012), which replaces the intractable
computations with tractable Monte Carlo estimates of the same
quantities. Despite the sampling error introduced by this substitution,
stochastic approximation provides strong guarantees for eventual
convergence to the maximum likelihood estimate (Younes 1999;
Salakhutdinov and Hinton 2012). This change in approach makes it
feasible to estimate the interactions among hundreds of species from
observational data while also accounting for possible responses to
abiotic factors (such as those used in species distribution models),
giving ecologists some power to distinguish between species pairs that
co-occur due to shared environmental tolerances from those that co-occur
due to direct interactions like mutualism.

\section{Methods}\label{methods}

\subsection{Markov networks}\label{markov-networks}

As discussed in Azaele et al. (2010) and D. J. Harris (2015), Markov
networks such as the Ising model (Cipra 1987) can be used to describe
community structure as follows. Each species is represented by a binary
random variable, describing whether it is present (1) or absent (0). A
species' conditional probability of presence under a given set of biotic
and abiotic conditions depends on its coefficients. The coefficients
linking species to one another describe their conditional associations:
all else equal, a species will occur less often in the presence of its
competitors and other species with which it is negatively associated,
and more often in the presence of its mutualists:

\[ p(y_i | \vec{y}_{j \neq i}) = \mathrm{logistic}\Big(\alpha_i + \sum_{j \neq i}{\beta_{ij}y_j}\Big), \]

where \(y_i\) is 1 when species \(i\) is present and 0 when it is
absent, \(\alpha_i\) is an intercept term, \(\beta_ij\) represents the
conditional relationship between species \(i\) and species \(j\), and
\(\mathrm{logistic}(x) = e^x / (1 + e^x)\). These conditional
probabilities can be combined to calculate the probability of observing
any given combination of presences and absences, up to a multiplicative
constant (Lee and Hastie 2012; Murphy 2012):

\[p(\vec{y}) \propto \mathrm{exp}\Big({\sum_{i}{\alpha_i y_i} + \sum_{j \neq i}}{\beta_{ij} y_i y_j}\Big).\]

Unfortunately, the exponentially large number of possible assemblages
makes the normalizing constant for this joint probability distribution
intractable when the number of species is larger than 20 or so. However,
it is generally straightforward to simulate examples of assemblages that
are consistent with graphical models such as these, using Monte Carlo
techniques like Gibbs sampling (Salakhutdinov and Hinton 2012). Gibbs
sampling is especially convenient for these models because it only
requires the conditional probabilities, which are straightforward to
compute (Harris 2015; Appendix).

In each of the scenarios below, the ``observed'' landscapes were
simulated using Gibbs sampling from a pre-specified set of ``true''
parameters, then stochastic approximation was used to recover these
parameters from the simulated presence-absence data.

\subsection{Conditioning the network on the abiotic
environment}\label{conditioning-the-network-on-the-abiotic-environment}

The Markov networks fit in D. J. Harris (2015) treated the \(\alpha_i\)
terms from Equations 1 and 2 as constants, meaning that the models
expected each species to have the same conditional occurrence
probability in any location with a given set of facilitators and
competitors. In this paper, I extend the model so that the local abiotic
conditions can also affect species occurrence probabilities directly.
Specifically, the \(\alpha_i\) terms in these analyses are linear
combinations of the local environmental conditions, represented by
\(x_1\) through \(x_5\). By estimating the coefficients associated with
each of these \(x\) variables for each species, we can account for their
responses to the abiotic environment as well as to one another. In other
fields, a model like this one, where a Markov network's parameters
depend on external factors is called a conditional random field (Lee and
Hastie 2012); these models have been used in a variety of contexts, from
language models to image processing (Murphy 2012).

\subsection{Coefficient estimation with stochastic
approximation}\label{coefficient-estimation-with-stochastic-approximation}

A Markov network describes a probability distribution over possible
assemblages. Since the model is a member of the exponential family it
can be summarized without loss of information by its ``sufficient
statistics''. For the model presented here, the sufficient statistics
include: the number of occurrences for each species, the number of
co-occurrences between each species, and the cross-product between the
species and environment matrices (Azaele et al. 2010; Lee and Hastie
2012; Murphy 2012). Finding the maximum likelihood estimate for the
model parameters is equivalent to minimizing the discrepancy between the
values of the data's sufficient statistics and the expected sufficient
statistics under the model (Bickel and Doksum 1977). Because
fully-observed Markov networks, such as the ones analyzed here, have
unimodal likelihood functions (Murphy 2012), it is possible to find the
global optimum by iteratively reducing the discrepancies between the
sufficient statistics of the model and of the data to zero.

In order to reduce the discrepancy between the observed and predicted
sufficient statistics, the \texttt{rosalia} calculates each value
exactly, averaging over all possible presence-absence combinations.
Stochastic approximation (Robbins and Monro 1951; Salakhutdinov and
Hinton 2012) instead estimates the expected values of the sufficient
statistics by averaging over a more manageable number of simulated
assemblages during each model-fitting iteration, while still retaining
maximum likelihood convergence guarantees. The procedure iterates
through the following three steps as many times as needed (50,000 for
these analyses; see Appendix for annotated R code):

1: simulate a set of assemblages from the current model parameters and
calculate sufficient statistics for the sample.

2: subtract the simulated sufficient statistics from the observed ones
to calculate the approximate likelihood gradient

3: Adjust the model parameters to climb the approximate gradient, using
a schedule of step sizes that satisfies the criteria in Chapter 6 of
Powell (2007).

Here, the simulations in Step 1 used Gibbs sampling to generate examples
of landscapes based on the model's current parameter estimates. While
the simulated landscapes produced by Gibbs sampling are serially
autocorrelated, statisticians have shown that this merely slows
convergence to the maximum likelihood estimate rather than preventing it
altogether (Younes 1999; Salakhutdinov and Hinton 2012).

The approximate likelihood gradients in Step 2 match the ones from D. J.
Harris (2015), except that they are averaged over a set of Monte Carlo
samples rather than over all possible presence-absence combinations.
These gradients were augmented with a momentum term (Hinton 2012) and by
regularizers based on a logistic prior with location 0 and scale 2.0
(for environmental responses) or 0.5 (for pairwise interactions).

The step size parameter in Step 3, \(\alpha_t\), decreased after each
iteration according to a generalized harmonic sequence,
\(\alpha_t = \alpha_0 1000/(999 + t)\), which satisfies the criteria in
Powell (2007). The scaling factor, \(\alpha_0\), was 1.0 for species'
intercepts and environmental responses. Because of the enormous number
of pairwise interaction coefficients, I set \(\alpha_0\) to 0.1 for the
\(\beta\) parameters.

\subsection{Simulated landscapes with known
interactions}\label{simulated-landscapes-with-known-interactions}

In order to assess the models' ability to recover the ``true''
parameters that generated a presence-absence matrix of interest, I first
needed to generate such matrices from known processes.

To demonstrate that my stochastic approximation implementation could
converge to the global optimum, I first simulated a landscape with 20
species and 500 sites, using Gibbs sampling. These landscapes had few
enough species that the \texttt{rosalia} package for estimating Markov
networks (David J. Harris 2015) could find the penalized maximum
likelihood estimates for the Markov network in a reasonable amount of
time. For each of these simulated landscapes, I then fit the same Markov
network model to the simulated data using the exact approach from D. J.
Harris (2015) and the stochastic approximation method described above.

I then simulated a large landscape with 250 species and 2500 sites,
representing a large data set roughly the size of the North American
Breeding Bird Survey. Each site on the landscape was represented by 5
environmental variables, which were drawn independently from Gaussian
distributions with mean 0 and standard deviation 2.5.

Each species was assigned two sets of coefficients. The coefficients
determining species' responses to the environment were each drawn from
standard normal distributions. The coefficients describing species'
pairwise interactions were drawn from a mixture of normals (see
Appendix) so that most interactions were weak and negative, but a few
strong positive and interactions also occurred.

The coefficients described above are sufficient to define each species'
conditional occurrence probability (i.e.~its probability of occurrence
against any given backdrop of biotic and abiotic features). I used these
conditional probabilities to produce examples of communities that were
consistent with the competition parameters and with the local
environmental variables via Markov chain Monte Carlo (specifically, 1000
rounds of Gibbs sampling; see code in the Appendix). I then used the
methods from the next section to attempt to infer the underlying
parameters from the simulated data.

\section{Results}\label{results}

\begin{figure}[htbp]
\centering
\includegraphics{convergence.pdf}
\caption{Stochastic approximation improves the model over time.
\textbf{A.} With small networks where the exact maximum likelihood
estimates are known, stochastic approximation approaches the global
optimum very quickly (note the log scale of the y axis). Within a few
seconds, the mean squared deviation between the approximate and exact
estimates drops to less than 0.01; in less than a minute, it drops below
0.0001. For comparison, the \texttt{rosalia} package took about six
minutes to find the maximum likelihood estimate. \textbf{B.} With larger
networks, where the maximum likelihood estimate cannot be calculated
exactly, stochastic approximation converges to an apparent optimum that
explains most (but not all) of the variation in the ``true'' parameter
values.}
\end{figure}

For the smaller communities, the squared deviations between the exact
estimates produced by the \texttt{rosalia} package and the ones produced
by stochastic approximation quickly decayed to negligible levels (Figure
2A), indicating that the stochastic approximation procedure was
implemented correctly and worked as the mathematical theory predicts.

\begin{figure}[htbp]
\centering
\includegraphics{estimates.pdf}
\caption{``True'' versus estimated network parameters for all 250
species' responses to the other 249 species in the data set (\textbf{A.}
\(R^2 = .71\)) and to the 5 abiotic variables (\textbf{B.}
\(R^2 = .95\)).}
\end{figure}

For the larger landscape, I found that the stochastic approximation
approach achieved reasonably good performance after ten minutes of
optimization (Figure 2B). After 50,000 iterations (about 5.4 hours on my
laptop), the model was able to recover more than two thirds of the
variance in species' pairwise interactions(\(R^2 = 0.71\); Figure 3A),
and nearly all of the variance in their responses to environmental
variables (\(R^2 = 0.95\); Figure 3B).

\section{Discussion}\label{discussion}

For decades, ecologists have relied on poor test statistics for
inferring species interactions from observational data (D. J. Harris
2015). As shown here and in D. J. Harris (2015), however, these
inferences can only be made reliably by methods that control for other
species in the network, such as partial correlations or Markov networks.
The biggest computational problem with large Markov networks is the
impossibility of averaging over all their possible states, but these
results demonstrate that this step can be avoided in ecological
analyses. Stochastic approximation is able to recover the same estimates
as exact methods for smaller networks, while scaling gracefully to
networks with hundreds of species.

The largest downside of using stochastic approximation instead of the
exact methods introduced in D. J. Harris (2015) is the difficulty in
generating confidence intervals. The \texttt{rosalia} package can
produce confidence intervals based on the Hessian matrix, but it is not
feasible to calculate this matrix for large networks. It may turn out
that the best way to generate confidence intervals for large networks is
by repeatedly fitting the model to bootstrapped samples of the data.
Alternatively, ecologists may be able to take advantage of the advanced
Monte Carlo techniques introduced in Murray, Ghahramani, and MacKay
(2012) to sample from the posterior distribution over possible parameter
values, which could be used to generate credible intervals for the
parameter estimates. Other advanced Monte Carlo methods (e.g.
Salakhutdinov 2009) may also speed up convergence to the maximum
likelihood estimate in cases where serial autocorrelation in the Gibbs
sampler is too strong for effectively sampling the space of possible
landscapes.

As the number of potentially-interacting species increases, the number
of adjustable parameters increases even faster, so overfitting becomes a
major concern. Fortunately, a number of good regularizers have been
proposed. Of these, some of the most interesting options include \(L_1\)
regularization, which ensures that many of the estimated coefficients
are exactly zero (Lee and Hastie 2012). This sparsity matches
ecologists' intuition that many species from different guilds are
unlikely to interact much at all (Faisal et al. 2010), and also produces
computational benefits for model estimation. Of course, the best
regularizers will take advantage of specific ecological knowledge
(e.g.~from field experiments, natural history, or trait data) to provide
information about individual pairwise interactions, rather than about
their overall distribution.

These richer sources of information will be especially important for
cases where ecologists expect that the interactions between species are
asymmetric (e.g.~where only one species is affected by an interaction or
where one species benefits at a cost to the other). With snapshot
observations of species' spatial associations, as discussed here,
interactions must be reduced to a single number describing the net
association, as in undirected models such as the Markov networks
presented here (Schmidt and Murphy 2012). With richer data types,
however, ecologists could more easily learn about asymmetric
interactions such as facilitation, predation, and parasitism.

Even without better data, ecologists have a number of options for
expanding beyond simple Markov networks in a number of ways that would
improve their ability to address a wider range of questions. This paper
demonstrated that it is possible to simultaneously estimate species'
responses to the abiotic and environment and to one another, but many
other extensions are possible. For example, ecologists could condition
the model on variables whose values have not been measured
(e.g.~partially-observed Markov networks, or the approximate networks in
the \texttt{mistnet} package; cf. Pollock et al. (2014)). This would
allow ecologists to account for measurement error and for other
ecologically-important factors that can be difficult to measure.
Ecologists should also explore higher-order networks, where one species'
presence can affect the relationship between two other species (Whittam
and Siegel-Causey 1981; Tjelmeland and Besag 1998).

\section*{References}\label{references}
\addcontentsline{toc}{section}{References}

Azaele, S, R Muneepeerakul, A Rinaldo, and I Rodriguez-Iturbe. 2010.
``Inferring Plant Ecosystem Organization from Species Occurrences.''
\emph{Journal of Theoretical Biology} 262 (2): 323--29.

Bickel, PJ, and K Doksum. 1977. \emph{Mathematical Statistics: Basic
Ideas and Selected Topics}. San Francisco: Holden---Day.

Cipra, Barry A. 1987. ``An Introduction to the Ising Model.''
\emph{American Mathematical Monthly} 94 (10): 937--59.

Connor, Edward F, and Daniel Simberloff. 1979. ``The Assembly of Species
Communities: Chance or Competition?'' \emph{Ecology}, 1132--40.

Faisal, Ali, Frank Dondelinger, Dirk Husmeier, and Colin M. Beale. 2010.
``Inferring Species Interaction Networks from Species Abundance Data: A
Comparative Evaluation of Various Statistical and Machine Learning
Methods.'' \emph{Ecological Informatics} 5 (6): 451--64.
doi:\href{http://dx.doi.org/http://dx.doi.org/10.1016/j.ecoinf.2010.06.005}{http://dx.doi.org/10.1016/j.ecoinf.2010.06.005}.

Friedman, Jerome, Trevor Hastie, and Robert Tibshirani. 2008. ``Sparse
Inverse Covariance Estimation with the Graphical Lasso.''
\emph{Biostatistics} 9 (3): 432--41.
doi:\href{http://dx.doi.org/10.1093/biostatistics/kxm045}{10.1093/biostatistics/kxm045}.

Gilpin, Michael E., and Jared M. Diamond. 1982. ``Factors Contributing
to Non-Randomness in Species Co-Occurrences on Islands.''
\emph{Oecologia} 52 (1): 75--84.
doi:\href{http://dx.doi.org/10.1007/BF00349014}{10.1007/BF00349014}.

Gotelli, Nicholas J., and Werner Ulrich. 2009. ``The Empirical Bayes
Approach as a Tool to Identify Non-Random Species Associations.''
\emph{Oecologia} 162 (2): 463--77.
doi:\href{http://dx.doi.org/10.1007/s00442-009-1474-y}{10.1007/s00442-009-1474-y}.

Harris, D. J. 2015. ``Estimating Species Interactions from Observational
Data with Markov Networks.'' \emph{BioRxiv}.
doi:\href{http://dx.doi.org/10.1101/018861}{10.1101/018861}.

Harris, David J. 2015. \emph{Rosalia: Exact Inference for Small Binary
Markov Networks. R Package Version 0.1.0}. Zenodo.
http://dx.doi.org/10.5281/zenodo.17808.
\url{http://dx.doi.org/10.5281/zenodo.17808}.

Hastings, Alan. 1987. ``Can Competition Be Detected Using Species
Co-Occurrence Data?'' \emph{Ecology}, 117--23.

Hinton, Geoffrey E. 2012. ``A Practical Guide to Training Restricted
Boltzmann Machines.'' In \emph{Neural Networks: Tricks of the Trade},
599--619. Springer.

Kissling, W. D., Carsten F. Dormann, Jürgen Groeneveld, Thomas Hickler,
Ingolf Kühn, Greg J. McInerny, José M. Montoya, et al. 2012. ``Towards
Novel Approaches to Modelling Biotic Interactions in Multispecies
Assemblages at Large Spatial Extents.'' \emph{Journal of Biogeography}
39 (12): 2163--78.
doi:\href{http://dx.doi.org/10.1111/j.1365-2699.2011.02663.x}{10.1111/j.1365-2699.2011.02663.x}.

Lee, Jason D., and Trevor J. Hastie. 2012. ``Learning Mixed Graphical
Models.'' \emph{CERN Document Server}.
\url{http://cds.cern.ch/record/1451206}.

Lessard, Jean-Philippe, Michael K. Borregaard, James A. Fordyce, Carsten
Rahbek, Michael D. Weiser, Robert R. Dunn, and Nathan J. Sanders. 2011.
``Strong Influence of Regional Species Pools on Continent-Wide
Structuring of Local Communities.'' \emph{Proceedings of the Royal
Society of London B: Biological Sciences}.
doi:\href{http://dx.doi.org/10.1098/rspb.2011.0552}{10.1098/rspb.2011.0552}.

Lewin, Roger. 1983. ``Santa Rosalia Was a Goat.'' \emph{Science} 221
(4611): 636--39.

Murphy, Kevin P. 2012. \emph{Machine Learning: A Probabilistic
Perspective}. The MIT Press.

Murray, Iain, Zoubin Ghahramani, and David MacKay. 2012. ``MCMC for
Doubly-Intractable Distributions.'' \emph{ArXiv Preprint
ArXiv:1206.6848}.

Pollock, Laura J., Reid Tingley, William K. Morris, Nick Golding, Robert
B. O'Hara, Kirsten M. Parris, Peter A. Vesk, and Michael A. McCarthy.
2014. ``Understanding Co-Occurrence by Modelling Species Simultaneously
with a Joint Species Distribution Model (JSDM).'' \emph{Methods in
Ecology and Evolution}, n/a--/a.
doi:\href{http://dx.doi.org/10.1111/2041-210X.12180}{10.1111/2041-210X.12180}.

Powell, Warren B. 2007. \emph{Approximate Dynamic Programming: Solving
the Curses of Dimensionality}. John Wiley \& Sons.

Robbins, Herbert, and Sutton Monro. 1951. ``A Stochastic Approximation
Method.'' \emph{The Annals of Mathematical Statistics} 22 (3): 400--407.
doi:\href{http://dx.doi.org/10.1214/aoms/1177729586}{10.1214/aoms/1177729586}.

Salakhutdinov, Ruslan. 2009. ``Learning in Markov Random Fields Using
Tempered Transitions.'' \emph{Advances in Neural Information Processing
Systems} 22: 1598--1606.
\url{http://citeseerx.ist.psu.edu/viewdoc/download?doi=10.1.1.163.9244\&rep=rep1\&type=pdf}.

Salakhutdinov, Ruslan, and Geoffrey Hinton. 2012. ``An Efficient
Learning Procedure for Deep Boltzmann Machines.'' \emph{Neural
Computation} 24 (8): 1967--2006.
\url{http://www.mitpressjournals.org/doi/pdf/10.1162/NECO_a_00311}.

Schmidt, Mark, and Kevin Murphy. 2012. ``Modeling Discrete
Interventional Data Using Directed Cyclic Graphical Models.''
\emph{ArXiv Preprint ArXiv:1205.2617}.

Strong, Donald R, Daniel Simberloff, Lawrence G Abele, and Anne B
Thistle. 1984. \emph{Ecological Communities: Conceptual Issues and the
Evidence}. Princeton University Press.

Tjelmeland, Håkon, and Julian Besag. 1998. ``Markov Random Fields with
Higher-Order Interactions.'' \emph{Scandinavian Journal of Statistics}
25 (3): 415--33.
doi:\href{http://dx.doi.org/10.1111/1467-9469.00113}{10.1111/1467-9469.00113}.

Veech, Joseph A. 2013. ``A Probabilistic Model for Analysing Species
Co-Occurrence.'' \emph{Global Ecology and Biogeography} 22 (2): 252--60.
doi:\href{http://dx.doi.org/10.1111/j.1466-8238.2012.00789.x}{10.1111/j.1466-8238.2012.00789.x}.

Whittam, Thomas S., and Douglas Siegel-Causey. 1981. ``Species
Interactions and Community Structure in Alaskan Seabird Colonies.''
\emph{Ecology} 62 (6): 1515--24.
doi:\href{http://dx.doi.org/10.2307/1941508}{10.2307/1941508}.

Younes, Laurent. 1999. ``On the Convergence of Markovian Stochastic
Algorithms with Rapidly Decreasing Ergodicity Rates.''
\emph{Stochastics: An International Journal of Probability and
Stochastic Processes} 65 (3-4): 177--228.
